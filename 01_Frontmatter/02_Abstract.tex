\textbf{Abstract Deutsch}\\
Ziel des Projektes \emph{Solar Butterfly} ist die Entwicklung eines autarken Wohnwagens, welcher sich mit selbst erzeugten Solarstrom versorgen und autonom operiert werden kann. Der Solar Butterfly soll international Aufmerksamkeit erregen und so nachhaltige Lösungen im Bereich des Klimaschutzes und Elektromobilität ermutigen und vorantreiben. In Zusammenarbeit mit drei weiteren Maschinentechnikstudenten und deren Bachelor-Thesen soll die Vision des Solar Butterflys in die Realität umgesetzt werden.\\
Diese Arbeit befasst sich mit dem Definieren der Anforderungen und Auslegungskriterien des Solar Butterflys, dem Bestimmen von Design-Allowables, der Ausarbeitung eines Lastenheftes und der Grobauslegung der Grundstruktur. Zur Bestimmung von Schnittgrössen soll dabei ein globales FEM-Modell zur Anwendung kommen.\\
Handrechnungen und FEM-Berechnungen zeigen, dass von den untersuchten Belastungen die Lastfälle der vertikalen und rotatorischen Beschleunigung, welche während der Fahrt auftreten, die grössten Beanspruchungen darstellen. Zugleich weisen diese Lastfälle aufgrund von nur bedingt abschätzbaren Randbedingungen die grössten Unsicherheiten und Risiken auf. Weiter konnte in Erfahrung gebracht werden, dass die Klebeverbindung zwischen dem Boden und Chassis als Kritisch zu beurteilen ist und dass weitere Untersuchungen und Abklärungen diesbezüglich nötig sind.
Ferner konnte Potential zur Gewichtsreduktion in Form einer Optimierung des Chassis in Verbindung mit dem Boden ausfindig gemacht werden.


\textbf{Abstract Englisch}\\
The goal of the project \emph{Solar Butterfly} is the development of a self-sufficient caravan, which can supply itself with self-generated solar power and be operated autonomously. The Solar Butterfly is intended to draw international attention and thus encourage and promote sustainable solutions in the field of climate protection and electromobility. In collaboration with three other mechanical engineering students and their bachelor thesis, the vision of the Solar Butterfly is to be turned into reality.\\
This thesis deals with the definition of the requirements and design criteria of the Solar Butterfly, the determination of design allowables, the elaboration of a specification sheet and the rough dimensioning of the basic structure. A global FEM model is to be used to determine cutting forces.\\
Manual calculations and FEM simulations show that of the loads investigated, the load cases of vertical and rotational acceleration, which occur while the vehicle is moving, represent the greatest stresses. At the same time, these load cases show the greatest uncertainties and risks due to boundary conditions which can only be estimated to a limited extent. It was also found that the adhesive bond between the floor and the chassis is critical and that further investigations and clarifications are necessary in this respect.\\
Furthermore, potential for weight reduction in the form of an optimization of the chassis in conjunction with the floor was identified.

\vspace{2cm}
Ort, Datum $\;\;\;\;\;\;\;\;\;\;\;\;\;\;\;\;\;\;\;\;$ Luzern, 11. Juni 2021\\
\textbf{{\small $^\copyright$} Andre Gut, Hochschule Luzern - Technik \& Architektur}

\vspace*{\fill}

\noindent
{\color{gray} \rule{\linewidth}{0.5px} }
\begin{footnotesize}
  \textcolor{gray}{Alle Rechte vorbehalten. Die Arbeit oder Teile davon dürfen ohne schriftliche Genehmigung der Rechteinhaber weder in irgendeiner Form reproduziert noch elektronisch gespeichert, verarbeitet, vervielfältigt oder verbreitet werden.}\\
  \textcolor{gray}{Sofern die Arbeit auf der Website der Hochschule Luzern online veröffentlicht wird, können abweichende Nutzungsbedingungen unter Creative-Commons-Lizenzen gelten. Massgebend ist in diesem Fall die auf der Website angezeigte Creative-Commons-Lizenz.}
\end{footnotesize}
\newpage
