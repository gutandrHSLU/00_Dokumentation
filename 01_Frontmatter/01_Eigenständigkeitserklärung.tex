\vspace{2cm}
\begin{large}
\textbf{Bachelor-Thesis an der Hochschule Luzern - Technik \& Architektur}\\
\end{large}
\vspace{1cm}

\begin{table}[H]
\small
  \begin{tabularx}{\linewidth}{llX}
    \textbf{Titel}                 & \textbf{Solar Butterfly - Auslegung Grundstruktur} &\\[4 mm]
    \textbf{Diplomandin/Diplomand} & \textbf{Gut, Andre}                                &\\[4 mm]
    \textbf{Bachelor-Studiengang}  & \textbf{Bachelor Maschinentechnik}                 &\\[4 mm]
    \textbf{Semester}              & \textbf{FS21}                                      &\\[4 mm]
    \textbf{Dozentin/Dozent}       & \textbf{Roman\v{c}uk, Dejan}                       &\\[4 mm]
    \textbf{Expertin/Experte}      & \textbf{Dubach, Roger}                             &
  \end{tabularx}
\end{table}

\vspace{1.5cm}
\textbf{Abstract Deutsch}\\
Ziel des Projektes \emph{Solar Butterfly} ist die Entwicklung eines autarken Wohnwagens, welcher sich mit selbst erzeugten Solarstrom versorgen und autonom operiert werden kann. Der Solar Butterfly soll international Aufmerksamkeit erregen und so nachhaltige Lösungen im Bereich des Klimaschutzes und Elektromobilität ermutigen und vorantreiben. In Zusammenarbeit mit drei weiteren Maschinenbaustudenten und deren Bachelorarbeiten soll die Vision des Solar Butterflys in die Realität umgesetzt werden.\\
Diese Arbeit befasst sich mit dem Definieren der Anforderungen und Auslegungskriterien des Solar Butterflys, dem Bestimmen von Design-Allowables, der Ausarbeitung eines Lastenheftes und der Grobauslegung der Grundstruktur. Zur Bestimmung von Schnittgrössen soll dabei ein globales FEM-Modell zur Anwendung kommen.\\
Handrechnungen und FEM-Berechnungen zeigen, dass von den untersuchten Belastungen die Lastfälle der vertikalen und rotatorischen Beschleunigung, welche während der Fahrt auftreten, die grössten Beanspruchungen darstellen. Zugleich weisen diese Lastfälle aufgrund von nur bedingt abschätzbaren Randbedingungen die grössten Unsicherheiten und Risiken auf. Weiter konnte in Erfahrung gebracht werden, dass die Klebeverbindung zwischen dem Boden und Chassis als Kritisch zu beurteilen ist und dass weitere Untersuchungen und Abklärungen diesbezüglich nötig sind.
Ferner konnte Potential zur Gewichtsreduktion in Form einer Optimierung des Chassis in Verbindung mit dem Boden ausfindig gemacht werden.


\textbf{Abstract Englisch}\\
The goal of the project \emph{Solar Butterfly} is the development of a self-sufficient caravan, which can be powered by self-generated solar electricity and operate autonomously. The Solar Butterfly is intended to attract international attention and thus encourage and promote sustainable solutions in the field of climate protection and electromobility. In collaboration with three
other mechanical engineering students and their bachelor thesis, the vision of the Solar Butterfly is to be turned into reality.
This work deals with the definition of the requirements and design criteria for the Solar Butterfly, the specification of design allowables, the elaboration of a specification sheet and the rough design of the basic structure. A global FEM model will be used to determine the sectional forces.
Hand calculations and FEM calculations show that of the loads investigated, the load cases of vertical and rotational acceleration, which occur during driving, represent the greatest stresses. It was also found that the adhesive bond between the floor and the chassis is critical and that further investigations and clarifications are necessary. Furthermore, potential for weight reduction could be identified by optimizing the chassis in conjunction with the floor.

\vspace{2cm}
Ort, Datum $\;\;\;\;\;\;\;\;\;\;\;\;\;\;\;\;\;\;\;\;$ Luzern, 11. Juni 2021\\
\textbf{{\small $^\copyright$} Andre Gut, Hochschule Luzern - Technik \& Architektur}

\vspace*{\fill}

\noindent
{\color{gray} \rule{\linewidth}{0.5px} }
\begin{footnotesize}
  \textcolor{gray}{Alle Rechte vorbehalten. Die Arbeit oder Teile davon dürfen ohne schriftliche Genehmigung der Rechteinhaber weder in irgendeiner Form reproduziert noch elektronisch gespeichert, verarbeitet, vervielfältigt oder verbreitet werden.}\\
  \textcolor{gray}{Sofern die Arbeit auf der Website der Hochschule Luzern online veröffentlicht wird, können abweichende Nutzungsbedingungen unter Creative-Commons-Lizenzen gelten. Massgebend ist in diesem Fall die auf der Website angezeigte Creative-Commons-Lizenz.}
\end{footnotesize}
\newpage
