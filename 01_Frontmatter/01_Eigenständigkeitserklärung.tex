\vspace{2cm}
\begin{large}
\textbf{Bachelor-Thesis an der Hochschule Luzern - Technik \& Architektur}\\
\end{large}
\vspace{1cm}

\begin{table}[H]
\small
  \begin{tabularx}{\linewidth}{llX}
    \textbf{Titel}                 & \textbf{Solar Butterfly - Auslegung Grundstruktur} &\\[4 mm]
    \textbf{Diplomandin/Diplomand} & \textbf{Gut, Andre}                                &\\[4 mm]
    \textbf{Bachelor-Studiengang}  & \textbf{Bachelor Maschinentechnik}                 &\\[4 mm]
    \textbf{Semester}              & \textbf{FS21}                                      &\\[4 mm]
    \textbf{Dozentin/Dozent}       & \textbf{Roman\v{c}uk, Dejan}                       &\\[4 mm]
    \textbf{Expertin/Experte}      & \textbf{Dubach, Roger}                             &
  \end{tabularx}
\end{table}

\vspace{1.5cm}
\textbf{Abstract Deutsch}\\
Die Anforderung an die Langlebigkeit des Solar Butterflys wird erfüllt, indem der Solar Butterfly Dauerfest ausgelegt wird. Für die Grobauslegung bedeutet dies konkret, dass die Ermüdung mit einer entsprechenden Wahl der Design-Allowables pauschal abgedeckt wird und dass Spannungserhöhungen mit gutem Design vermieden werden.\\
Als Dauerbelastung wird vereinfacht angenommen, dass die maximalen Lasten zu 50\% (die Hälfte der Amplitude) dauerhaft auftreten. Dies bedeutet, anderst formuliert, dass angenommen wird, dass die im Lastenheft definierten maximalen Lasten selten zu 50\% erreicht werden und dass diese somit keine Gefahr für das Versagen der Bauteile durch Ermüdung darstellen.\\
Ob dies eine angemessene Annahme ist, muss zu einem späteren Zeitpunkt und bei einem weiter fortgeschrittenen Projekstand, durch Erbringung eines Nachweises der Dauerfestigkeit, überprüft werden.\\

\textbf{Abstract Englisch}\\
Die Anforderung an die Langlebigkeit des Solar Butterflys wird erfüllt, indem der Solar Butterfly Dauerfest ausgelegt wird. Für die Grobauslegung bedeutet dies konkret, dass die Ermüdung mit einer entsprechenden Wahl der Design-Allowables pauschal abgedeckt wird und dass Spannungserhöhungen mit gutem Design vermieden werden.\\
Als Dauerbelastung wird vereinfacht angenommen, dass die maximalen Lasten zu 50\% (die Hälfte der Amplitude) dauerhaft auftreten. Dies bedeutet, anderst formuliert, dass angenommen wird, dass die im Lastenheft definierten maximalen Lasten selten zu 50\% erreicht werden und dass diese somit keine Gefahr für das Versagen der Bauteile durch Ermüdung darstellen.\\
Ob dies eine angemessene Annahme ist, muss zu einem späteren Zeitpunkt und bei einem weiter fortgeschrittenen Projekstand, durch Erbringung eines Nachweises der Dauerfestigkeit, überprüft werden.\\
Die Anforderung an die Langlebigkeit des Solar Butterflys wird erfüllt, indem der Solar Butterfly Dauerfest ausgelegt wird. Für die Grobauslegung bedeutet dies konkret, dass die Ermüdung mit einer entsprechenden Wahl der Design-Allowables pauschal abgedeckt wird und dass Spannungserhöhungen mit gutem Design vermieden werden.\\
Als Dauerbelastung wird vereinfacht angenommen, dass die maximalen Lasten zu 50\% (die Hälfte der Amplitude) dauerhaft auftreten. Dies bedeutet, anderst formuliert, dass angenommen wird, dass die im Lastenheft definierten maximalen Lasten selten zu 50\% erreicht werden und dass diese somit keine Gefahr für das Versagen der Bauteile durch Ermüdung darstellen.\\
Ob dies eine angemessene Annahme ist, muss zu einem späteren Zeitpunkt und bei einem weiter fortgeschrittenen Projekstand, durch Erbringung eines Nachweises der Dauerfestigkeit, überprüft werden.\\

\vspace{2cm}
Ort, Datum $\;\;\;\;\;\;\;\;\;\;\;\;\;\;\;\;\;\;\;\;$ Luzern, 11. Juni 2021\\
\textbf{{\small $\copyright$} Andre, Gut, Hochschule Luzern - Technik \& Architektur}

\vspace*{\fill}

\noindent
{\color{gray} \rule{\linewidth}{0.5px} }
\begin{footnotesize}
  \textcolor{gray}{Alle Rechte vorbehalten. Die Arbeit oder Teile davon dürfen ohne schriftliche Genehmigung der Rechteinhaber weder in irgendeiner Form reproduziert noch elektronisch gespeichert, verarbeitet, vervielfältigt oder verbreitet werden.}

  \textcolor{gray}{Sofern die Arbeit auf der Website der Hochschule Luzern online veröffentlicht wird, können abweichende Nutzungsbedingungen unter Creative-Commons-Lizenzen gelten. Massgebend ist in diesem Fall die auf der Website angezeigte Creative-Commons-Lizenz.}
\end{footnotesize}
\newpage
