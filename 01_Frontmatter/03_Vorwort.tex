\section*{Vorwort}
Diese Bachelor-Thesis wird als abschliessendes Projekt im Rahmen des Maschinentechnikstudiums an der Hochschule Luzern - Technik \& Architektur durchgeführt.

Die Wahl des Projektes ist auf dieses gefallen, da eine grosse und fordernde Problemstellung mit intensiver Teamarbeit erwartet wurde, was sich inzwischen mehr als bewahrheitet hat.\\
An dem Projekt \emph{Solar Butterfly} sind drei weitere Maschinentechnikstudenten mit ihrer Bachelor-Thesis beteiligt. Jedem der Studenten steht ein Dozent zur Betreuung und Unterstützung zur Verfügung. Mit den insgesamt neun Parteien war entsprechend viel Organisation, Austausch und Kommunikation nötig, was durch das Abhalten von  Wöchentliche Meetings, abwechslungsweise über Videokonferenz oder an der HSLU vor Ort, erreicht wurde.

Ohne die Unterstützung folgender Personen wäre mir das Ausarbeiten der Bachelor-Thesis in dieser Form nicht möglich gewesen. Dafür möchte ich an dieser Stelle meinen Dank aussprechen an:

Dejan Roman\v{c}uk, für die Betreuung und Unterstützung bei der Durchführung dieser Arbeit,\\
Louis Palmer für die Organisation und intensive Zusammenarbeit,\\
den Mitstudenten Michael Huber, Dominic Bacher und Yannick Buholzer für die angenehme, produktive und kurzweilige Teamarbeit und\\
den weiteren betreuenden Dozenten Pierre Kirchhofer, Rolf Kamps und Johann Lodewyks für ihre Unterstützung und hilfreichen Inputs.
\newpage
