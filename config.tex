% ========== Packages ==========
\usepackage[a4paper,
  left=25mm,
  right=25mm,
  top=30mm,
  bottom=35mm,
  headheight=35mm
]{geometry}

\usepackage[ngerman]{babel} %Ändert die Sprache
\usepackage[T1]{fontenc} %Wichtig für ä ö ü
\usepackage{amssymb} %Für mathematische Zeichen
\usepackage{amsthm} %Für mathematische Umgebungen
\usepackage{graphicx}
\usepackage{fancyhdr}
\usepackage[utf8]{inputenc}
\usepackage{multirow} %Für Tabellen
\usepackage{longtable} %Für lange Tabellen
\usepackage{tabularx}
\usepackage{pdfpages} %Zum einfügen von PDF's
\usepackage{hyperref} %Für hyperlinks
\hypersetup{bookmarks=true}
\usepackage{parskip}
\usepackage{caption} %Für die Beschriftung von Bilder
\captionsetup{justification=centering}
\captionsetup{font=it}
\setlength{\parindent}{0pt}
\usepackage{subcaption} %Für die Beschriftung unterteilter Bilder
\usepackage{float}
\floatstyle{plaintop}
\restylefloat{table}

\makeatletter %Für römische Zahlen
\newcommand*{\rom}[1]{\expandafter\@slowromancap\romannumeral #1@}

%Für die dicken Linien in Tabellen
\def\thickhline{%
  \noalign{\ifnum0=`}\fi\hrule \@height \thickarrayrulewidth \futurelet
   \reserved@a\@xthickhline}
\def\@xthickhline{\ifx\reserved@a\thickhline
               \vskip\doublerulesep
               \vskip-\thickarrayrulewidth
               \fi
      \ifnum0=`{\fi}}
\makeatother
\newlength{\thickarrayrulewidth}
\setlength{\thickarrayrulewidth}{2\arrayrulewidth}

%Sorgt dafür, dass nicht immer alles auf die ganze Seite verteilt wird.
\raggedbottom

% ========== Header and Footer ==========
\pagestyle{fancy}
\fancyhf{}
\fancyhead[RE,LO]{Seite \thepage}
\fancyhead[LE,RO]{\nouppercase{\leftmark}}
\fancyfoot[RE,LO]{BAT FS21}

%Eigens erstellte Variablen
\newcommand{\plotWidth}{0.7}
\newcommand{\garphWidth}{0.7}
