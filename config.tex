% ========== Packages ==========
\usepackage[a4paper,
  left=25mm,
  right=25mm,
  top=30mm,
  bottom=35mm,
  headheight=35mm
]{geometry}

\usepackage[ngerman]{babel} %Ändert die Sprache
\usepackage[T1]{fontenc} %Wichtig für ä ö ü
\usepackage{amssymb} %Für mathematische Zeichen
\usepackage{amsthm} %Für mathematische Umgebungen
\usepackage{graphicx}
\usepackage{wrapfig}% Zum Bilder im Text ein zu betten
\usepackage{fancyhdr}
\usepackage[utf8]{inputenc}
\usepackage{multirow} %Für Tabellen
\usepackage{multicol} %Für zwei Formeln nebeneinander
\usepackage{amsmath}
\usepackage{longtable} %Für lange Tabellen
\usepackage{tabularx}
\usepackage{arydshln} %Für gestrichelte Linien in Tabellen
\usepackage{booktabs}
\usepackage{pdfpages} %Zum einfügen von PDF's
\usepackage{hyperref} %Für hyperlinks
\hypersetup{bookmarks=true}
\usepackage{parskip}
\usepackage{caption} %Für die Beschriftung von Bilder
\captionsetup{justification=centering}
\captionsetup{font=it}
\setlength{\parindent}{0pt}
\usepackage{subcaption} %Für die Beschriftung unterteilter Bilder
\usepackage{lscape}
\usepackage{float}
\floatstyle{plaintop}
\restylefloat{table}
\usepackage{siunitx}%Für einheiten im Symbolverzeichnis
% \usepackage[symbols,nogroupskip,sort=none]{glossaries-extra}%Für Symbolverzeichnis
% % % \nomenclature{$c$}{Speed of light in a vacuum inertial system \nomunit{$299,792,458\, m/s$}}

% makeindex main.nlo -s nomencl.ist -o main.nls
\nomenclature{$N$}{The number of angels per needle point}%
\nomenclature{$B$}{The area of the needle point}%

\nomenclature{$t_k$}{Höhe des Schaumkernes}
\nomenclature{$E_k$}{E-Modul der Kernschicht}
\nomenclature{$G_k$}{Schub-Modul des Schaumkernes}
\nomenclature{$t_d$}{Höhe der Deckschicht}
\nomenclature{$E_d$}{E-Modul der Deckschicht}
\nomenclature{$h$}{Abstand der neutralen Fasern der Deckschichten}
\nomenclature{$w_b$}{Verformung durch Biegegelastung}
\nomenclature{$w_s$}{Verformung durch Schubbelastung}
\nomenclature{$w_{Ges}$}{Gesammtverformung}

\nomenclature{$P_kB$}{Euler-Knicklast des schubsteifen Balkens}
\nomenclature{$P_kS$}{Schubknicklast}
\nomenclature{$P_k$}{Kritische Knicklast}

\nomenclature{$n$}{Normalkraft pro Länge \nomunit{$N/mm$}}
\nomenclature{$m$}{Moment pro Länge \nomunit{$N$}}
\nomenclature{$q$}{Schubkraft Pro Länge \nomunit{$N/mm$}}

\nomenclature{$p$}{Streckenlast \nomunit{$N/mm^2$}}

\usepackage{nomencl}
\renewcommand{\nomname}{List of Symbols}
\newcommand{\nomunit}[1]{%
\renewcommand{\nomentryend}{\hspace*{\fill}#1}}
\makenomenclature




\makeatletter %Für römische Zahlen
\newcommand*{\rom}[1]{\expandafter\@slowromancap\romannumeral #1@}

%Für die dicken Linien in Tabellen
\def\thickhline{%
  \noalign{\ifnum0=`}\fi\hrule \@height \thickarrayrulewidth \futurelet
   \reserved@a\@xthickhline}
\def\@xthickhline{\ifx\reserved@a\thickhline
               \vskip\doublerulesep
               \vskip-\thickarrayrulewidth
               \fi
      \ifnum0=`{\fi}}
\makeatother
\newlength{\thickarrayrulewidth}
\setlength{\thickarrayrulewidth}{2\arrayrulewidth}

%Sorgt dafür, dass nicht immer alles auf die ganze Seite verteilt wird.
\raggedbottom

% ========== Header and Footer ==========
\pagestyle{fancy}
\fancyhf{}
\fancyhead[RE,LO]{Seite \thepage}
\fancyhead[LE,RO]{\nouppercase{\leftmark}}
\fancyfoot[RE,LO]{BAT FS21}

%Eigens erstellte Variablen
\newcommand{\plotWidth}{0.7}
\newcommand{\garphWidth}{0.7}
\newcommand{\lineHeightTable}{-0.2em}

\newcolumntype{L}[1]{>{\raggedright\let\newline\\\arraybackslash\hspace{0pt}}m{#1}}
