\documentclass[twoside, 11pt]{article}
% ========== Packages ==========
\usepackage[a4paper,
  left=25mm,
  right=25mm,
  top=30mm,
  bottom=35mm,
  headheight=35mm
]{geometry}

\usepackage[ngerman]{babel} %Ändert die Sprache
\usepackage[T1]{fontenc} %Wichtig für ä ö ü
\usepackage{amssymb} %Für mathematische Zeichen
\usepackage{amsthm} %Für mathematische Umgebungen
\usepackage{graphicx}
\usepackage{fancyhdr}
\usepackage[utf8]{inputenc}
\usepackage{multirow} %Für Tabellen
\usepackage{multicol} %Für zwei Formeln nebeneinander
\usepackage{amsmath}
\usepackage{longtable} %Für lange Tabellen
\usepackage{arydshln} %Für gestrichelte Linien in Tabellen
\usepackage{tabularx}
\usepackage{pdfpages} %Zum einfügen von PDF's
\usepackage{hyperref} %Für hyperlinks
\hypersetup{bookmarks=true}
\usepackage{parskip}
\usepackage{caption} %Für die Beschriftung von Bilder
\captionsetup{justification=centering}
\captionsetup{font=it}
\setlength{\parindent}{0pt}
\usepackage{subcaption} %Für die Beschriftung unterteilter Bilder
\usepackage{float}
\floatstyle{plaintop}
\restylefloat{table}
\usepackage{siunitx}%Für einheiten im Symbolverzeichnis
% \usepackage[symbols,nogroupskip,sort=none]{glossaries-extra}%Für Symbolverzeichnis
% % % makeindex main.nlo -s nomencl.ist -o main.nls
\nomenclature{$t_k$}{Höhe des Schaumkernes}
\nomenclature{$E_k$}{E-Modul der Kernschicht}
\nomenclature{$G_k$}{Schub-Modul des Schaumkernes}
\nomenclature{$t_d$}{Höhe der Deckschicht}
\nomenclature{$E_d$}{E-Modul der Deckschicht}
\nomenclature{$h$}{Abstand der neutralen Fasern der Deckschichten}
\nomenclature{$w_b$}{Verformung durch Biegegelastung\nomunit{$mm$}}
\nomenclature{$w_s$}{Verformung durch Schubbelastung\nomunit{$mm$}}
\nomenclature{$w_{Ges}$}{Gesammtverformung\nomunit{$mm$}}

\nomenclature{$p_kB$}{Euler-Knickstreckenlast des schubsteifen Balkens\nomunit{$N/mm$}}
\nomenclature{$p_kS$}{Schubknicklast\nomunit{$N/mm$}}
\nomenclature{$p_k$}{Kritische Knicklast\nomunit{$N/mm$}}

\nomenclature{$n$}{Normalkraft pro Länge \nomunit{$N/mm$}}
\nomenclature{$m$}{Moment pro Länge \nomunit{$N$}}
\nomenclature{$q$}{Schubkraft Pro Länge \nomunit{$N/mm$}}

\nomenclature{$p$}{Streckenlast \nomunit{$N/mm$}}

\usepackage{nomencl}
\renewcommand{\nomname}{List of Symbols}
\newcommand{\nomunit}[1]{%
\renewcommand{\nomentryend}{\hspace*{\fill}#1}}
\makenomenclature



\makeatletter %Für römische Zahlen
\newcommand*{\rom}[1]{\expandafter\@slowromancap\romannumeral #1@}

%Für die dicken Linien in Tabellen
\def\thickhline{%
  \noalign{\ifnum0=`}\fi\hrule \@height \thickarrayrulewidth \futurelet
   \reserved@a\@xthickhline}
\def\@xthickhline{\ifx\reserved@a\thickhline
               \vskip\doublerulesep
               \vskip-\thickarrayrulewidth
               \fi
      \ifnum0=`{\fi}}
\makeatother
\newlength{\thickarrayrulewidth}
\setlength{\thickarrayrulewidth}{2\arrayrulewidth}

%Sorgt dafür, dass nicht immer alles auf die ganze Seite verteilt wird.
\raggedbottom

% ========== Header and Footer ==========
\pagestyle{fancy}
\fancyhf{}
\fancyhead[RE,LO]{Seite \thepage}
\fancyhead[LE,RO]{\nouppercase{\leftmark}}
\fancyfoot[RE,LO]{KOHEB FS21}

%Eigens erstellte Variablen
\newcommand{\plotWidth}{0.7}
\newcommand{\garphWidth}{0.7}

% % makeindex main.nlo -s nomencl.ist -o main.nls
\nomenclature{$t_k$}{Höhe des Schaumkernes}
\nomenclature{$E_k$}{E-Modul der Kernschicht}
\nomenclature{$G_k$}{Schub-Modul des Schaumkernes}
\nomenclature{$t_d$}{Höhe der Deckschicht}
\nomenclature{$E_d$}{E-Modul der Deckschicht}
\nomenclature{$h$}{Abstand der neutralen Fasern der Deckschichten}
\nomenclature{$w_b$}{Verformung durch Biegegelastung\nomunit{$mm$}}
\nomenclature{$w_s$}{Verformung durch Schubbelastung\nomunit{$mm$}}
\nomenclature{$w_{Ges}$}{Gesammtverformung\nomunit{$mm$}}

\nomenclature{$p_kB$}{Euler-Knickstreckenlast des schubsteifen Balkens\nomunit{$N/mm$}}
\nomenclature{$p_kS$}{Schubknicklast\nomunit{$N/mm$}}
\nomenclature{$p_k$}{Kritische Knicklast\nomunit{$N/mm$}}

\nomenclature{$n$}{Normalkraft pro Länge \nomunit{$N/mm$}}
\nomenclature{$m$}{Moment pro Länge \nomunit{$N$}}
\nomenclature{$q$}{Schubkraft Pro Länge \nomunit{$N/mm$}}

\nomenclature{$p$}{Streckenlast \nomunit{$N/mm$}}



% \usepackage{nomencl}
% \makenomenclature
\begin{document}
% ==================== Titelseite ====================
  \begin{titlepage}
    \begin{center}
        % \vspace*{1cm}

        \LARGE
        Bachelor-Thesis an der Hochschule Luzern\\
        Technik \& Architektur

        \vspace{0.8cm}
        \Huge
        \textbf{Solar Butterfly - Auslegung Grundstruktur}

        \vspace{3cm}

        \begin{center}
          \makebox[\textwidth]{\includegraphics[width=1\paperwidth]{04_Figures/SB0.png}}
        \end{center}

        \vfill
        \begin{table}[b]
        \small
          \begin{tabularx}{\linewidth}{llX}
            \textbf{Diplomandin/Diplomand} & \textbf{Gut, Andre}                                &\\[4 mm]
            \textbf{Bachelor-Studiengang}  & \textbf{Bachelor Maschinentechnik}                 &\\[4 mm]
            \textbf{Semester}              & \textbf{FS21}                                      &\\[4 mm]
            \textbf{Dozentin/Dozent}       & \textbf{Roman\v{c}uk, Dejan}                       &\\[4 mm]
            \textbf{Expertin/Experte}      & \textbf{Dubach, Roger}                             &
          \end{tabularx}
        \end{table}

    \end{center}
\end{titlepage}


% ==================== Frontmatter ====================
  \pagenumbering{roman}
  \setcounter{page}{2}
  \section*{Eigenständigkeitserklärung}

Hiermit erkläre ich, dass ich die vorliegende Arbeit selbständig angefertigt und keine anderen als die angegebenen Hilfsmittel verwendet habe. Sämtliche verwendeten Textausschnitte, Zitate oder Inhalte anderer Verfasser wurden ausdrücklich als solche gekennzeichnet.

\vspace{1 cm}

\begin{minipage}[t]{0.31\linewidth}
	\centering
	Luzern, 18. Dezember 1492
\end{minipage}
\hfill
\begin{minipage}[t]{0.45\linewidth}
	\includegraphics[width=0.45\textwidth]{04_Figures/Unterschrift2.png}
\end{minipage}


\vspace{-0.5 cm}
\begin{tabular}{p{7cm}p{.5cm}l}
	\dotfill \\
	Ort, Datum
\end{tabular}
\vspace{1,5 cm}
\begin{tabular}{p{7cm}p{.5cm}l}
	\dotfill \\
Unterschrift
\end{tabular}
\hfill

  % \section*{Abstract}

Bla Bla
\newpage


% ==================== Table of contents ====================
  \tableofcontents
  \newpage

  % % makeindex main.nlo -s nomencl.ist -o main.nls
\nomenclature{$t_k$}{Höhe des Schaumkernes}
\nomenclature{$E_k$}{E-Modul der Kernschicht}
\nomenclature{$G_k$}{Schub-Modul des Schaumkernes}
\nomenclature{$t_d$}{Höhe der Deckschicht}
\nomenclature{$E_d$}{E-Modul der Deckschicht}
\nomenclature{$h$}{Abstand der neutralen Fasern der Deckschichten}
\nomenclature{$w_b$}{Verformung durch Biegegelastung\nomunit{$mm$}}
\nomenclature{$w_s$}{Verformung durch Schubbelastung\nomunit{$mm$}}
\nomenclature{$w_{Ges}$}{Gesammtverformung\nomunit{$mm$}}

\nomenclature{$p_kB$}{Euler-Knickstreckenlast des schubsteifen Balkens\nomunit{$N/mm$}}
\nomenclature{$p_kS$}{Schubknicklast\nomunit{$N/mm$}}
\nomenclature{$p_k$}{Kritische Knicklast\nomunit{$N/mm$}}

\nomenclature{$n$}{Normalkraft pro Länge \nomunit{$N/mm$}}
\nomenclature{$m$}{Moment pro Länge \nomunit{$N$}}
\nomenclature{$q$}{Schubkraft Pro Länge \nomunit{$N/mm$}}

\nomenclature{$p$}{Streckenlast \nomunit{$N/mm$}}

  % \printunsrtglossary[type=symbols,style=symbunitlong]
  % \printnomenclature
  % \newpage

% ==================== Mainmatter ====================
  \pagenumbering{arabic}
  \setcounter{page}{1}
  \part{Dokumentation}
  \section{Einleitung}
Der Klimawandel äussert sich in der Schweiz überdurchschnittlich. So ist die mittlere Jahrestemperatur in der Schweiz seit Messbeginn im Jahr 1864 um 2 °C gestiegen, was rund doppelt so stark ist, wie das globale Mittel. In der Schweiz wird rund ein Drittel aller Treibhausgasemissionen durch den Verkehr (ohne internationalen Flug- und Schiffsverkehr) verursacht \cite{BAFU}. Um das \emph{Netto-Null-Ziel} der \emph{Langfristigen Klimastrategie der Schweiz} zu erfüllen, müssen daher unteranderem im Verkehrssektor Veränderungen vorgenommen und Entwicklungen getätigt werden.\\
Louis Palmer, ein Schweizer Umweltaktivist und \glqq \emph{Macher}\grqq{}, umrundete im Jahr 2007 als erster mit einem Elektrofahrzeug - dem Solarfahrzeug \emph{Solartaxi} - die Erde und gilt somit als ein Pionier im Bereich der Elektromobilität \cite{Palmer}.\\
Sein neustes Projekt ist der \emph{Solar Butterfly} - ein autarker Wohnwagen, mit welchem er \glqq eine Reise zu den Klimalösungen dieser Welt [...] im ersten solarbetriebenen <<Mobile Home>> der Welt\grqq{} antreten will. Ein Ziel von Palmer ist es, mit dem Solar Butterfly weltweite Aufmerksamkeit zu erregen und so nachhaltige Lösungen im Bereich des Klimaschutzes und Elektromobilität zu ermutigen und voranzutreiben.\\
Die erneute Weltumrundung soll dieses Mal \glqq mit etwas mehr Komfort\grqq{} geschehen. Seine Vision ist es, ein Wohnwagen mit zwei ausziehbaren Wohnelementen und rund 100 $m^2$ integrierter Photovoltaikfläche, zu realisieren. Der Wohnwagen soll sich selbst mit Solar-Energie versorgen und autonom operiert werden können. Im Rahmen dieser Bachelor-Thesis soll, in Zusammenarbeit mit drei weiteren Studenten der HSLU T\&A, seine Vision des Solar Butterflys in die Realität umgesetzt werden.\\
Das Projekt wurde neben dieser Arbeit in die weiteren Teilgebiete \emph{Auslegung Klappmechanismen}, \emph{Auslegung Antriebstechnik} und \emph{Auslegung Solar Butterfly (Globales CAD)} aufgeteilt.\\
Das Auslegen der Klappmechanismen, was von \emph{Buholzer} in angriff genommen wird, beinhaltet das Entwerfen und Dimensionieren aller beweglichen Teilen wie der klappbaren Solarpanelen und den Ausfahrmechanismus der seitlichen Raumelementen. Die Arbeit \emph{Auslegen der Antriebstechnik} von \emph{Bacher} befasst sich mit der Technik, mit welcher die beweglichen Bauteile in Bewegung gesetzt werden. Im Teilgebiet \emph{Auslegung Solar Butterfly (Globales CAD)} von \emph{Huber} werden die jeweiligen Teilgebiete zusammengeführt. Ebenfalls beinhaltet diese Aufgabenstellung das Erstellen eines globalen CAD-Modells, das Zusammentragen der allgemeinen Anforderungen sowie eine Risikobewertung des Projektes.

\subsection{Aufgabenstellung}
\label{Aufgabenstellung}
Der Fokus dieser Arbeit liegt in der Festlegung der Anforderungen und Auslegungskriterien, der Ausarbeitung eines detaillierten Lastenheftes, sowie in der Dimensionierung der Grundstruktur. Zur Bestimmung von Schnittgrössen, mit welchen Handrechnungen gemacht oder verifiziert werden können, soll dabei ein globales FEM-Modell zur Anwendung kommen. Ebenfalls sollen zulässige Festigkeitswerte abhängig von der gewählten Bauweise abgeschätzt werden (Design-Allowables).\\
Weiter beinhaltet die Aufgabenstellung eine enge Zusammenarbeit mit den drei weiteren Mitstudenten. Es soll sich aktiv an der Lösungsfindung und weiteren Ausarbeitung des Konzeptes beteiligt werden. Dabei sollen besonders die Aspekte und Positionen der strukturellen Integrität berücksichtigt und vertreten werden.

\subsection{Vorgehen und Methodik}
In diesem Kapitel wird beschrieben, wie beim Lösen der Aufgabenstellung vorgegangen wird. Die Struktur des vorliegenden Dokumentes entspricht dabei dem nun vorgestellten Vorgehen.

In einem ersten Schritt wird definiert, welchen Anforderungen der Solar Butterfly als Ganzes, von einem Standpunkt der strukturellen Integrität aus betrachtet, gerecht werden muss. Weiter werden die Auslegungskriterien bestimmt. Sie beschreiben im Detail, nach welchen Kriterien die einzelnen Komponenten des Solar Butterflys ausgelegt werden. So wird zum Beispiel beschrieben, welche Kriterien die Sandwichplatten erfüllen müssen, dass diese unter Belastung nicht beulen.\\
Anschliessend wird ein Lastenheft erstellt, welches eine Zusammenstellung von verschiedenen Lastfällen darstellt, welchen der Solar Butterfly ausgesetzt werden kann. Ferner werden sogenannte \emph{Modi} eingeführt, welche die Zustände und Positionen, in welchen sich der Solar Butterfly befinden kann, beschreiben. Für die Lastfälle des Lastenheftes - und Kombinationen davon - soll der Solar Butterfly in den verschiedenen Modi ausgelegt werden.\\
Als nächstes wird der Solar Butterfly grob als \glqq Kasten\grqq{} idealisiert. Es werden für die kritischen Lastfälle Handrechnungen durchgeführt, um so Kräfte und Schubflüsse bestimmen zu können. Dies wird zum einen gemacht, um die Grössenordnung der Lasten besser abschätzen zu können. Andererseits kann dadurch eine erste grobe Dimensionierung der wichtigsten Komponenten erfolgen und die zu diesem Zeitpunkt bereits getroffenen Annahmen bezüglich den Lasten beurteilt werden.\\
In einem letzten Schritt wird der Solar Butterfly in FEM-Simulationen verschiedenen kritischen Lastfällen ausgesetzt, um so Lastpfade und Schnittkräfte zu bestimmen, anhand welchen eine Verifizierung der Handrechnungen und eine exaktere Dimensionierung und Beurteilung der Komponenten und Verbindungen erfolgen kann. Weiter können mit den FEM-Berechnungen für die Funktionstauglichkeit kritische Verformungen festgestellt werden, welche in der weiteren Ausarbeitung des Konzeptes berücksichtigt werden sollen.\\
Im Kapitel \ref{Diskussion} werden zum Schluss die erlangten Erkenntnisse zusammengefasst und Empfehlungen für das weitere Vorgehen abgegeben.

\subsection{Der Solar Butterfly}
Diese Kapitel beabsichtig, einen Überblick des Solar Butterflys zu verschaffen und die Funktionen der wichtigsten Komponenten zu erklären.
In den folgenden drei Abbildungen ist der Solar Butterfly schematisch dargestellt. Im Anhang \ref{Bilder SB} sind realitätsgetreuere Abbildungen des CAD-Modelles und im elektronischen Anhang \ref{Bilder des Solar Butterflys} Detailansichten und weitere Bilder zu finden. Die Abbildungen in den Anhängen wurden von \emph{Huber} erstellt.

\begin{figure}[H]
  \includegraphics[width=\linewidth]{04_Figures/SB1.png}
  \caption{Der Solar Butterfly von Oben}
  \label{img:SB1}
\end{figure}

\begin{figure}[H]
  \includegraphics[width=\linewidth]{04_Figures/SB3.png}
  \caption{Seitenansicht des Solar Butterflys}
  \label{img:SB3}
\end{figure}

\begin{figure}[H]
  \includegraphics[width=\linewidth]{04_Figures/SB2.png}
  \caption{Schnittansicht des Solar Butterflys}
  \label{img:SB2}
\end{figure}

Grundbaustein des Solar Butterflys ist das Chassis, an welchem die Träger A und B und der Boden befestigt sind. An den Trägern A und B sind wiederum die Scharniere, mit welchen die seitlichen Raumelemente ausgefahren werden können, sowie die Wände und das Dach befestigt.\\
In Bewegung gebracht werden die Bauteile des Solar Butterflys durch Pneumatik. Die seitlichen Raumelemente werden durch je zwei Pneumatikzylinder im Chassis aus- und eingefahren. Die beweglichen Solarpanelen werden ebenfalls mittels Pneumatik, in Kombination mit Gasdruckfedern, bewegt. Die Solarpanelen der Reihe D können über Teleskopscharniere unterhalb der Reihe C hervorgeschoben werden.\\
Der Solar Butterfly soll verschifft werden können, muss daher in einen Container passen und darf die Masse von 2.64 x 10.2 x 2.3 (Höhe x Länge x Breite) nicht überschreiten - sie sollen jedoch ausgenützt werden. Die Gewichtslimite des Solar Butterflys beträgt für Europa 2200 und für den Rest der Welt 3000 kg. Das Einhalten dieser Limite stellt eine der grössten Herausforderungen des Projektes dar. Die Thematik des Gewichtes prägt entsprechend die Arbeiten, wird immer wieder aufgegriffen und hat die getroffenen Entscheidungen massgebend beeinflusst.\\
Während der Fahrt sollen die beweglichen Teile des Solar Butterflys über formschlüssige Verbindungen mit dem Rest der Struktur verbunden werden. So kann gewährleistet werden, dass die beweglichen Bauteilen auch bei einer Fahrt mit hohen Geschwindigkeiten nicht ausfahren oder ausklappen.\\
Im Innern des Solar Butterflys soll sich Moblilar befinden (Sofa, Tische etc.) welches während der Fahrt befestigt und verstaut werden muss.

\newpage

  \section{Anforderungen und Auslegungskriterien}
In diesem Kapitel wird beschrieben, welchen Anforderungen der Solar Butterfly und dessen Komponenten gerecht werden müssen. In einem ersten Schritt werden auf die allgemeinen Anforderungen des Solar Butterflys und anschliessen auf die daraus resultierenden Auslegungskriterien der einzelnen Komponenten eingegangen. Es wird beschrieben, was die Anforderungen konkret für die einzelnen Komponenten bedeuten und wie gewährleistet wird, dass diese erfüllt werden.\\
Im rahmen dieser Arbeit wird lediglich auf diejenigen Anforderungen eingegangen, welche für die strukturelle Auslegung und Festigketisberechnungen relevant sind. Die komplette Liste der Anforderungen an den Solar Butterfly ist in der Arbeit von [HUBER] zu finden.

\subsection{Anforderungen an den Solar Butterfly}
\begin{itemize}
  \item Der Solar Butterfly muss den Angreifenden Kräften und herrschenden Lastfällen standhalten. (Vgl. Lastenheft [KAPITEL]) Konkret bedeutet dies, dass die Struktur sich bei den verschiedenen Lastfällen, und Kombinationen davon, nicht plastisch verformen darf und somit eine genügend grosse Sicherheit gegen Fliessen aufweisen muss.\\
  \item Weiter darf der Solar Butterfly sich nicht so stark verformen, dass seine Funktionstauglichkeit eingeschränkt wird. Die exakten Anforderungen an die Steifigkeit werden bei der Abhandlnung der einzelnen Komponenten genauer betrachtet und beschrieben.\\
  \item Die Struktur des Solar Butterflys soll so ausgelegt werden, dass dieser ca. 300'000 km Fahrt auf zum teil recht holperiger Strasse auf sich nehmen kann. Dies beinhaltet die Auslegung der Komponenten auf Dauerfestigkeit.
\end{itemize}


\subsection{Auslegungskriterien}
Nachdem die allgemeinen Kriterien für den Solar Butterfly abgehandeln wurden, wird in diesem Unterkapitel behandelt, was die Anforderungen konkret für die einzelnen Komponenten und Strukturelementen bedeutet. Es wird beschrieben mit welchen Methoden die Auslegung angegangen wird und welche Vereinfachungen getroffen werden.\\

\paragraph{Design-Allowables}
Design-Allowables: Materialkennwerte mit welchen die Auslegung gemacht wird.\\
In diesem Materialkennwerte sind die Sicherheitsfaktoren drinnen und die Absicherung gegen ermüdung.\\

Die Dauerfestigkeit wird vorallem durch gutes Design erreicht. Lokal müssen beim Nachweis evenutell anpassungen gemacht werden und exaktere Werte zur Hilfe genommen werden.

  \subsubsection{Aluminiumstrukturen}
  Zu den Auslegungskriterien der Aluminiumstrukturen gehört das Festigkeitsproblem der plastschen Verformung (Fliessen) und das Stabilitätsproblem der Knickung. Die Aluminiumstrukturen werden so ausgelegt, dass diese eine Sicherheit gegen Fliessen von 1.5 und gegen Knicken eine von 2 aufweisen.

  \paragraph{Sicherheit gegen Fliessen}
  Um die Sicherheit eines Strukturelementes gegen Fliessen zu gewährleisten, wird überprüft, ob die \emph{Von Mises}-Vergleichsspannung kleiner als die zulässige Spannung ist, wobei sich die zulässige Spannung aus der Dehngrenze des gewählten Materials und dem Sicherheitsfaktor zusammensetzt. Die \emph{Von Mises}-Vergleichsspannung kann gemäss der Formel \ref{Von Mises} berechet werden \cite{Baertsch}.
  \begin{equation}
    \label{Von Mises}
    \sigma_{zul} \geq \sigma_v = \sqrt{\sigma_x^{2}-\sigma_x \cdot \sigma_y + \sigma_y^2 + 3\tau^2}
  \end{equation}
  Wobei die Annahmen getroffen werden, dass es sich um einen ebenen Spannungszustand handelt und die angeifenden Lasten dem selben Lastfall angehören.

  \paragraph{Knicken}
  Wird durch design verhindert (?)

  \subsubsection{Sandwichstrukturen}
  Versagenskriterien der Sandwichstrukturen können in die beiden Kategorien \emph{Festigkeitsprobleme} und \emph{Stabilitätsprobleme} eingeteilt werden \cite{ETH}. Zu den Festigkeitsproblemen gehören;
  \begin{itemize}
    \item Fliessen der Deckschicht,
    \item Schubbruch der Kernschicht,
    \item Delamination und
    \item Ermüdung.
  \end{itemize}

  Zu den Stabilitätsproblemen gehören unteranderem;
  \begin{itemize}
    \item Knickung,
    \item Schubbeulung der Kernschicht (Shear Crimping) und
    \item Kurzwelliges Beulen der Deckschicht (Wrinkling).
  \end{itemize}

  Die auszulegenden Sandwichstrukturen werden gegenüber diesen Festigkeits und Stabilitätsproblemen abesichert. Um den Rechenaufwand und die Komplexität zu verringern werden Annahmen und Vereinfachungen getroffen. Für die Auslegung von Sandwichstrukturen können folgende Annahmen getroffen werden \cite{ETH}\cite{klein};
  \begin{itemize}
    \item linear elastische und isentrope Materialverhalten,
    \item Eigenbiegesteifigkeiten der Deckschichten sind vernachlässigbar,
    \item Dehnsteifigkeit der Kernschicht ist vernachlässigbar und
    \item die Kernschicht lässt sich nicht zusammendrücken.
  \end{itemize}
  Aus den getroffenen Annahmen reulstiert ein vereinfachter Spannungszustand welcher besagt, dass die Deckschichten jeweils die Normalkräfte und die Kernschichten die Schubkräfte aufnehmen. (Sandwichmembrantheorie)

    \paragraph{Festigkeitsprobleme}
    Aus den getroffenen Annahmen und Vereinfachungen lassen sich die Formeln \ref{Spannung in Deckschicht} und \ref{Schubspannungen im Kern} herleiten. Mit der Formel \ref{Spannung in Deckschicht} lassen sich die Spannungen in den Deckschichten berechnen. Die Dicke der Deckschicht wird so gewählt, dass die zulässige Spannung höher liegt als jene, welche in der Deckschicht herrscht.

    \begin{equation}
      \label{Spannung in Deckschicht}
      \sigma_{zul} \geq \sigma_d = \frac{1}{t_d}\cdot \left ( \frac{n}{2} \pm \frac{m}{h}\right )
    \end{equation}

    Mit der Formel \ref{Schubspannungen im Kern} lassen sich die Schubspannungen in der Kernschicht berechnen und somit Aussagen über ihre Resistenz gegenüber dem Schubbruch machen. Die Dicke der Kernschicht wird so ausgelegt, dass die in der Kernschicht herrscheden Spannungen tiefer liegen als die zulässigen.
    \begin{equation}
      \label{Schubspannungen im Kern}
      \tau_{k,zul} \geq \tau_k = \frac{q}{t_k}
    \end{equation}

    Die Delamination der Deckschichten wird abgesichert, indem die Auswahl des Klebers, oder im Falle einer Laminierung die Matrix, so getroffen wird, dass dieser eine höhere Schubfestigkeit aufweist als das Material der jeweiligen Kernschicht.

    \paragraph{Stabilitätsprobleme}
    Die Stabilitätsprobleme der Sandwichstrukturen lassen sich in globale und lokale Instabilitäten einteilen. Zur globalen Instabilität gehört das Knicken, welches sich aus der Eueler-Knickung des schubsteifen Balkens und dem Schubknicken zusammensetzt. Die kritische Belastung, bei welcher es zur Euler-Knickung kommt, lässt sich gemäss Klein \cite{klein} mit der Formel \ref{Euler-Knicklast} berechnen.

    \begin{equation}
      \label{Euler-Knicklast}
      F_{kB}=\frac{\pi^2 \cdot E_d \cdot I_y}{l_k^{2}}
    \end{equation}

    Wobei sich die Biegesteifigkeit $I_y$ vereinfacht gemäss der Formel \ref{ID} berechnen lässt. Hier wurde die Annahme getroffen, dass die Eigenbiegesteifigkeiten der Deckschichten vernachlässigbar sind. Diese Annahme kann gemäss Klein \cite{klein} ab einem Verhältnis von $t_d$ zu $t_k$ von 0.25, getroffen werden.
    \begin{equation}
      \label{ID}
      I_y = 2 \cdot b \cdot t_d \cdot \left( \frac{t_k}{2} + t_d \right )^{2}
      % B_y = 2\cdot b \cdot t_d\cdot \left ( \frac{t_k}{2}+t_d \right )^2
    \end{equation}

    Die kritische Schubknicklast lässt sich gemäss Klein \cite{klein} mit der Formel \ref{Schubknicklast} berechnen.
    \begin{equation}
      \label{Schubknicklast}
      F_{kS} = b \cdot t_k \cdot G_k
    \end{equation}

    Die totale kritische Knicklast \(F_k\) ergibt sich dann aus der Formel \ref{Knicklast}:
    \begin{equation}
      \label{Knicklast}
      F_{k, vorh.} \leq F_k=\frac{1}{\frac{1}{F_{kB}}+\frac{1}{F_{kS}}}
    \end{equation}

    Zu den lokalen Instabilitäten zählen das Schubbeulen und das Knittern der Deckschicht. Die kritischen Spannunge, bei welcher Schubbeulung auftritt, lässt sich aus den Formel \ref{Schubbeulen} berechnen. \cite{ETH}
    \begin{equation}
      \label{Schubbeulen}
      \sigma_k = G_k \cdot \frac{h}{2 \cdot t_d}
    \end{equation}

    Die kritischen Spannunge, bei welcher das Knittern der Deckschicht auftritt, lässt sich mit der Formel \ref{Knittern} berechnen. \cite{ETH}
    \begin{equation}
      \label{Knittern}
      \sigma_k = k_s\sqrt[3]{E_d \cdot E_k \cdot G_k}
    \end{equation}
    Wobei für Auslegungen \(k_s = 0.5\) gilt.

    \paragraph{Design-Allowables und Materialkennwerte}
    \begin{table}
      \centering
      \caption{Design-Allowables Sandwichplatten}% Add 'table' caption
      \begin{tabular}{llcccc}
        \thickhline
        Bezeichnung & & Einheit & & Sicherh. Fakt. & Zul. Festigkeit\\
        \hline
        \multicolumn{2}{l}{\textbf{Deckschicht}}\\
        \thickhline
        \multirow{4}{*}{Aluminium}  & Dichte            & $\frac{kg}{m^3}$  & 2710      & &\\
                                    & E-Modul           & $MPa$             & 70'000    & &\\
                                    & Zugfestigkeit     & $MPa$             & 150       & 1.5 & $\sigma_{zul} = 100$\\
                                    & Dauerfestigkeit   & $MPa$             & 100       & 1.5 & $\sigma_{D,zul} = 75$\\

        \hline
        \multirow{4}{*}{GFK}        & Dichte            & $\frac{kg}{m^3}$  & 2000      & &\\
                                    & E-Modul           & $MPa$             & 16'000    & &\\
                                    & Zugfestigkeit     & $MPa$             & 250       & 1.5 & $\sigma_{zul} = 66$\\
                                    & Dauerfestigkeit   & $MPa$             & 50       & 1.5 & $\sigma_{D,zul} = 33$\\
        \hline

        \multicolumn{2}{l}{\textbf{Kern}}\\
        \thickhline
        \multirow{4}{*}{Airex T92.60} & Dichte            & $\frac{kg}{m^3}$  & 65      & &\\
                                      & E-Modul (Druck)   & $MPa$             & 55      & &\\
                                      & Schubmodul        & $MPa$             & 15      & &\\
                                      & Schubfestigkeit   & $MPa$             & 0.55    & 1.5 & $\tau_{zul} = 0.5$\\

        \hline
        \multirow{4}{*}{Airex T92.80} & Dichte            & $\frac{kg}{m^3}$  & 85      & &\\
                                      & E-Modul (Druck)   & $MPa$             & 75      & &\\
                                      & Schubmodul        & $MPa$             & 22      & &\\
                                      & Schubfestigkeit   & $MPa$             & 0.72    & 1.5 & $\tau_{zul} = 0.6$\\

      \thickhline
      \end{tabular}
    \end{table}


  \subsubsection{Nieten}
    % \paragraph{Tragfähigkeitsnachweis}
    Laut Klein \cite{klein} gehört zum Tragfähigkeitsnachweis für gewöhnlich ein Abscher- und Lochleibungsnachweis. Insofern sei für Nietverbindungen ein Nachweis auf Scherbruch (Formel \ref{Scherbruch}) und Lochleibung (Formel \ref{Lochleibung}) zu erbringen:
    \begin{multicols}{2}
      \begin{equation}
        \label{Scherbruch}
        F \leq F_{SB} = \frac{d_N^2 \cdot \pi}{4}\cdot \tau_B
      \end{equation}\break
      \begin{equation}
        \label{Lochleibung}
        F \leq F_{LF} = d_N \cdot t \cdot \sigma_{FL}
      \end{equation}
    \end{multicols}
    Wobei $d_N$ der Nietlochdurchmesser, $\tau_B$ die Scherfestigkeit, $t$ die Blechdicke und $\sigma_{FL}$ die Lochleibungs-Dehngrenze ist. Für dynamische Wechselfestigkeitswerte sei die Scherfestigkeit $\tau_B$ noch um den Faktor 2 bis 2.2 zu verringern.

    \paragraph{Überlagerte Scher- und Zugbeanspruchung}
    In der Praxis werden Nietverbindungen aus einer Kombination von Scher- und Zugbeanspruchung beansprucht. Der Nachweis der Tragfähigkeit der überlagerten Belastung wird durch die Ausweisung des Reservefaktors $R_f$ bewerkstelligt. Dazu werden gemäss den Formeln \ref{Rs} und \ref{Rz} der Schubreservefaktor $R_s$ und der Zugreservefaktor $R_z$ berechnet

    \begin{multicols}{2}
      \begin{equation}
        \label{Rs}
        F \leq F_{SB} = \frac{d_N^2 \cdot \pi}{4}\cdot \tau_B
      \end{equation}\break
      \begin{equation}
        \label{Rz}
        R_z = \frac{F_z}{k \cdot F_{ZB}}
      \end{equation}
    \end{multicols}

    \subsubsection{Klebeverbindungen}
    \begin{equation}
      \label{Kleben}
      \tau_K = \frac{F}{b \cdot l_{\ddot{u}}} \leq \frac{\tau_{KB}}{S}
    \end{equation}

    \begin{equation}
      \label{Zulässige Schubspannungen}
      \begin{split}
      wechselnd: & \: \tau_{KW} \approx \left (0.2 ... 0.4 \right ) \cdot \tau_{KB}\\
      schwellend: & \: \tau_{KSch} \approx 0.8 \cdot \tau_{KB}
      \end{split}
    \end{equation}

    \paragraph{Design-Allowables und Materialkennwerte}\mbox{}\\

    \begin{table}[h]
      \centering
      \caption{Design-Allowables Kleber}% Add 'table' caption
      \begin{tabular}{llcccc}
        \thickhline
        Bezeichnung & & Einheit & & Sicherh. Fakt. & Zul. Festigkeit\\
        \thickhline
        \multirow{3}{*}{Delo-Duopox\textsuperscript{\textregistered} AD840} & E-Modul               & $MPa$             & 1700    & &\\
                                           & Zugscherfestigkeit    & $MPa$             & 5       & 3 & $\sigma_{zul} = 1.6$\\
                                           & Druckscherfestigkeit  & $MPa$             & 26      & 3 & $\sigma_{zul} = 8.6$\\
                                           \hline

       \multirow{1}{*}{Sikaflex\textsuperscript{\textregistered}-552 AT} & Zugscherfestigkeit    & $MPa$ & 2 & 3 & $\sigma_{zul} = 0.6$\\

      \thickhline
      \end{tabular}
    \end{table}
\newpage

  \section{Lasten}
Begründung und Erklärung der Lasten
Aufbau des Lastenheftes A B1, B2 und C, mit verschiedenen Konfigurationen

\subsection{Belastungen beim Fahren}
Zustand A:


\paragraph{Beschleunigungen}
Beschleunigungen werden Absolut angegeben. 0g heisst schwerelosigkeit. wird in G angegeben wobei damit $9.81 \left[\frac{m}{s^2}\right]$ gemeint ist.\\
Die folgenden Beschleunigungen werden immer in Kombination mit der Erdbeschleunigung von 1g kombiniert.


\begin{description}
  % \item \textbf{1.1 Erdbeschleunigung}\\
  % Standardmässig ist der Solar Butterfly der Erdbeschleunigung von 1g ausgesetzt. \\

  \item \textbf{1.2 Vertikale Beschleunigung}\\
  Zusätzlich zur vertikalen Beschleunigung durch die Erdanziehung, entstehen durch das Überfahren von Schlaglöcher und Bremsschwellen vertikale Beschleunigungen.\\
  In einem ersten Ansatz wurde der Solar Butterflys als ein \emph{Ein-Massen-Schwinger}-System modelliert und die Beschleunigung beim Überfahren einer Sinusförmigen Bremsschwelle numerisch berechnet.\\

  Die Position des Rades während dem Überfahren der Bremsschwelle ist gegeben durch folgenden Zusammenhang:
  \begin{equation}
    x_r^n = h \cdot \sin(\pi \cdot \frac{n\Delta t \cdot v}{l})
  \end{equation}
  $l$ steht dabei für die Länge, und h für die Höhe der Bremsschwelle.\\

  Um die Beschleunigung des Solar Butterflys zu berechnen, wird in einem ersten Schritt dessen Position zum Zeitpunk $n$ $x_{SB}^n$ aus der vorangehenden Situation berechnet.
  \begin{equation}
    x_{SB}^n = x_{SB}^{(n-1)} + v^{(n-1)} \cdot \Delta t
  \end{equation}

  Als nächstes wird der Federweg $s^n$, sowie die Änderungsrate des Federwegs $v_s^n$ zum Zeitpunkt $n$ berechnet.
  \begin{equation}
    s^n = x_r^n - x_{SB}^n
  \end{equation}
  \begin{equation}
    v_s^n = \frac{s^n - s^{(n-1)}}{\Delta t}
  \end{equation}

  Die Beschleunigung des Solar Butterfly ergibt sich dann zu:\\
  \begin{equation}
    a_{SB}^n = \frac{k \cdot s^n + d \cdot v_s^n}{m}
  \end{equation}

  womit die daraus resultierende neue Geschwindigkeit des Chassis berechnet werden kann.
  \begin{equation}
    v^n = v^{(n-1)} + a^n \cdot \Delta t
  \end{equation}

  Die Berechnungen wurden bis zum ersten Null-Durchgang gemacht. Die Anzahlberechnungspunkte und $\Delta t$ werden entsprechend gewählt.

  Bei einer Geschwindigkeit von 40 km/h\\
  k = ~ 177'000 N/m\\
  d = 2000 Ns/m\\
  l = 0.9 m\\
  h = 0.1 m\\
  Resultat: 1.6g\\
  Konservatives Modell: Federung durch Reifen fehlt, weiter ist der Massenschwerpunkt aus der Feder-achse Verschoben, was eine weitere abminderung der Beschl. zur folge hat.

  \cite{Beschl.1}: PKW; 0.36x0.05 Speed bump; 0.71 g (Fahrzeug Mitte) und 1.5 g (Über der Achse) bei 40 km/h\\
  \cite{Beschl.2}: PKW; 0.90x0.10 Speed bump; 0.73 g (Fahrzeug Mitte) bei 50 km/h\\
  \cite{Beschl.3}: PKW; 0.50x0.05 Speed bump; 1 g (Person hinten im Fahrzeug) und 1.3 g (Über der Achse) bei 30 km/h\\
  \cite{Beschl.4}: LKW; +-1g Beschleunigung für Transportware in einem Sattelschlepper.

  Schluss: 1.25g vertikale Beschleunigung


  \item \textbf{1.3 Longitudinale Beschleunigung}\\
  - Beschleunigung durch Notbremse\\
  \cite{Verz.1} Pkw max. über 0.9 g\\
  \cite{Verz.2} Pkw bis 0.8 g, Lkw bis 0.7 g\\

  - Beschleunigung druch Unfall\\
  Wird ausgeschlossen?\\

  Schluss: +-1 g

  \item \textbf{1.4 Laterale Beschleunigung}\\
  Beschleunigung durch Kurven\\
  \cite{Kurv.1} ca. 0.6 g auf Landstrasse\\
  \cite{Kurv.2} mehrheit in bergigem Gebiet über 0.5 und max. über 0.8 g\\

  Schluss: +-1 g\\
\end{description}

\paragraph{Belastung durch Wind}

\cite{Wind.1} Laterale Windgeschwindigkeiten von 108 km/h seien kritisch für Fahrzeuge auf trokener Strasse.\\
\cite{Wind.2} Bei Windgeschwindigkeiten von mehr als 80 km/h wird empfohlen nicht mehr zu Fahren. Windgeschwindigkeiten von 95 km/h sei genug um Wohnmobile umzustossen\\
\cite{Wind.3} Bei Windgeschwindigkeiten von mehr als 155 km/h können "high profile Trucks, Trailers and Busse" überkippen. Minimale OVERTURNING WIND SPEEDS von 108 km/h für 9 m langes Motor home und 160 km/h für ein 5m langes Wohnmobil.

\begin{description}
  \item \textbf{2.1 Wind}\\
  Schluss: bei 80km/h soll nicht mehr gefahren werden: Winddruck bei 100km/h\\
  Definitive Windgeschw. zum kippen kann schlussendlich mit dem FEM ermittelt werden. Dies soll nur eine erste Annahme sein.
  \begin{equation}
    \label{Winddruck}
    W_D = c_p \: \frac{\rho}{2}\: v^2
  \end{equation}
  $\rho = 1.2 \frac{kg}{m^3}$\\
  $c_p \: Rechteck = 1.05$\\

  $W_D \:@\: 100 km/h= 486 \left[MPa \right]$
\end{description}

\paragraph{Belastung durch Neigung}
Absprache mit Palmer: Furkapass max. 6.3° -> auslegung auf 10° (17.5\%)
Neigung in beide Richtungen
\begin{description}
  \item \textbf{3.1 Geneigte Strasse Länges} +-10° Neigung in Fahrtrichtung.
  \item \textbf{3.2 Geneigte Strasse Quer} +-10° Neigung normal zur Fahrtrichtung.

\end{description}

\paragraph{Mobiliar}
50kg Mobiliar im Fahrzeug verteilt
\begin{description}
  \item \textbf{4.1 Mobiliar vorne}
  \item \textbf{4.2 Mobiliar hinten}
\end{description}

  \input{02_Mainmatter/04_Handrechnungen}
  \section{Dimensionierung}
Dieses Kapitel stellt eine Zusammenstellung der ausgelegten Komponenten dar. Es wird beschrieben, wie die Komponenten ausleget, und welche Überlegungen bei der Auslegung gemacht wurden. Die Berechnungen wurden mit einer Excel-Tabelle getätigt, welche im elektronischen Anhang ANHANG einsehbar ist.

\subsection{Boden}
Der Fussboden des Solar Butteflys soll als Sandwichstruktur realisiert werden, wobei als Deckschicht Aluminiumblech und als Kern geschäumtes Ocean-PET verwendet werden soll. Der Fussboden muss die Personenlasten aufnehmen und auf das Chassis übertragen. Weiter sieht das zum Zeitpunkt der Durchführung dieser Arbeit verfolgte Konzept vor, dass die Seitenmodule während der Fahrt über den Boden mit dem Rest der Struktur verbunden und befestigt werden. Im Rande des Bodens sollen abschnittweise Aluminiumprofile eingebettet werden, an welchen die Seitenmodule befestigt werden können.\\
Zu Begin der Ausarbeitung des Konzeptes wurden umrahmende Aluminiumprofile aus dem Fahrzeugbau in betracht gezogen, welche auf Platten mit einer Dicke von 25mm passen. Eine erste Annahme der Dicke des Bodens wurde so getroffen, dass diese in die besagten Profilen passen. Über eine Absprache mit einem Experten aus dem Wohnmobilbau wurde in erfahrung gebracht, dass in Wohnmobilen häuffig Fussböden mit einer Dicke von 30 mm, mit einer Dicke der Deckschichten von 1 mm, verbaut werden. [QUELLE] Weiter wurde mitgeteilt, dass die erste Abschätzung der Dicke von 25 mm eine plausible sei.\\

Um die getroffene Abschätzung zu überprüfen und die Dicke der Deckschichten zu bestimmen, wurden für zwei Belastungsfälle berechnungen angestellt. Die Fallunterscheidung sowie die Idealisierungen der Lagerung und Krafteinleitung der beiden Fälle ist in der Abbildung \ref{Boden Idealisierung} dargestellt.

\begin{figure}[!ht]
  \centering
    \begin{subfigure}{.5\textwidth}
      \centering
      \includegraphics[width=.98\linewidth]{04_figures/Boden Fall1.png}
      \caption{Belastungsfall 1}
      \label{Belastungsfall 1}
    \end{subfigure}%
    \begin{subfigure}{.5\textwidth}
      \centering
      \includegraphics[width=.98\linewidth]{04_figures/Boden Fall2.png}
      \caption{Belastungsfall 2}
      \label{Belastungsfall 2}
    \end{subfigure}%
  \caption{Darstellung der beiden Belastungsfällen und deren Idealisierungen}
\label{Boden Idealisierung}
\end{figure}

Als Belastung wird eine Masse von 200 kg gewählt, welche auf einen 1000 mm langen Bodenabschnitt eingeleitet wird. Der Boden hat eine Breite von 2050 mm und der Abstand zwischen den Lagern beträgt 1410 mm. In der Abbildung \ref{Boden QM} sind die Querkraft- und Biegemomentenverläufe der beiden Fälle dargestellt.

\begin{figure}[!ht]
  \centering
    \begin{subfigure}{.5\textwidth}
      \centering
      \includegraphics[width=.98\linewidth]{04_figures/Boden QM1.png}
      \caption{Belastungsfall 1}
      \label{Belastungsfall 1}
    \end{subfigure}%
    \begin{subfigure}{.5\textwidth}
      \centering
      \includegraphics[width=.98\linewidth]{04_figures/Boden QM2.png}
      \caption{Belastungsfall 2}
      \label{Belastungsfall 2}
    \end{subfigure}%
  \caption{Querkraft- und Biegemomentenverläufe der Bodenplatten}
\label{Boden QB}
\end{figure}

Deckschicht von 0.36 mm spannung von 80 MPa. 

Schubbeulen und Knittern!


Missbrauchslastfälle: Spitzer Schuh, Druckbelastung, Dellen,
Empfehlung: 1mm Gwicht: Viel Potential, Risiko in kauf nehmen, Falls schaden: einfach zu reparieren.




\subsection{Dach - Hauptmodul}
Beschreibung:
  Aluminiumprofile: Tragen der Panelen
  Unterbringung der Verschlüsse

  \paragraph{Berechnung}\mbox{}//
    Limitierender Faktor: Deformation
    Berücksichtigung des Eigengewichtes,
    Mittragende Breite des Daches


\subsection{Panelen}
  Erste Berechnung wie Dach, dann jedoch schwierig wegen komplexer aufhängung und 2D-Problem.
  Darum FEM:
    Eigengewicht, Beschleunigung, Extra Last mit Druck.

  Einfache Lagerung: Resultate stimmen überein mit Excel.
  Neue Aufhängungen:
    Verformungen und Bilder.
    Schluss in der Arbeit von BACHER.

  \section{Globales FEM}
Wie in der Aufgabenstellung beschrieben, soll zur Überprüfung der Handrechnungen und zur Bestimmung von Schnittgrössen ein gloables FEM-Modell zur Anwendung kommen. In diesem Kapitel wird nun beschrieben, wie dieses FEM-Modell aufgesetzt und welche vereinfachende Annahmen getroffen werden. Weiter werden die Ergebnisse der Simulationen aufgeführt und mit den Handrechnungen verglichen und beurteilt.


Analog zu den Handrechnungen werden vier verschiedene FEM-Berechnungen durchgeführt welche jeweils ein Lastfall der Beschleunigung des Modus \emph{A} genauer untersuchen.\\

Mit dem globalen FEM-Modell sollen folgende Punkte bestimmt werden:
\begin{itemize}
  \item Lagerreaktionen
  \item Maximale Axialkräfte, Querkräfte und Biegemomente in Chassis, Dach und den Trägern A und B
  \item Kontaktreaktion: Verbindung Chassis zu Träger A und B
  \item Kontaktreaktion: Verbindung Chassis zu Boden
  \item Deformation
\end{itemize}

\subsection{Idealisierung und Modell}
Der Solar Butterflys wird, wie in den Handrechnungen, als ``Kasten'' betrachtet und mit Balken und Schalen idealisiert. Das Chassis, die Deichsel, die Träger A und B sowie die Dachträger werden als Balkenelementen (Beam) mit den entsprechenden Querschnitten modeliert. Die Wände, Dächer und der Boden werden als Schalenkörper (Shell) modelliert, wobei den Schalenkörper jeweils ein Lagenaufbau (Layered Section) zugewiesen wird, welcher ihre Sandwichbauweise nachstellt. In der Abbildung \ref{FEM Mesh1} ist das komplette Modell des Solar Butterflys dargestellt. In der Abbildung \ref{FEM Mesh3} wurden die Schalenkörper ausgeblendet, sodass nur die Balken sichtbar sind.

\begin{figure}[H]
  \centering
  \centering
  \includegraphics[width=.8\linewidth]{04_figures/FEM Mesh1.png}
  \caption{Darstellung der Balken und Schalenkörper im FEM-Modell}
  \label{FEM Mesh1}
\end{figure}
\begin{figure}[H]
  \centering
  \includegraphics[width=.8\linewidth]{04_figures/FEM Mesh3.png}
  \caption{Darstellung der als Balken idealisierten Körper}
  \label{FEM Mesh3}
\end{figure}

Um die Masse des Solar Butterflys modellieren zu können, werden, zusätzlich zu den Massen der modellierten Bauteilen, Punktmassen (Point Mass) eingeführt. Es werden für die drei Raumelemente Küche, Mittelkörper und Bad je eine Punktmasse definiert, deren Masse und Trägheitsmomente mit der Hilfe der Massenverteilung aus dem Kapitel \ref{Massenverteilung} bestimmt werden. In der Abbildung \ref{img:FEM Punktmasse} sind die Verbindungen der Punktmassen mit dem Rest des Modelles dargestellt. Sie werden über das Chassis, die Träger A und B, sowie über die Verbindungsstellen wischen den Wänden und dem Boden getragen.\\
\begin{figure}[H]
  \centering
  \includegraphics[width=0.8\linewidth]{04_Figures/FEM Punktmasse.png}
  \caption{Verbindungen der Punktmassen zum Rest des Modelles}
  \label{img:FEM Punktmasse}
\end{figure}

Die Deichsel, Längsträger und Querträger des Chassis werden durch das Zusammenführen der deckungsgleichen Koten miteinander verbunden (Node Merge). Auf die selbe Art und Weise werden die Träger A und B, die Träger des Daches sowie der Boden, die Wände und das Dach des Aufbaus miteinander verbunden. Der verbundene Aufbau wiederum wird auf zwei Arten mit dem Chassis verbunden. Einerseits werden die Träger A und B über einen \emph{Fix-Joint} (Body-Body Verbindung, alle Freiheitsgrade eingeschränkt) an ihrem untersten Knoten mit dem Chassis verbunden. Weiter wird der Boden über \emph{General-Joint}-Verbindungen%Fussnote...
\footnote{Die \emph{General-Joint}-Verbindungen wurden mit der Hilfe von \emph{Named Selections} und der \emph{Object Generator} Funktion erstellt. Die Kraftreaktionen wurden durch die Parametrisierung der Ergebnisse ausgelesen.}
(Body-Body Verbindung, die rotatorischen Freiheitsgrade sind frei, die translatorischen eingeschränkt.) mit den Längsträgern des Chassis verbunden. Insgesammt ist der Boden an jedem Längsträger über 30 Knotenverbindungen it dem Chassis verbunden. Sie räpresentieren die Klebestellen zwischen Boden und Chassis.\\
In allen folgenden beschriebenen FEM-Simulationen ist der Solar Butterfly analog zu den Handrechnungen im Kapitel \ref{sub:Longitudinale Beschleunigung} (Lastfall \emph{1.1 Vertikale Beschleunigung}) gelagert. Am Spitz der Deichsel sind die rotatorischen Freiheitsgrade frei, die translatorischen jedoch eingeschränkt. An der Achse wird lediglich die Verschiebung in x-Richtung (Fahrtrichtung) zugelassen.

\subsection{Ergebnisse}
Im Anhang \ref{FEM Ergebnisse} sind die Ergebnisse der FEM-Berechnungen Tabelarisch festgehalten. Sofern für die ausgelesenen Grössen Handrechnungen durchgeführt wurden, sind deren Ergebnisse ebenfalls in den besagten Tabellen zu finden, sodass diese direkt mit den Ergebnissen der FEM-Berechnungen verglichen werden können. Die Schnittkräfte und Kontaktreaktionen der Tabellen beziehen sich jeweils auf einen einzelnen Balken oder Verbindung. Die Kontaktreaktion zwischen Chassis und Boden bezieht sich auf eine einzelne Knotenverbindung. Die in den Tabellen aufgeführten Werte stellen jeweils den Maximalwert dar.\\
Im Anhang \ref{FEM Deformation} sind Bilder, welche die Deformation des Solar Butterflys dokumentieren, zu finden. Die FEM-Datei ist im elektronischen Anhang ANHANG angefügt. Die Auswertung der Ergebnisse wurde mit einer Exceltabelle durchgeführt, welche im elektronsichen Anhang AHNAHNG zu finden ist.


\subsubsection{Vergleich mit Handrechnungen}
Im Lastfall \emph{1.1 Vertikale Beschleunigung} sind die berechneten Axialkräfte (15.8 kN) im Dach rund acht mal so hoch, wie jene des FEM-Modelles (1.9 kN). Dies ist vermutlich darauf zurück zu führen, dass das mittragende Dach, welches ebenfalls Axialkräfte aufnimmt, in den Handrechnungen nicht mit berücksichtigt wurde.\\
Im Lastfall \emph{1.4 Laterale Beschleunigung} sind die mit der FEM-Berechnung erhaltenen Axialkräfte im Chassis und den Längsträger des Daches gut drei mal höher als jene der Handrechnungen. Dies, da sich der Solar Butterfly unter lateraler Beschleunigung, nicht wie angenommen verbiegt, sondern verdreht. Die Art der Deformation ist ähnlich wie jene im Lastfall \emph{1.5 Rotatorische Beschleunigung} (vgl. Abbildungen \ref{FEM 1.4} und \ref{FEM 1.5} im Anhang \ref{FEM Deformation}). Da diese grundlegende Annahme der Auswirkungen der Belastung (und der Deformation) falsch getroffen wurde, sind die Ergebnisse auch dem entsprechend unterschiedlich. Die erhaltenen Kräfte sind in ihrer Art vergleichbar mit jenen des Lastfalles \emph{1.5 Rotatorische Beschleunigung}, im Betrag liegen sie jedoch tiefer.\\
Abschliessend kann zum Vergleich gesagt werden, dass die FEM-Ergebnisse plausieble sind.

\subsubsection{Beurteilung Dach}
In der folgenden Tabelle sind die Schnittgrössen der Träger des Daches enthalten.

\begin{table}[H]
\centering
\begin{tabular}{lccccccc}
\thickhline
  &	Einheit	&	1.1	&	1.3	&	1.4	&	1.5	&	Max	&	Min	\\	\hline
Axialkraft	&	N	&	1949	&	1551	&	-1731	&	-2450	&	1949	&	-2450	\\
Querkraft	&	N	&	108	&	55	&	14	&	32	&	108	&	14	\\
Biegemoment	&	kNmm	&	17	&	17	&	9	&	19	&	19	&	9	\\	\thickhline
\end{tabular}
\caption{Schnittgrössen des Daches in den unterschiedlichen Lastfällen}
\label{tab:FEMres Dach}
\end{table}


Wie im Kapitel \ref{sec:Dach} beschrieben, ist das dimensionierende Kriterium des Daches dessen Verformung aufgrund des Eigengewichtes. Dem entsprechend stellen die in der Tabelle \ref{tab:FEMres Dach} aufgeführten Schnittgrössen keine kritischen Lasten dar und werden hier nicht vertieft aufgegriffen.

Das Potential der Gewichtsoptimierung des Daches wird als gering eingestuft. Auch wenn die Träger des Daches global gesehen überdimensioniert sind, werden über sie, im eingefahrenen Zustand, die ausfahrbaren Seitenmodule befestigt und gesichert. Würde ein anderes Konzept zur Versperrung der ausfahrbaren Seitenmodulen ausgearbeitet, könnte das Dach eventuell auf eine andere Weise versteift (z.B. mit aufgeklebten CFK-Hutprofilen) und die Dachträger weggelassen werden.\\
Wird vom jetzigen Konzept noch weiter abgewichen und der Entwurf des unterbrochenen Daches (4 GFK-Sandiwichpanelen à ca. 2 x 1.3 m im Mittelkörper) verworfen, gäbe es allenfalls die Möglichkeit, ein durchgehendes Dach in Sandwichbauweise zu verwenden. Dieses könnte, ähnlich wie der Boden, mit Ocean-PET und Aluminium Deckschichten in einem Stück gefertigt und mit Hartschaum-Einsätzen und Verstärkungen individuell angepasst und optimiert werden. Der Nachteil diese Konzeptes ist jedoch, dass nicht die Standard-Solarmodule verwendet werden können, welche von Begin an des Projektes als vorgegeben betrachtet wurden (Sponsoring). Es stellt sich entsprechend die Frage, zum einen \emph{wer} und \emph{wie} die Solarzellen auf das Dach laminiert werden, da diese nicht direkt im Herstellungsprozess der Sandwichkonstuktion mitlaminiert werden können. Dass die Solarzellen unter Umständen ``von Hand'' auf das Dach laminiert werden müssen, könnte sich aufgrund der Flexibilität bezüglich Dimensionen und Verkabelung, auch als Vorteil erweisen. Als weiterer Vorteil ist zu ergänzen, dass die Verbindungsstellen zwischen den Solarpanelen und Träger, sowie zwischen den einzelnen Solarpanelen, wegfallen würden.\\
Auch wenn die Gewichtsersparnisse vermutlich gering sind (oder wohl möglich auch nicht vorhanden sind), würde die Komplexität des Daches potenziell reduziert werden können.


\subsubsection{Beurteilung Träger A und B}
In der Tabelle \ref{tab:FEMres Träger} sind die maximalen Schnittgrössen der Träger A und B zusammengestellt.
\begin{table}[H]
\centering
\begin{tabular}{lccccccc}
\thickhline
	&	Einheit	&	1.1	&	1.3	&	1.4	&	1.5	&	Max	&	Min	\\	\hline
Axialkraft	&	N	&	-10846	&	1551	&	2654	&	-4066	&	2654	&	-10846	\\
Querkraft	&	N	&	93	&	55	&	1071	&	1311	&	1311	&	55	\\
Biegemoment	&	kNmm	&	326	&	17	&	639	&	788	&	788	&	17	\\	\thickhline
\end{tabular}
\caption{Schnittgrössen der Träger in den unterschiedlichen Lastfällen}
\label{tab:FEMres Träger}
\end{table}


Die maximale Axialkraft von -10.8 kN hat, bei einer Querschnittsfläche eines Trägers von rund 1180 $mm^2$, Druckspannungen von 9.2 MPa zur folge. Die Gefahr des Knickens ist nicht vorhanden, da die Träger auf mindestens zwei Seiten über die Wände gestützt werden und die Druckbelastung im verhältniss zum Flächenträgheitsmoment des Trägers eher tief ist (Knickung nach Euler). Das Maximale Biegemoment von 772 kNmm führt, bei einem minimalen Widerstandsmoment von 11900 $mm^3$, zu Spannungen in der höhe von 65 MPa.\\
Ob die Dimensionen der Träger A und B optimal gewählt wurden lässt sich anhand der FEM-Ergebnissen nicht beurteilen, da angenommen wird, dass die dimensionierenden Belastungen währen dem Ausfahren der Seitenmoudlen (Modus B3) auftreten. Wie im Kapitel KAPITEL vorgeschlagen, wird unter anderem empfohlen in einem weiteren Schritt ein globales FEM-Modell für den Modus B3 zu erstellen und dieses zu analysieren. Es kann jedoch gesagt werden, dass die überprüften Lastfälle keine kritischen Belastungen für die Träger A und B darstellen, diese jedoch auch nicht überdimensioniert sind. Dem entsprechend kann das Potential zur Gewichtseinsparung nur bedingt abgeschätzt werden.

\subsubsection{Verbindung Boden zu Chassis}
In der folgenden Tabelle sind die maximalen Kontaktreaktion und Spannungen der Verbindung zwischen Chassis und Boden zu finden.

\begin{table}[H]
\centering
\begin{tabular}{lcccccc}
\thickhline
	&	Einheit	&	1.1	&	1.3	&	1.4	&	1.5	&	Max	\\	\hline
Normalkraft (Zug)	&	N	&	883	&	288	&	1942	&	3118	&	3118	\\
Schubkraft (xz-Ebene)	&	N	&	9933	&	1731	&	10972	&	10761	&	10972	\\	\hline
Normalspannungen	&	MPa	&	0.05	&	0.02	&	0.11	&	0.17	&	0.17	\\
Schubspannungen	&	MPa	&	0.56	&	0.10	&	0.61	&	0.60	&	0.61	\\	\thickhline
\end{tabular}
\caption{Schnittgrössen und Spannungen der Verbindung zwischen Chassis und Boden in den unterschiedlichen Lastfällen}
\label{tab:FEMres Boden}
\end{table}


Wird die Klebefläche des Chassis auf die 60 Knotenverbindungen verteilt ergibt sich eine Fläche von 17880 $mm^2$ pro Knotenverbindung. Mit den in der Tabelle \ref{tab:FEMres Boden} angegebenen Kontaktreaktionen ergeben sich somit maximale Normalspannungen von 0.17 MPa und maximale Schubspannungen von 0.61 MPa. Die Normalspannungen liegen unterhalb den Design-Allowables und die Schubspannungen deutlich darüber. Hierbei muss zusätzlich angemerkt werden, dass aufgrund der mangelnden Auflösung (30 Knoten pro Längsträger des Chassis verteilt auf ca. 9 m) und nicht optimaler Modellierung, lokal die Spannungen deutlich höher liegen könnten und dass das verwendete Modell nicht geeignet ist um diese Spannungskonzentrationen fest zu stellen.\\
Würde die FEM-Berechnung erneut durchgeführt werden, wird empfohlen, das Chassis mittels Schalenkörper zu modellieren, wodurch eine Kontakt-Verbindung (an Stelle einer Joint-Verbindung) verwendet werden könnte. Eine weitere Möglichkeit währe, die Klebeverbindung mittels MPC-Kontakten (MPC184 Elementen) mit definierbarer Steifigkeit zu modellieren, was als eine exaktere Modellierung erachtet wird.

Um die Klebeverbindung und die darin erlangten Spannungen besser beurteilen zu können, müssten unterschiedliche Klebstoffe in betracht gezogen und deren Design-Allowables bestimmt werden. Es wurde sich nicht vertieft mit Klebstoffen auseinandergesetzt, sodass die Klebstoff-Design-Allowables keine abschliessende Werte darstellen. Dennoch werden sie als gute Näherung erachtet.\\
Auch wenn bessere Klebstoffe gefunden werden könnten ist die Klebeverbindung, aufgrund der vielen unbekannten Grössen und mangelnder Erfahrung, als Risiko zu betrachten und muss in einem weiteren Schritt genauer untersucht werden. Für das weitere Vorgehen wird, je nach zur verfügung stehender Zeit, empfohlen auf eine alternative Verbindngsmethode wie Nieten oder Schrauben zu wechseln oder Personen und Firmen mit der benötigten Erfahrungen auf diesem Gebiet dem Projekt zur Unterstützung bei zu ziehen.\\
Es müssen auf jeden Fall weitere Untersuchungen und Abklärungen vorgenommen werden bevor ein definitiver Entscheid getroffen oder mit der Herstellung des Solar Butterflys begonnen wird.

\subsubsection{Verbindung Träger A und B zu Chassis}
In der Tabelle \ref{tab:FEMres Träger Kont} im Anhang \ref{sec:FEMres Träger} sind die maximalen Kontaktreaktionen der Fixed-Joint-Verbingungen zu finden.\\
Die Idealisierung der Verbindungen zwischen den Trägern A und B und dem Chassis stellt eine schlechte Abbildung der wirklichen Verbindung dar. So sind die Träger A und B zum Beispiel in der Realität direkt am Chassis befestigt nicht wie modelliert, mit einem Abstand von ca. 400 mm (vlg. Abbildung \ref{FEM Mesh3}). Dieser zusätzliche Hebelarm verfälscht Biegemomentreaktionen sehr, sodass zu ihnen keine Aussagen gemacht werden.\\
Bei den Axialkräften der Verbindung wird angenommen, dass diese nicht stark verfälscht werden und dass sie für eine grobe Auslegung der Verbindung benütz werden können. Die maximale Axialkraft tritt mit 15.8 kN in negativer y-Richtung (Nach unten) im Lastfall \emph{1.1 VertikaleBeschleunigung} auf.


\subsubsection{Deformationen}
\label{Deformation}
Die FEM-Berechnungen zeigen, dass das Chassis, im Bezug auf das Übernehmen von Lasten, eine wichtigere Funktion übernimmt, als zuvor angenommen. Die Funktion es Aufbaus wurde wiederrum überschätzt. Diese Feststellung lässt sich unteranderem an der Abbildungen \ref{Def3} anhand den Deformationen erkennen. Das Chassis verformt sich realtiv stark, während der Aufbau seine rechteckige Form nahezu bei behält. Besonders in den Lastfällen der lateralen und rotatorischen Beschleunigung ist zu erkennen, dass sich lediglich das Chassis stark verdreht, und nicht wie angenommen der komplette Solar Butterfly. Dies zeigt, dass die Eigenschaft des Chassis bezüglich Steifigkeit, im Vergleich zum Aufbau, eine entscheidende Rolle spielt. \\

\begin{figure}[H]
  \centering
  \includegraphics[width=.98\linewidth]{04_figures/Def3.png}
  \caption{Deformation des Solar Butterflys in den  Belastungsfällen 1.1 Vertikale Beschleunigung (links) und 1.5 Rotatorische Beschleunigung (rechts)}
  \label{Def3}
\end{figure}

Es ist jedoch nicht klar, ob dieses Ergebniss zum Teil auch auf die Art der Einbindung der Punktmassen zurück zu führen ist. Oder anderst ausgedrückt: es ist nicht klar, ob das selbe Ergebnis erzielt werden könnte, wenn die Massen realitätsgetreuer modelliert und ins Modell eingebunden worden wären. So befindet sich in der Realität ein grösserer Teil der Masse, in Form der ausfahrbaren Solarmodulen und den dazugehörigen Antriebselementen, an den Wänden des Solar Butterflys und nicht, wie modelliert, in den Zentern der Raumelemente.
Die Masse der ausfahrbaren Solarmodulen muss über die Wände und Träger A und B, zu einem gewissen Ausmass auch über das Dach, getragen und dessen Trägheitskräfte auf das Chassis übertragen werden. Die Punktmassen sind jedoch fast ausschliesslich direkt über das Chassis und die Träger A und B befestigt worden. Durch eine exaktere Verteilung und Einbindung der Massen ins Modell, würde sich der Lastpfad entsprechend verändern, was andere FEM-Ergebnisse hervor bringen würde.
Es ist wahrscheinlich, dass der Aufbau in der realität eine tragendere Funktion übernimmt, als dies durch die FEM-Berechnungen gezeigt wird und dass dessen Deformation stärker ausfallen würde.\\
Auch wenn mit einer exakteren Modellierung gezeigt werden könnte, dass der Aufbau eine wichtigere Rolle übernimmt als dies durch die vorliegenden FEM-Berechnungen nahe gelegt wird, steht fest, dass die Eigenschaften des Chassis das Verhalten des Solar Butterflys dominieren.\\

Das Gewicht des Chassis beansprucht mit 650 kg rund ein Viertel der Gewichtslimite für Europa von 2200 kg. Weiter handelt es sich beim Chassis um ein ``Standard-Chassis'', welches nicht spezifisch für die Anwendung in diesem Projekt ausgelegt und optimiert wurde. Ferner wird der Boden zur Zeit nicht optimal ausgenützt. So entspringt die dimensionierende Grösse des Bodens aus einem Missbrauchslastfall (``Spitzer Schuh'') und nicht aus dessen Funktion als tragendes Strukturelement. In der Ausarbeitung des Konzeptes wurde der Boden als einzelnes Bauteil, und nicht, in Verbindung mit dem Chassis, als integraler Bestandteil der tragenden Struktur betrachtet. Folglich wird das grösste Potential zur Gewichtsreduktion im Bereich der Grundstruktur, in der Optimierung des Chassis in Kombination mit dem Boden gesehen.\\
Es ist zu vermuten, dass durch die Optimierung die Klebeverbindung zwischen Chassis und Boden noch stärker beansprucht werden würde, als dies bereits der Fall ist. Ob eine Klebeverbindung noch immer eine angemessene Wahl ist, müsste in der Optimierung genauer beurteilt werden.

Keine kritischen Verformungen festgestllt. Max. 15mm, unkritisch:)
\newpage

  \section{Diskussion}
\label{Diskussion}

\subsection{Lastenheft}
Beim Durchführen der Handrechnungen und der Auswertung der FEM-Berechnungen hat sich gezeigt, dass die Lastfälle \emph{1.1 Vertikale Beschleunigung} und \emph{1.5 Rotatorische Beschleunigung} die stärksten Belastungen für den Solar Buttefly darstellen. Zugleich weisen diese beiden Lastfälle die grössten Risiken auf, da sie auf groben Annahmen beruhen und nur mit grösserem Aufwand exakter abgeschätzt werden können.\\
Bei der Festlegung der Lasten im Lastenheft wurde eine tendenziell konservative Position eingenommen, so dass die Beschleunigungen eher zu hoch als zu tief liegen. Folglich ist es wahrscheinlich, dass durch eine Überarbeitung des Lastenheftes und der genaueren Bestimmung der kritischen Beschleunigungen, tiefere Werte bestimmt werden können und der Solar Butterfly exakter für die auftretenden Belastungen ausgelegt werden kann.\\
Sofern dem Projekt genügen Ressourcen und Zeit zur Verfügung stehen und sich dafür entschieden wird das Lastenheft zu überarbeiten, wird empfohlen, den Solar Butterfly als Feder-Dämpfer-Modell zu Modellieren und die Beschleunigungen für verschiede Bedingungen und Situationen zu ermitteln. Mit den gewonnenen Erkenntnissen könnte sich Gewissenheit bezüglich den bei der Ausarbeitung des Lastenheftes getroffenen Annahmen verschaffen werden und die Ungenauigkeit - und somit auch das Risiko - der Lastfälle minimiert werden. Dies würde eine präzisere und vorallem sichrere (weniger Unsicherheiten) Auslegung ermöglichen, was wiederrum mit einer eventuellen Reduktion des Gewichtes verbunden werden kann.\\
Es muss betont werden, dass das Lastenheft eine erste Abschätzung der Lasten darstellt und nicht als definitive oder abschliessende Beurteilung betrachtet werden soll.
Auch wenn die Aussagekraft des Lastenheftes hier zu einem gewissen Ausmass in frage gestellt wird, wird das Lastenheft als eine gute Einschätzung der Lasten erachtet und kann, unter Berücksichtigung der einhergehenden Risiken, für den weitere Verlauf des Projektes verwendet werden.

Den Fokus auf andere Punkte legen.

\subsection{Gebrauchsanleitung}
Ein weiterer Punkt, welcher als notwendig erachtet wird, ist die Ausarbeitung einer ``Gebrauchsanleitung'' in welcher der bedienenden Person klar beschrieben wird, wie der Solar Butterfly zu benützen ist, was ``erlaubt'' ist und was nicht und wie sich in spezifischen Situationen (z.B. bei Wind oder technischen Defekten) verhalten werden soll. Einige dieser Punkte sind im Lastenheft und der Anforderungsliste bereits definiert. Diese Dokumente sind in dieser hinsicht jedoch keines Wegs vollständig, decken nicht annähernd alle Szenarien ab und sind für die Funktion als Gebrauchsanleitung nicht passend strukturiert. Die Gebrauchsanleitung könnte auch als eine Zustammenstellung von Informationen der verschiednene Arbeit, welche für die Bedienung des Solar Butterflys relevant sind, betrachtet werden. Weiter kann mit einer Gebrauchsanleitung das Eintreten von gewissen Missbrauchslastfällen verhindert werden, da die Benützung engeschränkt wird.\\
Eine Gebrauchsanleitung müsste unter anderem folgende Fragestellungen beantworten können.
Was ist die maximale Zuladung und an welcher Stelle darf sich diese im Solar Butterfly befinden? (Schwerpunkt)
Was sind die erlaubten Windgeschiwndigkeiten für die unterschiedlichen Positionen in welchen sich der Solar Butterfly befinden kann? Wann darf zum Beispiel die Panelenreihe C oder D nicht mehr ausgefahren sein?
Wie können die Seitenmodule bei ausgefallener Pneumatik ein- und ausgefahren werden? Muss die Pneumatik unterhalten und gewartet werden?
Wie muss der Solar Butterfly im Fahr-Modus versperrt und befestigt werden dass während einer Fahrt keine Gefahren entstehen?

Eine Gebrauchsanleitung aus zu arbeiten ist an dieser Stelle des Projektes nicht dringend notwendig und auch schwierig zu bewerkstelligen, da viele Komponenten noch nicht definitig ausgewählt wurden. Zu einem späteren Zeitpunkt und weiter vortgeschrittenen Projektstand ist die Erstellung einer Gebrauchsanleitung jedoch empfehlenswert. Dies nicht nur um den Solar Butterfly sicherer zu machen, sondern auch, um dessen Bedienung zu vereinfachen. Dass der Solar Butterfly von verschiedenen Personen gefahren und bedient werden soll, spricht ebenfalls für das Erstellen Gebrauchsanleitung.

\subsection{Auslegung}

\subsection{Gewichtsoptimierung}

ermüdung nochmals abchecken. Nachweis erbringen usw.
Vorgehen:
Steifigkeiten von Chassis ermitteln: Research oder Auswertung bereits verwendeter Chassis von Geser.
Steifigkeitskriterien bestimmen.
Was ist überhaupt möglich? zum einen Verbindungsarten und Chassis. Wandstärken, aussparungen usw.
Problematik: Wärmeausdehnung in bezug auf Boden
Verbindungsmethoden zwischen Boden und Chassis. Befragung von Leuten mit erfahrungen. Kleben, nieten usw???

Dynamik?

Rechtliche Problemaitk? (Autobahn usw.) Nachweis erstellen.







\newpage

\section{Fazit}


\newpage

\section{Danksagung}
Ohne die Unterstützung folgender Personen wäre mir das Ausarbeiten der Bachelorarbeit in dieser Form nicht möglich gewesen. Dafür möchte ich an dieser Stelle meinen Dank aussprechen an:\\
Dejan Roman\v{c}uk, für die Betreuung und Unterstützung bei der Durchführung dieser Arbeit.\\
Damian Gwerder und David Schiffmann für das Durchführen und Auswerten der CT-Scans.\\
Simon Gerig und Marcel Furrer für das Herstellen der Proben.\\
Ruedi Pflugshaupt und sein Team für das mechanische Bearbeiten der Proben und Herstellen\\
der Probenhalterung\\
\newpage


% ==================== Backmatter ====================
  \part{Anhang}
  \appendix
  \section{Quellenverzeichnis}

\renewcommand\refname{\vskip -1cm}
\bibliography{03_Backmatter/mybib}
\bibliographystyle{ieeetr}

  \section{Abbildungsverzeichnis}
\renewcommand\listfigurename{}
\vspace*{-1cm}
\listoffigures

  \section{Tabellenverzeichnis}
\renewcommand\listtablename{}
\vspace*{-1cm}
\listoftables
\newpage

  \section{Rissfortschritt}

\subsection{Zeichnungen}
  \label{Zeichnungen}

  \subsubsection{Zeichnung des Probenrohlings - Erste Serie}

  \subsubsection{Zeichnung des Probenrohlings - Zweite Serie}


  \part{Elektronischer Anhang}
  \section{Bilder des Solar Butterflys}
\label{Bilder des Solar Butterflys}

\section{Dokumente aus fremden Arbeiten}
  \subsection{Anforderungsliste}
  \label{e:Anforderungsliste}
  \subsection{Gewichtsberechnung}
  \label{e:Gewichtsberechnung}




\section{Datenblätter}
  \subsection{Materialien}
  \label{e:Materialien}
    \subsubsection{TDS Airex-T92}
    \label{Airex}
    \subsubsection{TDS Sikafles 552-AT}
    \label{Sikaflex}


  \subsection{Komponenten}
    \subsubsection{Federkonstante}
    \label{Federkonstante}




\section{Berechnungen}
\label{e:Berechnungen}
  \subsection{Lastenheft - Beschleunigungen}
  \label{e:Lastenheft}
  \subsection{Handrechnungen}
  \label{e:Handrechnungen}
  \subsection{Dimensionierung}
  \label{e:Dimensionierung}

\end{document}
