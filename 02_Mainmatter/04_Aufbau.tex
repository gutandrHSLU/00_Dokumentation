\section{Komponenten und Verbindungen}
In diesem Kapitel wird beschrieben, wie der Solar Butterfly aufgebaut ist. Es werden verschiedene Komponenten eingeführt und analysiert wie diese Komponenten miteinander Verbunden sind und welche Kräfte die Verbindungen übertragen müssen.\\
Weiter wird beschrieben, wie der SB vereinfacht betrachtet wird (Biegebalken) in zwei Moden. (A und C)

\subsection{Komponenten}
bla bla

\paragraph{Hauptkörper}
Ganzer Körper als einen Kasten betrachten
\begin{description}
  \item \textbf{Chassis}\\
  Idealisierung: Beam
  \item \textbf{Boden}\\
  Auslegung: Biegebalken
  Idealisierung: Schalenkörper\\
  \item \textbf{Stützen A und B}\\
  Auslegung: Schubwand mit Türe\\
  Profile nehmen Kräfte auf, geben diese Jedoch an die Schubwand weiter\\
  \item \textbf{Dach}\\
  Panelen: Schubfläche\\
  Dach an sich: Biegebalken (Durch eigengewicht)
  Im Modus \emph{C} Kräfte auf nehmen durch Verriegelung der Seitenwände\\
  \item \textbf{}
\end{description}

\paragraph{Seitenmodul}
\begin{description}
  \item \textbf{Boden}\\
  Biegebalken und Schubfläche
  \item \textbf{Seitenwand}\\
  Modus A: Schubwand\\
  Modus C: Keine
  \item \textbf{Ausfahrmechanismus (Scharniere)}\\
  Wand: Schubwand
  \item \textbf{}\\
  \item \textbf{}\\
\end{description}
