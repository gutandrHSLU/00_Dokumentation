\section{Einleitung}
Der Klimawandel äussert sich in der Schweiz überdurchschnittlich. So ist die mittlere Jahrestemperatur in der Schweiz seit Messbeginn im Jahre 1864 um 2 °C gestiegen, was rund doppelt so stark ist, wie das globale Mittel. In der Schweiz wird rund ein Drittel aller Treibhausgasemissionen durch den Verkehr (ohne internationalen Flug- und Schiffsverkehr) verursacht \cite{BAFU}. Um das \emph{Netto-Null-Ziel} der \emph{Langfristigen Klimastrategie der Schweiz} zu erfüllen, müssen daher unteranderem im Verkehrssektor Veränderungen vorgenommen und Entwicklungen getätigt werden.\\
Louis Palmer, ein Schweizer Umweltaktivist und ``\emph{Macher}'', umrundete im Jahre 2004 als erster mit einem Elektrofahrzeug - dem Solarfahrzeug \emph{Solartaxi} - die Erde und gilt somit als ein Pionier im Bereich der Elektromobilität \cite{Palmer}.\\
Sein neustes Projekt ist der \emph{Solar Butterfly} - ein autarker Wohnwagen, mit welchem er ``eine Reise zu den Klimalösungen dieser Welt [...] im ersten solarbetriebenen <<Mobile Home>> der Welt'' antreten will. Ein Ziel von Palmer ist es, mit dem Solar Butterfly weltweite Aufmerksamkeit zu erregen und so nachhaltige Lösungen im Bereich des Klimaschutzes und Elektromobilität zu ermutigen und voranzutreiben.\\
Die erneute Weltumrundung soll dieses Mal mit ``etwas mehr Komfort'' geschehen. Seine Vision ist es, ein Wohnwagen, mit zwei Ausziehbaren Wohnmodulen mit rund 100 $m^2$ integrierte Photovoltaikfläche, zu realisieren. Der Wohnwagen soll sich selbst mit Solar-Energie versorgen und autonom operiert werden können. Im Rahmen dieser Bachelorarbeit soll, in Zusammenarbeit mit drei weiteren Studenten der HSLU T\&A, seine Vision des Solar Butterflys in die Realität umgesetzt werden.\\
Das Projekt wurde neben dieser Arbeit in die weiteren Teilgebiete \emph{Auslegung Klappmechanismen}, \emph{Auslegung Antriebstechnik} und \emph{Auslegung Solar Butterfly (Globales CAD)} aufgeteilt.\\
Das Auslegen der Klappmechanismen beinhaltet das Entwerfen und Dimensionieren aller beweglichen Teilen wie die klappbaren Panelen und den Ausfahrmechanismus der Seitenmodulen. Die Arbeit \emph{Auslegen der Antriebstechnik} befasst sich mit der Technik, mit welcher die beweglichen Bauteile in Bewegung gesetzt werden. Im Teilgebiet \emph{Auslegung Solar Butterfly (Globales CAD)} werden die jeweiligen Teilgebiete zusammengeführt. Ebenfalls beinhaltet diese Aufgabenstellung das Erstellen eines globalen CAD-Modells, das Zusammentragen der allgemeinen Anforderungen sowie eine Risikobewertung des Projektes.\\
% Diese Arbeit, welche zum Teilgebiet \emph{Auslegung Grundstruktur Solar Butterfly} gehört, befasst sich mit der Festlegung der Auslegungskriterien, der Ausarbeitung eines detailierten Lastenheftes sowie mit der Analyse des Konzeptes.

\subsection{Aufgabenstellung}
Der Fokus dieser Arbeit liegt im Festlegen der Auslegungskriterien, dem Ausarbeitung eines detailierten Lastenheftes, sowie mit der Dimensionierung der Grundstruktur. Zur Bestimmung von Schnittgrössen, mit welchen Handrechnungen getätigt oder verifiziert werden können, soll dabei ein globales FEM-Modell zur Anwendung kommen. Ebenfalls sollen zulässige Festigkeitswerte abhängig von der gewählten Bauweise abgeschätzt werden (Design-Allowables).\\
Weiter beinhaltet die Aufgabenstellung eine enge Zusammenarbeit mit den Mitstudenten. Es soll sich aktiv an der Lösungsfindung beteiligt und dabei besonders die Aspekte der strukturellen integrität berücksichtigt und vertreten werden.

\subsection{Vorgehen und Methodik}
In diesem Kapitel wird beschrieben, wie beim Lösen der Aufgabenstellung vorgegangen wird. Die Struktur des vorliegenden Dokumentes entspricht dabei dem nun vorgsetellten Vorgehen.\\

In einem ersten Schritt wird definiert, welchen Anforderungen der Solar Butterfly, von einem Standpunkt der strukturellen integrität aus betrachtet, gerecht werden muss. Weiter werden die Auslegungskriterien bestimmt. Sie beschreiben im Detail, nach welchen Kriterien die einzelnen Komponenten des Solar Butterflys ausgelegt werden. So wird zum Beispiel beschrieben, welche Kriterien die Sandwichplatten erfüllen müssen, dass diese unter Belastung nicht bäulen.\\
Anschliessend wird ein Lastenheft erstellt, welches eine Zusammenstellung von verschiedenen Lastfällen darstellt, welchen der Solar Butterfly ausgesetzt werden kann. Für diese Lastfälle - und Kombinationen davon - wird der Solar Butterfly ausgelegt.\\
Als nächster Schritt wird der Solar Butterfly grob idealisiert und für die kritischen Lastfälle Handrechnungen durchgeführt, um so Kräft und Spannungen bestimmen zu können. Dies wird zum einen gemacht, um die Grössenordnung der Kräfte besser abschätzen zu können. Andererseits kann dadurch eine erste grobe Dimensionierung der wichtigsten Komponenten erfolgen.\\
Weiter werden die festigkeitstechnischen Funktionen der einzelnen Komponenten analysiert. Es wird zum Beispiel analysiert welche Funktionen das Dach des Solar Butterfly übernehmen muss und wie dieses Idealisiert betrachtet werden kann. Das Ergebniss dieser Analyse ist das erlangte Verständniss für Belastungsarten und idealisierte Kraftverläufe durch die Komponenten und Struktur des Solar Butterflys für verschiedene Lastfälle. Mit der Hilfe dieser Analyse können die verschiedenen Komponenten grob ausgelegt und Verbindungen zwischen den Komponenten optimal konstruiert werden.\\
In einem letzten Schritt wird der Solar Butterfly in FEM-Analysen verschiedenen Lastkombinationen ausgesetzt um so Lastpfade und Schnittkräfte zu bestimmen, anhand welchen eine Verifizierung der Handrechnungen und eine genauere Dimensionierung der Komponenten erfolgen kann. Weiter können in den FEM-Analysen für die Funktionstauglichkeit kritische Verformungen festgestellt werden, welche in der Konstruktion berücksichtigt werden müssen.\\

Hierbei handelt es sich um ein stark iteratives Vorgehen. Erkenntnisse in einem Bereich des Projektes haben Auswirkungen


\subsection{Theorie}
Leichtbau:

Als Einschränkung ist dabei zu berücksichtigen, dass hierdurch weder die Funktion noch die Sicherheit und Langlebigkeit /s. DIN EN 1993/ beeinträchtigt werden dürfen. Maßnahmen, mit denen man dies heute zu erreichen versucht, sind:
- Umsetzung des Integrationsprinzips,
- Wahl leichter und hochfester Werkstoffe,
- neue Herstelltechnologien
- analytische Beherrschung der Beanspruchungs- bzw. Instabilitätsfälle durch hochwertige Analysemethoden (FEM, BEM).

Im Zuge der Umsetzung dieser Prinzipien kommen bestimmte Entwurfsstrategien /BLE 74/ zum Tragen, deren Merkmale sich verkürzt klassifizieren lassen in
  einen Form- oder Funktionsleichtbau, bei dem integrative Konstruktionslösungen, dünnwandige Querschnittsgeometrien und eindeutige Kraftleitungspfade umgesetzt werden;
  einen Stoffleichtbau, bei dem spezifisch schwere Werkstoffe durch leichtere Werkstoffe mit möglichst hohen Gütekennzahlen substituiert werden;
  einen Fertigungsleichtbau, in dem alle technologischen Möglichkeiten ausgeschöpft werden, um das Ziel der Funktionsintegration (Einstückigkeit) bei geringstem Materialeinsatz und minimalem Fügeaufwand zu realisieren
und
  einen Sparleichtbau, mit dem Ziel hohe Kosten zu vermeiden durch eine gerade noch ausreichende Werkstoffqualität, minimalem Werkstoffeinsatz und vereinfachte Herstellung.
(S16)

Da ein typisches Einsatzgebiet von Leichtbaukonstruktionen die Verkehrstechnik (Automobilbau, Schienen- und Luftfahrzeuge) ist, dürfen Leichtbaukonstruktionen nicht „unsicherer“ als vergleichbare Massivkonstruktionen sein. Dies bedingt eine sorgfältige Auslegung auf Steifigkeit (Instabilitäten), Bruchfestigkeit sowie Zuverlässigkeit und Nutzungsdauer. (S20)

Die Philosophie des „safe-life-quality“, die absolute Schadensfreiheit für das ganze Leben verlangt, und die Philosophie des „fail-safe-quality“, die Schadenstoleranz und hinreichende Resttragfähigkeit voraussetzt. Dem Ziel nach sollten alle erforderlichen Leichtbaumaßnahmen begründbar sein.(S21)
Auslegungsphilosophie:
  Safe-Life-Quality:
    Absolute Schadensfreiheit für die angestrebte Lebensdauer
    Statistische Ausfallwahrscheinlichkeit
  Fail-Safe-Quality:
    Schadenstolerant
    Hinreichende Resttragfähigkeit

aufeinander aufbauende Arbeitsschritte mit etwa folgenden Inhalten:
  - Klären der Aufgabenstellung: Informationsbeschaffung über die Anforderungen einer Aufgabe und Erstellung einer Anforderungsliste; Eingrenzung bestehender Bedingungen und ihre Bewertung für die Lösungserfüllung; Festlegung einer Lösungsrichtung; technisch-wirtschaftliche Konsequenzen.
  - Konzipieren (Findung einer prinzipiellen Lösung): Hinterfragung der Aufgabe und Sichten des Kernproblems; Zerlegung des Kernproblems in untergeordnete Teilprobleme; Suche nach Lösungswegen zur Erfüllung der Teilprobleme; Kombination der Teilproblemlösungen zu Lösungsansätzen für das Kernproblem; Bewertung der Lösungen; Erstellung von Konzeptskizzen. Voraussetzungen einer sinnvollen Konzepterstellung sind Kenntnisse über die Größe und Richtung der wirkenden Kräfte, die Möglichkeiten des gewählten Werkstoffs, die Bauweiseneigenschaften und eine angepasste Vordimensionierung. Ein gutes Konzept ist letztlich auch der Garant für eine innovative Problemlösung. Der Konzeptentwicklung sollte daher große Bedeutung beibemessen werden.
  - Entwerfen (gestalterische Konkretisierung einer Lösung): maßstäbliche Ausarbeitung der Konzeptskizzen zu Bauvarianten; Bewertung, Vereinfachung und Auswahl einer Variante; Überarbeitung zu einem Gesamtentwurf und
  - Ausarbeiten (fertigungs- und montagegerechte Festlegung einer Lösung): endgültige Bestimmung der Geometrie, Dimensionen, Werkstoffe und Herstellung, um die notwendigen Fertigungsunterlagen erstellen zu können.

Hieran schließen sich eine oder mehrere Schleifen an, die der Optimierung der Lösung dienen. Dem zuzuordnende Phasen sind:
  - Prototypen-Herstellung (Kontrolle der Funktionen, Montage etc.),
  - Testprozeduren (Überprüfung der Tragfähigkeit, Zuverlässigkeit, Lebensdauer).

FEM
  Die FEM ist eine rechnerorientierte Methode, die softwaretechnisch über einen Vorrat an mechanischen Grundelementen (Balken, Scheibe, Platte, Schale, Volumina), einen Zusammenbau- und einen Lösungsalgorithmus verfügt.

S206 Abb.

\subsection{Der Solar Butterfly}
Ziel diese Unterkapitels ist es, einen Überblick des Solar Butterflys zu verschaffen. Die funktionalität
Ziel: Überblick vermitteln. Funktionalität veranschaulichen, Begriffe Definieren.

Chassis, Hauptkörper, Seitenteil, Küche, Bad, Stützen, Panelen Gross, Panelen Klein

Verbindung von Dach zu den Raumelementen ref Kap. Dach - mittleres Raumelement
Europa Gewicht von 2200 kg
Rest: Gewicht von  3000 kg
\newpage
