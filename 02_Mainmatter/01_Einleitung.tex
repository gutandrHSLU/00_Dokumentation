\section{Einleitung}
\cite{Huber}
Der Klimawandel äussert sich in der Schweiz überdurchschnittlich. So ist die mittlere Jahrestemperatur in der Schweiz seit Messbeginn im Jahr 1864 um 2 °C gestiegen, was rund doppelt so stark ist, wie das globale Mittel. In der Schweiz wird rund ein Drittel aller Treibhausgasemissionen durch den Verkehr (ohne internationalen Flug- und Schiffsverkehr) verursacht \cite{BAFU}. Um das \emph{Netto-Null-Ziel} der \emph{Langfristigen Klimastrategie der Schweiz} zu erfüllen, müssen daher unteranderem im Verkehrssektor Veränderungen vorgenommen und Entwicklungen getätigt werden.\\
Louis Palmer, ein Schweizer Umweltaktivist und ``\emph{Macher}'', umrundete im Jahr 2004 als erster mit einem Elektrofahrzeug - dem Solarfahrzeug \emph{Solartaxi} - die Erde und gilt somit als ein Pionier im Bereich der Elektromobilität \cite{Palmer}.\\
Sein neustes Projekt ist der \emph{Solar Butterfly} - ein autarker Wohnwagen, mit welchem er ``eine Reise zu den Klimalösungen dieser Welt [...] im ersten solarbetriebenen <<Mobile Home>> der Welt'' antreten will. Ein Ziel von Palmer ist es, mit dem Solar Butterfly weltweite Aufmerksamkeit zu erregen und so nachhaltige Lösungen im Bereich des Klimaschutzes und Elektromobilität zu ermutigen und voranzutreiben.\\
Die erneute Weltumrundung soll dieses Mal ``mit etwas mehr Komfort'' geschehen. Seine Vision ist es, ein Wohnwagen, mit zwei Ausziehbaren Wohnmodulen und rund 100 $m^2$ integrierte Photovoltaikfläche, zu realisieren. Der Wohnwagen soll sich selbst mit Solar-Energie versorgen und autonom operiert werden können. Im Rahmen dieser Bachelorarbeit soll, in Zusammenarbeit mit drei weiteren Studenten der HSLU T\&A, seine Vision des Solar Butterflys in die Realität umgesetzt werden.\\
Das Projekt wurde neben dieser Arbeit in die weiteren Teilgebiete \emph{Auslegung Klappmechanismen}, \emph{Auslegung Antriebstechnik} und \emph{Auslegung Solar Butterfly (Globales CAD)} aufgeteilt.\\
Das Auslegen der Klappmechanismen beinhaltet das Entwerfen und Dimensionieren aller beweglichen Teilen wie die klappbaren Panelen und den Ausfahrmechanismus der Seitenmodulen. Die Arbeit \emph{Auslegen der Antriebstechnik} befasst sich mit der Technik, mit welcher die beweglichen Bauteile in Bewegung gesetzt werden. Im Teilgebiet \emph{Auslegung Solar Butterfly (Globales CAD)} werden die jeweiligen Teilgebiete zusammengeführt. Ebenfalls beinhaltet diese Aufgabenstellung das Erstellen eines globalen CAD-Modells, das Zusammentragen der allgemeinen Anforderungen sowie eine Risikobewertung des Projektes.\\


\subsection{Aufgabenstellung}
\label{Aufgabenstellung}
Der Fokus dieser Arbeit liegt in der Festlegung der Anforderungen und Auslegungskriterien, der Ausarbeitung eines detailierten Lastenheftes, sowie in der Dimensionierung der Grundstruktur. Zur Bestimmung von Schnittgrössen, mit welchen Handrechnungen gemacht oder verifiziert werden können, soll dabei ein globales FEM-Modell zur Anwendung kommen. Ebenfalls sollen zulässige Festigkeitswerte abhängig von der gewählten Bauweise abgeschätzt werden (Design-Allowables).\\
Weiter beinhaltet die Aufgabenstellung eine enge Zusammenarbeit mit den drei weiteren Mitstudenten. Es soll sich aktiv an der Lösungsfindung und weiteren Ausarbeitung des Konzeptes beteiligt und dabei besonders die Aspekte und Position der strukturellen integrität berücksichtigt und vertreten werden.

\subsection{Vorgehen und Methodik}
In diesem Kapitel wird beschrieben, wie beim Lösen der Aufgabenstellung vorgegangen wird. Die Struktur des vorliegenden Dokumentes entspricht dabei dem nun vorgsetellten Vorgehen.

In einem ersten Schritt wird definiert, welchen Anforderungen der Solar Butterfly als Ganzes, von einem Standpunkt der strukturellen Integrität aus betrachtet, gerecht werden muss. Weiter werden die Auslegungskriterien bestimmt. Sie beschreiben im Detail, nach welchen Kriterien die einzelnen Komponenten des Solar Butterflys ausgelegt werden. So wird zum Beispiel beschrieben, welche Kriterien die Sandwichplatten erfüllen müssen, dass diese unter Belastung nicht bäulen.\\
Anschliessend wird ein Lastenheft erstellt, welches eine Zusammenstellung von verschiedenen Lastfällen darstellt, welchen der Solar Butterfly ausgesetzt werden kann. Ferner werden sogenannte \emph{Modi} eingeführt, welche die Zustände und Positionen, in welchen sich der Solar Butterfly befinden kann, beschreiben. Für die Lastfälle des Lastenheftes - und Kombinationen davon - soll der Solar Butterfly in den verschiedenen Modi ausgelegt werden.\\
Als nächstes wird der Solar Butterfly grob als ``Balken'' idealisiert. Es werden für die kritischen Lastfälle Handrechnungen durchgeführt, um so Kräft und Schubflüsse bestimmen zu können. Dies wird zum einen gemacht, um die Grössenordnung der Lasten besser abschätzen zu können. Andererseits kann dadurch eine erste grobe Dimensionierung der wichtigsten Komponenten erfolgen und die zu diesem Zeitpunkt bereits getroffenen Annahmen bezüglich den Lasten überprüft werden.\\
In einem letzten Schritt wird der Solar Butterfly in FEM-Berechnungen verschiedenen kritischen Lastfällen ausgesetzt, um so Lastpfade und Schnittkräfte zu bestimmen, anhand welchen eine Verifizierung der Handrechnungen und eine exaktere Dimensionierung der Komponenten und Verbindungen erfolgen kann. Weiter können in den FEM-Berechnungen für die Funktionstauglichkeit kritische Verformungen festgestellt werden, welche in der weiteren Ausarbeitung des Konzeptes berücksichtigt werden sollen.\\
Im Kapitel \ref{Fazit} werden zum Schluss die erlangten Erkenntnisse zusammengetragen und Empfehlungen für das weitere Vorgehen abgegeben.

\subsection{Der Solar Butterfly}
Ziel diese Unterkapitels ist es, einen Überblick des Solar Butterflys zu verschaffen. Die funktionalität
Ziel: Überblick vermitteln. Funktionalität veranschaulichen, Begriffe Definieren.

Chassis, Hauptkörper, Seitenteil, Küche, Bad, Stützen, Panelen Gross, Panelen Klein

Verbindung von Dach zu den Raumelementen ref Kap. Dach - mittleres Raumelement
Europa Gewicht von 2200 kg
Rest: Gewicht von  3000 kg

\subsection{Projektorganisation}
Vier studenten, entsprechend viel Kommunikation.
All zwei wochen ein Meeting vor Ort.
Sonst über zooom.
Besuch bei Geser-Fahrzeugbau.
Wöchentliche Meetings mit Advisor

Risiko des Gewichtes, was immer wieder vorkommen wird. Nicht der Fokus, beeinfluss jedoch die Entscheidungen.


\subsection{Zum vorliegenden Dokument}
Elektronischer Anhang
what?

\newpage
