\section{Anforderungen und Methodik}
In diesem Kapitel wird beschrieben, welchen Anforderungen der Solar Butterfly und dessen Komponenten gerecht werden müssen. In einem ersten Schritt werden auf die allgemeinen Anforderungen des Solar Butterflys als Ganzes, und anschliessen auf die daraus folgenden Auslegungskriterien der einzelnen Komponenten eingegangen. Es wird beschrieben, was die Anforderungen konkret für die einzelnen Komponenten bedeuten und wie gewährleistet wird, dass diese erfüllt werden.\\
Im rahmen dieser Arbeit wird lediglich auf diejenigen Anforderungen eingegangen, welche für die strukturelle Auslegung und Festigketisberechnungen relevant sind. Die komplette Liste der Anforderungen an den Solar Butterfly ist in der Arbeit von [HUBER] zu finden.

\subsection{Anforderungen an den Solar Butterfly}
- Der Solar Butterfly muss den Angreifenden Kräften und herrschenden Lastfällen standhalten. (Vgl. Lastenheft [KAPITEL]) Konkret bedeutet dies, dass die Struktur sich bei den verschiedenen Lastfällen, und Kombinationen davon, nicht plastisch vervormen darf und eine genügend grosse Sicherheit gegen Fliessen aufweisen muss.\\
- Weiter darf der Solar Butterfly sich nicht so stark verformen, dass seine Funktionstauglichkeit eingeschränkt wird. Die exakten Anforderungen an die Steifigkeit werden bei der Abhandlnung der einzelnen Komponenten genauer beschrieben.\\
- Die Struktur des Solar Butterflys soll so ausgelegt werden, dass dieser ca. 200'000 km Fahrt auf zum teil recht holperiger Strasse auf sich nehmen kann. Dies beinhaltet die Auslegung der Komponenten auf Dauerfestigkeit. Es wird die vereinfachte Annahme getroffen, dass die Struktur für 50\% der Maximallast dauerfest ausgelegt werden soll.

\subsection{Auslegungskriterien}
Nachdem die allgemeinen Kriterien für den Solar Butterfly abgehandeln wurden, wird in diesem Unterkapitel behandelt, was die Anforderungen konkret für die einzelnen Komponenten und Strukturelementen bedeutet. Es wird beschrieben mit welchen Methoden die Auslegung angegangen wird und welche Vereinfachungen gemacht werden.\\
Verwendete Komponenten erwähnen? Sandwich, Profile usw...?

  \subsubsection{Aluminiumstrukturen}
    - Verschmierte Materialwerte
    - Max Spannungen, dehnung usw.
    - Knickung
    - Designallowables: 100MPa usw.

  \subsubsection{Sandwichstrukturen}
  Auslegungskriterien der Sandwichstrukturen können gemäss \cite{ETH} in die beiden Kategoriene \emph{Festigkeitsprobleme} und \emph{Stabilitätsprobleme} eingeteilt werden. Zu den Festigkeitsproblemen gehören;
  \begin{itemize}
    \item Brechung der Deckschicht,
    \item Schubbruch der Kernschicht,
    \item Delamination und
    \item Ermüdung.
  \end{itemize}

  Zu den Stabilitätsproblemen gehören;
  \begin{itemize}
    \item Knickung,
    \item Schubbeulung der Kernschicht (Shear Crimping) und
    \item Kurzwelliges Beulen der Deckschicht (Wrinkling).
  \end{itemize}

  Die auszulegenden Sandwichstrukturen werden gegenüber diesen Problemen überprüft. Um den Rechenaufwand zu verringern werden Annahmen und Vereinfachungen getroffen.

    \paragraph{Annahmen und Vereinfachungen}
    Für die Auslegung von Sandwichstrukturen können gemäss \cite{ETH} und \cite{klein} folgende Annahmen getroffen werden;
    \begin{itemize}
      \item linear elastische und isentrope Materialverhalten,
      \item Eigenbiegesteifigkeiten der Deckschichten sind vernachlässigbar,
      \item Dehnsteifigkeit der Kernschicht ist vernachlässigbar und
      \item die Kernschicht lässt sich nicht zusammendrücken.
    \end{itemize}

    Aus den getroffenen Annahmen reulstiert ein vereinfachter Spannungszustand welcher besagt, dass die Deckschichten jeweils die Normalkräfte und die Kernschichten die Schubkräfte aufnehmen. (Sandwichmembrantheorie)

    \paragraph{Festigkeitsprobleme}
    Aus den getroffenen Annahmen und Vereinfachungen lassen sich die Formeln \ref{Spannung in Deckschicht} und \ref{Schubspannungen im Kern} herleiten. Mit der Formel \ref{Spannung in Deckschicht} lassen sich die Spannungen in den Deckschichten berechnen. Mit ihr kann überprüft werden, ob die Deckschicht gegen Bruch abgesichert ist.

    \begin{equation}
      \label{Spannung in Deckschicht}
      \sigma = \frac{1}{t_d}\cdot \left ( \frac{n}{2} \pm \frac{m}{h}\right )
    \end{equation}

    Mit der Formel \ref{Schubspannungen im Kern} lassen sich die Schubspannungen in der Kernschicht berechnen und somit Aussagen über ihre Resistenz gegenüber dem Schubbruch machen.
    \begin{equation}
      \label{Schubspannungen im Kern}
      \tau_c = \frac{q}{t_c}
    \end{equation}

    Die Delamination der Deckschichten wird abgedeckt, indem die Auswahl des Klebers so getroffen wird, dass dieser eine höhere Resistenz gegenüber Beanspruchung auf Schub aufweist als die Kernschicht.

    Ermüdung?\\

    Verformung?\\

    \paragraph{Stabilitätsprobleme}
    Die Stabilitätsprobleme der Sandwichstrukturen lassen sich in globale und lokale Instabilitäten einteilen. Zur globalen Instabilität gehört das Knicken, welches sich aus der Eueler-Knickung des schubsteifen Balkens und dem Schubknicken zusammensetzt.
    Die kritische Belastung, bei welcher es zur Euler-Knickung kommt, lässt sich gemäss \cite{ETH} mit der Formel \ref{Euler-Knicklast} berechnen, wobei \emph{b} für die Breite der Sandwichstruktur steht.

    \begin{equation}
      \label{Euler-Knicklast}
      P_{kB}=\frac{\pi^2 \cdot E_d \cdot I_D}{l_k^2}
    \end{equation}

    wobei
    \begin{equation}
      \label{JD}
      I_D = 2\cdot b \cdot t_d\cdot \left ( \frac{t_k}{2}+t_d \right )^2
    \end{equation}

    Die kritische Schubknicklast lässt sich gemäss der Formel \ref{Schubknicklast} berechnen.
    \begin{equation}
      \label{Schubknicklast}
      P_{kS} = b \cdot c \cdot G_k
    \end{equation}

    Die totale kritische Knicklast \(P_k\) ergibt sich dann aus der Formel \ref{Knicklast}:
    \begin{equation}
      \label{Knicklast}
      P_k=\frac{1}{\frac{1}{P_{kB}}+\frac{1}{P_{kS}}}
    \end{equation}

    Zu den lokalen Instabilitäten zählen das Schubbeulen und das Knittern der Deckschicht. Die kritischen Spannunge, bei welcher Schubbeulung auftritt, lässt sich gemäss \cite{ETH} aus den Formel \ref{Schubbeulen} berechnen.
    \begin{equation}
      \label{Schubbeulen}
      \sigma_k^* = G_k \cdot \frac{h^2}{2 \cdot t_k \cdot t_d}
    \end{equation}

    Die kritischen Spannunge, bei welcher das Knittern der Deckschicht auftritt, lässt sich gemäss \cite{ETH} mit der Formel \ref{Knittern} berechnen.
    \begin{equation}
      \label{Knittern}
      \sigma_k = k_s\sqrt[3]{E_d \cdot E_k \cdot G_k}
    \end{equation}

    Wobei \(k_s\) für Auslegungen = 0.5 ist.


  \subsubsection{Verbindungen}
    - Nieten\\
    - Schrauben\\
    - Klebeverbindungen\\

  \subsubsection{Dauerfestigkeit}
  Auslegung: Spannungskonzentrationen vermeiden
  50\% der Maximallast Dauerfest
