\section{Anforderungen und Auslegung}
In diesem Kapitel wird beschrieben, welchen Anforderungen der Solar Butterfly und dessen Komponenten gerecht werden müssen. In einem ersten Schritt werden auf die allgemeinen Anforderungen des Solar Butterflys und anschliessen auf die daraus folgenden Auslegungskriterien der einzelnen Komponenten eingegangen. Es wird beschrieben, was die Anforderungen konkret für die einzelnen Komponenten bedeuten und wie gewährleistet wird, dass diese erfüllt werden.\\
Im rahmen dieser Arbeit wird lediglich auf diejenigen Anforderungen eingegangen, welche für die strukturelle Auslegung und Festigketisberechnungen relevant sind. Die komplette Liste der Anforderungen an den Solar Butterfly ist in der Arbeit von [HUBER] zu finden.

\subsection{Anforderungen an den Solar Butterfly}
- Der Solar Butterfly muss den Angreifenden Kräften und herrschenden Lastfällen standhalten. (Vgl. Lastenheft [KAPITEL]) Konkret bedeutet dies, dass die Struktur sich bei den verschiedenen Lastfällen, und Kombinationen davon, nicht plastisch vervormen dürfen und somit eine genügend grosse Sicherheit gegen Fliessen aufweisen muss.\\
- Weiter darf der Solar Butterfly sich nicht so stark verformen, dass seine Funktionstauglichkeit eingeschränkt wird. Die exakten Anforderungen an die Steifigkeit werden bei der Abhandlnung der einzelnen Komponenten genauer beschrieben.\\
- Die Struktur des Solar Butterflys soll so ausgelegt werden, dass dieser ca. 180'000 km Fahrt auf zum teil recht holperiger Strasse auf sich nehmen kann. Dies beinhaltet die Auslegung der Komponenten auf Dauerfestigkeit.

\subsection{Auslegungskriterien}
Nachdem die allgemeinen Kriterien für den Solar Butterfly abgehandeln wurden, wird in diesem Unterkapitel behandelt, was die Anforderungen konkret für die einzelnen Komponenten und Strukturelementen bedeutet. Es wird beschrieben mit welchen Methoden die Auslegung angegangen wird und welche Vereinfachungen getroffen werden.\\

\paragraph{Dauerfestigkeit}
Die Dauerfestigkeit des Solar Butteflys wird erreicht, indem durch entsprechendes Design, Spannungskonzentrationen vermieden werden. Weiter wird der Solar Butterfly so ausgelegt, dass dieser bei 50\% der Lasten aus dem Lastenheft, mit einem R von 0, dauerfest ist. Bei den Sandwichstrukturen wird dabei lediglich die Deckschicht auf die Dauerfestigkeit ausgelegt.

\paragraph{Sicherheitsfaktor}
SF = 1.5

  \subsubsection{Aluminiumstrukturen}
  Zu den Auslegungskriterien der Aluminiumstrukturen gehört das Festigkeitsproblem der plastschen Verformung (Fliessen) und das Stabilitätsproblem der Knickung. Die Aluminiumstrukturen werden so ausgelegt, dass diese genügend Sicherheiten gegen Fliessen und Knickung aufweisen.

  \paragraph{Sicherheit gegen Fliessen}
  Um die Sicherheit eines Strukturelementes gegen Fliessen zu gewährleisten, wird überprüft, ob die \emph{Von Mises}-Vergleichsspannung kleiner als die zulässige Spannung ist, wobei sich die zulässige Spannung aus der Dehngrenze des gewählten Materials und einem Sicherheitsfaktor zusammensetzt. Die \emph{Von Mises}-Vergleichsspannung kann gemäss der Formel \ref{Von Mises} berechet werden \cite{Baertsch}.
  \begin{equation}
    \label{Von Mises}
    \sigma_{zul} \geq \sigma_v = \sqrt{\sigma_x^{2}-\sigma_x \cdot \sigma_y + \sigma_y^2 + 3\tau^2}
  \end{equation}
  Wobei die Annahmen getroffen werden, dass es sich um einen ebenen Spannungszustand handelt und die angeifenden Lasten dem selben Lastfall angehören.

  \paragraph{Knicken}
  Wird durch design verhindert (?)

  \subsubsection{Sandwichstrukturen}
  Versagenskriterien der Sandwichstrukturen können in die beiden Kategorien \emph{Festigkeitsprobleme} und \emph{Stabilitätsprobleme} eingeteilt werden \cite{ETH}. Zu den Festigkeitsproblemen gehören;
  \begin{itemize}
    \item Fliessen der Deckschicht,
    \item Schubbruch der Kernschicht,
    \item Delamination und
    \item Ermüdung.
  \end{itemize}

  Zu den Stabilitätsproblemen gehören unteranderem;
  \begin{itemize}
    \item Knickung,
    \item Schubbeulung der Kernschicht (Shear Crimping) und
    \item Kurzwelliges Beulen der Deckschicht (Wrinkling).
  \end{itemize}

  Die auszulegenden Sandwichstrukturen werden gegenüber diesen Festigkeits und Stabilitätsproblemen abesichert. Um den Rechenaufwand und die Komplexität zu verringern werden Annahmen und Vereinfachungen getroffen.
  Für die Auslegung von Sandwichstrukturen können folgende Annahmen getroffen werden \cite{ETH}\cite{klein};
  \begin{itemize}
    \item linear elastische und isentrope Materialverhalten,
    \item Eigenbiegesteifigkeiten der Deckschichten sind vernachlässigbar,
    \item Dehnsteifigkeit der Kernschicht ist vernachlässigbar und
    \item die Kernschicht lässt sich nicht zusammendrücken.
  \end{itemize}
  Aus den getroffenen Annahmen reulstiert ein vereinfachter Spannungszustand welcher besagt, dass die Deckschichten jeweils die Normalkräfte und die Kernschichten die Schubkräfte aufnehmen. (Sandwichmembrantheorie)

    \paragraph{Festigkeitsprobleme}
    Aus den getroffenen Annahmen und Vereinfachungen lassen sich die Formeln \ref{Spannung in Deckschicht} und \ref{Schubspannungen im Kern} herleiten. Mit der Formel \ref{Spannung in Deckschicht} lassen sich die Spannungen in den Deckschichten berechnen. Die Dicke der Deckschicht wird so gewählt, dass die zulässige Spannung höher liegt als jene, welche in der Deckschicht herrscht.

    \begin{equation}
      \label{Spannung in Deckschicht}
      \sigma_{zul} \geq \sigma_d = \frac{1}{t_d}\cdot \left ( \frac{n}{2} \pm \frac{m}{h}\right )
    \end{equation}

    Mit der Formel \ref{Schubspannungen im Kern} lassen sich die Schubspannungen in der Kernschicht berechnen und somit Aussagen über ihre Resistenz gegenüber dem Schubbruch machen. Die Dicke der Kernschicht wird so ausgelegt, dass die in der Kernschicht herrscheden Spannungen tiefer liegen als die zulässigen.
    \begin{equation}
      \label{Schubspannungen im Kern}
      \tau_{k,zul} \geq \tau_k = \frac{q}{t_k}
    \end{equation}

    Die Delamination der Deckschichten wird abgesichert, indem die Auswahl des Klebers so getroffen wird, dass dieser eine höhere Schubfestigkeit aufweist als das Material der jeweiligen Kernschicht.

    \paragraph{Stabilitätsprobleme}
    Die Stabilitätsprobleme der Sandwichstrukturen lassen sich in globale und lokale Instabilitäten einteilen. Zur globalen Instabilität gehört das Knicken, welches sich aus der Eueler-Knickung des schubsteifen Balkens und dem Schubknicken zusammensetzt. Die kritische Belastung, bei welcher es zur Euler-Knickung kommt, lässt sich gemäss Klein \cite{klein} mit der Formel \ref{Euler-Knicklast} berechnen.

    \begin{equation}
      \label{Euler-Knicklast}
      F_{kB}=\frac{\pi^2 \cdot E_d \cdot I_y}{l_k^{2}}
    \end{equation}

    Wobei sich die Biegesteifigkeit $I_y$ vereinfacht gemäss der Formel \ref{ID} berechnen lässt. Hier wurde die Annahme getroffen, dass die Eigenbiegesteifigkeiten der Deckschichten vernachlässigbar sind. Diese Annahme kann gemäss Klein \cite{klein} ab einem Verhältnis von $t_d$ zu $t_k$ von 0.25, getroffen werden.
    \begin{equation}
      \label{ID}
      I_y = 2 \cdot b \cdot t_d \cdot \left( \frac{t_k}{2} + t_d \right )^{2}
      % B_y = 2\cdot b \cdot t_d\cdot \left ( \frac{t_k}{2}+t_d \right )^2
    \end{equation}

    Die kritische Schubknicklast lässt sich gemäss Klein \cite{klein} mit der Formel \ref{Schubknicklast} berechnen.
    \begin{equation}
      \label{Schubknicklast}
      F_{kS} = b \cdot t_k \cdot G_k
    \end{equation}

    Die totale kritische Knicklast \(F_k\) ergibt sich dann aus der Formel \ref{Knicklast}:
    \begin{equation}
      \label{Knicklast}
      F_{k, vorh.} \leq F_k=\frac{1}{\frac{1}{F_{kB}}+\frac{1}{F_{kS}}}
    \end{equation}

    Zu den lokalen Instabilitäten zählen das Schubbeulen und das Knittern der Deckschicht. Die kritischen Spannunge, bei welcher Schubbeulung auftritt, lässt sich gemäss \cite{ETH} aus den Formel \ref{Schubbeulen} berechnen.
    \begin{equation}
      \label{Schubbeulen}
      \sigma_k = G_k \cdot \frac{h}{2 \cdot t_d}
    \end{equation}

    Die kritischen Spannunge, bei welcher das Knittern der Deckschicht auftritt, lässt sich nach \cite{ETH} mit der Formel \ref{Knittern} berechnen.
    \begin{equation}
      \label{Knittern}
      \sigma_k = k_s\sqrt[3]{E_d \cdot E_k \cdot G_k}
    \end{equation}
    Wobei für Auslegungen \(k_s = 0.5\) gilt.

    % \paragraph{Verwendete Materialien und zulässige Werte}
    % Bsp.\\
    % Bodenplatte aus T92.9 mit 1 mm Alublech:
    % \begin{itemize}
    %   \item $\tau_{zul} = 10 MPa$
    %   \item $\sigma_{zul} = 100 MPa$
    %   \item $Dauerfestigkeit = 42.0 MPa$
    %   \item $\sigma_{Knittern} = 200 MPa$
    % \end{itemize}
    % Wand aus T92.6 mit 0.5 mm Alublech:
    % \begin{itemize}
    %   \item $\tau_{zul} = 5 MPa$
    %   \item $\sigma_{zul} = 100 MPa$
    %   \item $Dauerfestigkeit = 42.0 MPa$
    %   \item $\sigma_{Knittern} = 150 MPa$
    % \end{itemize}

  \subsubsection{Nieten}
    % \paragraph{Tragfähigkeitsnachweis}
    Laut Klein \cite{klein} gehört zum Tragfähigkeitsnachweis für gewöhnlich ein Abscher- und Lochleibungsnachweis. Insofern sei für Nietverbindungen ein Nachweis auf Scherbruch (Formel \ref{Scherbruch}) und Lochleibung (Formel \ref{Lochleibung}) zu erbringen:
    \begin{multicols}{2}
      \begin{equation}
        \label{Scherbruch}
        F \leq F_{SB} = \frac{d_N^2 \cdot \pi}{4}\cdot \tau_B
      \end{equation}\break
      \begin{equation}
        \label{Lochleibung}
        F \leq F_{LF} = d_N \cdot t \cdot \sigma_{FL}
      \end{equation}
    \end{multicols}
    Wobei $d_N$ der Nietlochdurchmesser, $\tau_B$ die Scherfestigkeit, $t$ die Blechdicke und $\sigma_{FL}$ die Lochleibungs-Dehngrenze ist. Für dynamische Wechselfestigkeitswerte sei die Scherfestigkeit $\tau_B$ noch um den Faktor 2 bis 2.2 zu verringern.

    \paragraph{Überlagerte Scher- und Zugbeanspruchung}
    In der Praxis werden Nietverbindungen aus einer Kombination von Scher- und Zugbeanspruchung beansprucht. Der Nachweis der Tragfähigkeit der überlagerten Belastung wird durch die Ausweisung des Reservefaktors $R_f$ bewerkstelligt. Dazu werden gemäss den Formeln \ref{Rs} und \ref{Rz} der Schubreservefaktor $R_s$ und der Zugreservefaktor $R_z$ berechnet

    \begin{multicols}{2}
      \begin{equation}
        \label{Rs}
        F \leq F_{SB} = \frac{d_N^2 \cdot \pi}{4}\cdot \tau_B
      \end{equation}\break
      \begin{equation}
        \label{Rz}
        R_z = \frac{F_z}{k \cdot F_{ZB}}
      \end{equation}
    \end{multicols}

    \subsubsection{Klebeverbindungen}
    \begin{equation}
      \label{Kleben}
      \tau_K = \frac{F}{b \cdot l_{\ddot{u}}} \leq \frac{\tau_{KB}}{S}
    \end{equation}

    \begin{equation}
      \label{Zulässige Schubspannungen}
      \begin{split}
      wechselnd: & \: \tau_{KW} \approx \left (0.2 ... 0.4 \right ) \cdot \tau_{KB}\\
      schwellend: & \: \tau_{KSch} \approx 0.8 \cdot \tau_{KB}
      \end{split}
    \end{equation}

    \subsubsection{Schrauben}

    \subsubsection{Spannpratzen}


  \subsection{Verformung}
\newpage
