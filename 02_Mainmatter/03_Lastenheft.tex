\section{Lasten}
Begründung und Erklärung der Lasten
Aufbau des Lastenheftes A B1, B2 und C, mit verschiedenen Konfigurationen

\subsection{Belastungen beim Fahren}
Zustand A:


\paragraph{Beschleunigungen}
Beschleunigungen werden Absolut angegeben. 0g heisst schwerelosigkeit. wird in G angegeben wobei damit $9.81 \left[\frac{m}{s^2}\right]$ gemeint ist.\\
Die folgenden Beschleunigungen werden immer in Kombination mit der Erdbeschleunigung von 1g kombiniert.


\begin{description}
  % \item \textbf{1.1 Erdbeschleunigung}\\
  % Standardmässig ist der Solar Butterfly der Erdbeschleunigung von 1g ausgesetzt. \\

  \item \textbf{1.2 Vertikale Beschleunigung}\\
  Zusätzlich zur vertikalen Beschleunigung durch die Erdanziehung, entstehen durch das Überfahren von Schlaglöcher und Bremsschwellen vertikale Beschleunigungen.\\
  In einem ersten Ansatz wurde der Solar Butterflys als ein \emph{Ein-Massen-Schwinger}-System modelliert und die Beschleunigung beim Überfahren einer Sinusförmigen Bremsschwelle numerisch ermittelt.\\

  Die Position des Rades während dem Überfahren der Bremsschwelle ist gegeben durch folgenden Zusammenhang:
  \begin{equation}
    x_r^n = h \cdot \sin(\pi \cdot \frac{n\Delta t \cdot v}{l})
  \end{equation}
  $l$ steht dabei für die Länge, und h für die Höhe der Bremsschwelle.\\

  Um die Beschleunigung des Solar Butterflys zu berechnen, wird in einem ersten Schritt dessen Position zum Zeitpunk $n$ $x_{SB}^n$ aus der vorangehenden Situation berechnet.
  \begin{equation}
    x_{SB}^n = x_{SB}^{(n-1)} + v^{(n-1)} \cdot \Delta t
  \end{equation}

  Als nächstes wird der Federweg $s^n$, sowie die Änderungsrate des Federwegs $v_s^n$ zum Zeitpunkt $n$ berechnet.
  \begin{equation}
    s^n = x_r^n - x_{SB}^n
  \end{equation}
  \begin{equation}
    v_s^n = \frac{s^n - s^{(n-1)}}{\Delta t}
  \end{equation}

  Die Beschleunigung des Solar Butterfly ergibt sich dann zu:\\
  \begin{equation}
    a_{SB}^n = \frac{k \cdot s^n + d \cdot v_s^n}{m}
  \end{equation}

  womit die daraus resultierende neue Geschwindigkeit des Chassis berechnet werden kann.
  \begin{equation}
    v^n = v^{(n-1)} + a^n \cdot \Delta t
  \end{equation}

  Das \emph{Ein-Massen-Schwinger}-Modell wurde mit einer Masse von 2200 kg, einer Federkonstante, gegeben aus den Datenblättern des Herstellers [ANHANG], von 353'000 N/m und einer Dämpfungskonstante von 3500 Ns/m modelliert. Beim Überfahren einer Bremsschwelle von 0.9 m Länge und 0.1 m Höhe mit einer Geschwindigkeit von 40 km/h resultierte eine maximale Beschleunigung von rund 1.6 g. Die Berechnung ist im [Elektronischen Anhang] zu finden.\\
  Zu der Berechnung muss gesagt werden, dass es sich um ein eher konservatives Modell handelt und die erhaltene Beschleunigung zu hoch liegt. So wurde zum Beispiel die Federung durch die Reifen nicht berücksichtigt. Weiter befindet sich der Massenschwerpunkt nicht in der Federachse, was eine weitere Abminderung der Beschleunigung zur folge hat.\\



  \cite{Beschl.1}: PKW; 0.36x0.05 Speed bump; 0.71 g (Fahrzeug Mitte) und 1.5 g (Über der Achse) bei 40 km/h\\
  \cite{Beschl.2}: PKW; 0.90x0.10 Speed bump; 0.73 g (Fahrzeug Mitte) bei 50 km/h\\
  \cite{Beschl.3}: PKW; 0.50x0.05 Speed bump; 1 g (Person hinten im Fahrzeug) und 1.3 g (Über der Achse) bei 30 km/h\\
  \cite{Beschl.4}: LKW; +-1g Beschleunigung für Transportware in einem Sattelschlepper.\\

  Schluss: 1.25g vertikale Beschleunigung

  \item \textbf{1.3 Longitudinale Beschleunigung}\\
  Longitudinale Beschleunigungen entstehen durch erhöhung der Geschwindigkeit durch das Zugfahrzeug und durch Auffahrunfälle von Dritten. Das \emph{Institut für Unfallanalysen Hamburg} benützt die Beschleunigung von Personenwagen von maximal 0.3g und von Lastkraftwagen von 0.1 g als Anhaltswerte. \cite{Verz.3}
  Schluss: 0.2 g

  \item \textbf{1.4 Longitudinale Verzögerung}\\
  Longitudinale Verögerungen werden durch graduelle abminderung der Geschwindigkeit und durch Auffahrunfälle des Zugfahrzeuges herbeigeführt. Die extremste graduelle Verzögerung entsteht dabei durch eine Notbremsung.\\
  \emph{Kudarauskas} zeigt bei seiner Analyse der Notbremsungen von Personenwagen, dass die maximale Verzögerung bei rund 0.9 g liegt. \cite{Verz.1} Das \emph{Institut für Unfallanalysen Hamburg} zeiht bei Gutachten die Vollverzögerung von 0.8 g für Personenwagen und 0.7 g für Lastkraftwagen als standard Werte herbei. \cite{Verz.2}\\

  Schluss: 0.8 g

  \item \textbf{1.5 Laterale Beschleunigung}\\
  Laterale Beschleunigungen entstehen hauptsächlich beim Kurvenfahren und sind abhängig von der Geschwindigkeit mit welcher die Kurve durchfahren wird und des Kurvenradius.\\
  \emph{Hugemann} et al. massen in einem Personenwagen auf einer Landstrasse Laterale Beschleunigungen von 0.6 g .\cite{Kurv.1} \emph{Xu} et al. zeigen, dass die Mehrheit der gemessenen Beschleunigung durch Kurvenfahrten in bergigem Gebiet über 0.5 g und maximale über 0.8 g liegen. \cite{Kurv.2}

  Schluss: +-0.8 g\\
\end{description}

\paragraph{Belastung durch Wind}

\cite{Wind.1} Laterale Windgeschwindigkeiten von 108 km/h seien kritisch für Fahrzeuge auf trokener Strasse.\\
\cite{Wind.2} Bei Windgeschwindigkeiten von mehr als 80 km/h wird empfohlen nicht mehr zu Fahren. Windgeschwindigkeiten von 95 km/h sei genug um Wohnmobile umzustossen\\
\cite{Wind.3} Bei Windgeschwindigkeiten von mehr als 155 km/h können "high profile Trucks, Trailers and Busse" überkippen. Minimale OVERTURNING WIND SPEEDS von 108 km/h für 9 m langes Motor home und 160 km/h für ein 5m langes Wohnmobil.

\begin{description}
  \item \textbf{2.1 Wind}\\
  Schluss: bei 80km/h soll nicht mehr gefahren werden: Winddruck bei 100km/h\\
  Definitive Windgeschw. zum kippen kann schlussendlich mit dem FEM ermittelt werden. Dies soll nur eine erste Annahme sein.
  \begin{equation}
    \label{Winddruck}
    W_D = c_p \: \frac{\rho}{2}\: v^2
  \end{equation}
  $\rho = 1.2 \frac{kg}{m^3}$\\
  $c_p \: Rechteck = 1.05$\\

  $W_D \:@\: 100 km/h= 486 \left[MPa \right]$
\end{description}

\paragraph{Belastung durch Neigung}
Absprache mit Palmer: Furkapass max. 6.3° -> auslegung auf 10° (17.5\%)
Neigung in beide Richtungen
\begin{description}
  \item \textbf{3.1 Geneigte Strasse Länges} +-10° Neigung in Fahrtrichtung.
  \item \textbf{3.2 Geneigte Strasse Quer} +-10° Neigung normal zur Fahrtrichtung.

\end{description}

\paragraph{Mobiliar}
50kg Mobiliar im Fahrzeug verteilt
\begin{description}
  \item \textbf{4.1 Mobiliar vorne}
  \item \textbf{4.2 Mobiliar hinten}
\end{description}
