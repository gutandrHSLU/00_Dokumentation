\section{Lasten}
Begründung und Erklärung der Lasten
Aufbau des Lastenheftes A B1, B2 und C, mit verschiedenen Konfigurationen

\subsection{Belastungen beim Fahren}
Zustand A:


\paragraph{Beschleunigungen}\vfill
Beschleunigungen werden Absolut angegeben. 0g heisst schwerelosigkeit. wird in G angegeben wobei damit $9.81 \left[\frac{m}{s^2}\right]$ gemeint ist.\\

Die folgenden Beschleunigungen werden immer in Kombination mit der Erdbeschleunigung von 1g kombiniert.


\begin{description}
  % \item \textbf{1.1 Erdbeschleunigung}\\
  % Standardmässig ist der Solar Butterfly der Erdbeschleunigung von 1g ausgesetzt. \\

  \item \textbf{1.2 Vertikale Beschleunigung}\\
  Zusätzlich zur vertikalen Beschleunigung durch die Erdanziehung, entstehen durch das Überfahren von Schlaglöcher und Bremsschwellen vertikale Beschleunigungen.\\
  In einem ersten Ansatz wurde der Solar Butterflys als ein \emph{Ein-Massen-Schwinger}-System modelliert und die Beschleunigung beim Überfahren einer Sinusförmigen Bremsschwelle numerisch berechnet.\\

  Die Position des Rades während dem Überfahren der Bremsschwelle ist gegeben durch folgenden Zusammenhang:
  \begin{equation}
    x_r^n = h \cdot \sin(\pi \cdot \frac{n\Delta t \cdot v}{l})
  \end{equation}
  $l$ steht dabei für die Länge, und h für die Höhe der Bremsschwelle. $n$ und $\Delta t$ werden so gewählt, dass\\

  Um die Beschleunigung des Solar Butterflys zu berechnen, wird in einem ersten Schritt dessen Position zum Zeitpunk $n$ $x_{SB}^n$ aus der vorangehenden Situation berechnet.
  \begin{equation}
    x_{SB}^n = x_{SB}^{(n-1)} + v^{(n-1)} \cdot \Delta t
  \end{equation}

  Als nächstes wird der Federweg $s^n$, sowie die Änderungsrate des Federwegs $v_s^n$ zum Zeitpunkt $n$ berechnet.
  \begin{equation}
    s^n = x_r^n - x_{SB}^n
  \end{equation}
  \begin{equation}
    v_s^n = \frac{s^n - s^{(n-1)}}{\Delta t}
  \end{equation}

  Die Beschleunigung des Solar Butterfly ergibt sich dann zu:\\
  \begin{equation}
    a_{SB}^n = \frac{k \cdot s^n + d \cdot v_s^n}{m}
  \end{equation}

  womit die daraus resultierende neue Geschwindigkeit des Chassis berechnet werden kann.
  \begin{equation}
    v^n = v^{(n-1)} + a^n \cdot \Delta t
  \end{equation}

  Die Berechnungen wurden bis zum ersten Null-Durchgang gemacht. Die Anzahlberechnungspunkte und $\Delta t$ werden entsprechend gewählt.

  Bei einer Geschwindigkeit von 40 km/h\\
  k = ~ 177'000 N/m\\
  d = 2000 Ns/m\\
  l = 0.9 m\\
  h = 0.1 m\\
  Resultat: 1.6g\\
  Konservatives Modell: Federung durch Reifen fehlt, weiter ist der Massenschwerpunkt aus der Feder-achse Verschoben, was eine weitere abminderung der Beschl. zur folge hat.

  \cite{Beschl.1}, \cite{Beschl.2} und \cite{Beschl.3} zeigen Beschleunigung von PKW's zwischen 0.8 und 1.3 g Beschleunigung. (Des Massenschwerpunktes)

  \cite{Beschl.4} zeigt +-1g Beschleunigung für Transportware in einem Sattelschlepper.

  Schluss: 1.25g vertikale Beschleunigung



  \item \textbf{1.3 Longitudinale Beschleunigung}\\
  - Beschleunigung durch Notbremse\\
  \cite{Verz.1} max. 9.5 m/s^2\\
  https://copradar.com/chapts/references/acceleration.html
  - Beschleunigung druch Unfall\\
  \item \textbf{1.4 Laterale Beschleunigung}\\
  Beschleunigung durch Kurven
\end{description}

\paragraph{Belastung durch Wind}
\begin{description}
  \item \textbf{2.1 Wind}
\end{description}

\paragraph{Belastung durch Neigung}
\begin{description}
  \item \textbf{3.1 Geneigte Strasse}
\end{description}

\paragraph{Mobiliar}
\begin{description}
  \item \textbf{4.1 Mobiliar vorne}
  \item \textbf{4.2 Mobiliar hinten}
\end{description}
