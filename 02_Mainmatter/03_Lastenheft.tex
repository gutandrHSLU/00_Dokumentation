\section{Lasten}
Begründung und Erklärung der Lasten
Aufbau des Lastenheftes A B1, B2 und C, mit verschiedenen Konfigurationen

\subsection{Belastungen beim Fahren}
Zustand A:


\paragraph{Beschleunigungen}
Beschleunigungen werden Absolut angegeben. 0g heisst schwerelosigkeit. wird in G angegeben wobei damit $9.81 \left[\frac{m}{s^2}\right]$ gemeint ist.\\
Die folgenden Beschleunigungen werden immer in Kombination mit der Erdbeschleunigung von 1g kombiniert.


\begin{description}
  % \item \textbf{1.1 Erdbeschleunigung}\\
  % Standardmässig ist der Solar Butterfly der Erdbeschleunigung von 1g ausgesetzt. \\

  \item \textbf{1.2 Vertikale Beschleunigung}\\
  Zusätzlich zur vertikalen Beschleunigung durch die Erdanziehung, entstehen durch das Überfahren von Schlaglöcher und Bremsschwellen vertikale Beschleunigungen.\\
  In einem ersten Ansatz wurde der Solar Butterflys als ein \emph{Ein-Massen-Schwinger}-System modelliert und die Beschleunigung beim Überfahren einer Sinusförmigen Bremsschwelle numerisch berechnet.\\

  Die Position des Rades während dem Überfahren der Bremsschwelle ist gegeben durch folgenden Zusammenhang:
  \begin{equation}
    x_r^n = h \cdot \sin(\pi \cdot \frac{n\Delta t \cdot v}{l})
  \end{equation}
  $l$ steht dabei für die Länge, und h für die Höhe der Bremsschwelle.\\

  Um die Beschleunigung des Solar Butterflys zu berechnen, wird in einem ersten Schritt dessen Position zum Zeitpunk $n$ $x_{SB}^n$ aus der vorangehenden Situation berechnet.
  \begin{equation}
    x_{SB}^n = x_{SB}^{(n-1)} + v^{(n-1)} \cdot \Delta t
  \end{equation}

  Als nächstes wird der Federweg $s^n$, sowie die Änderungsrate des Federwegs $v_s^n$ zum Zeitpunkt $n$ berechnet.
  \begin{equation}
    s^n = x_r^n - x_{SB}^n
  \end{equation}
  \begin{equation}
    v_s^n = \frac{s^n - s^{(n-1)}}{\Delta t}
  \end{equation}

  Die Beschleunigung des Solar Butterfly ergibt sich dann zu:\\
  \begin{equation}
    a_{SB}^n = \frac{k \cdot s^n + d \cdot v_s^n}{m}
  \end{equation}

  womit die daraus resultierende neue Geschwindigkeit des Chassis berechnet werden kann.
  \begin{equation}
    v^n = v^{(n-1)} + a^n \cdot \Delta t
  \end{equation}

  Die Berechnungen wurden bis zum ersten Null-Durchgang gemacht. Die Anzahlberechnungspunkte und $\Delta t$ werden entsprechend gewählt.

  Bei einer Geschwindigkeit von 40 km/h\\
  k = ~ 177'000 N/m\\
  d = 2000 Ns/m\\
  l = 0.9 m\\
  h = 0.1 m\\
  Resultat: 1.6g\\
  Konservatives Modell: Federung durch Reifen fehlt, weiter ist der Massenschwerpunkt aus der Feder-achse Verschoben, was eine weitere abminderung der Beschl. zur folge hat.

  \cite{Beschl.1}: PKW; 0.36x0.05 Speed bump; 0.71 g (Fahrzeug Mitte) und 1.5 g (Über der Achse) bei 40 km/h\\
  \cite{Beschl.2}: PKW; 0.90x0.10 Speed bump; 0.73 g (Fahrzeug Mitte) bei 50 km/h\\
  \cite{Beschl.3}: PKW; 0.50x0.05 Speed bump; 1 g (Person hinten im Fahrzeug) und 1.3 g (Über der Achse) bei 30 km/h\\
  \cite{Beschl.4}: LKW; +-1g Beschleunigung für Transportware in einem Sattelschlepper.

  Schluss: 1.25g vertikale Beschleunigung


  \item \textbf{1.3 Longitudinale Beschleunigung}\\
  - Beschleunigung durch Notbremse\\
  \cite{Verz.1} Pkw max. über 0.9 g\\
  \cite{Verz.2} Pkw bis 0.8 g, Lkw bis 0.7 g\\

  - Beschleunigung druch Unfall\\
  Wird ausgeschlossen?\\

  Schluss: +-1 g

  \item \textbf{1.4 Laterale Beschleunigung}\\
  Beschleunigung durch Kurven\\
  \cite{Kurv.1} ca. 0.6 g auf Landstrasse\\
  \cite{Kurv.2} mehrheit in bergigem Gebiet über 0.5 und max. über 0.8 g\\

  Schluss: +-1 g\\
\end{description}

\paragraph{Belastung durch Wind}

\cite{Wind.1} Laterale Windgeschwindigkeiten von 108 km/h seien kritisch für Fahrzeuge auf trokener Strasse.\\
\cite{Wind.2} Bei Windgeschwindigkeiten von mehr als 80 km/h wird empfohlen nicht mehr zu Fahren. Windgeschwindigkeiten von 95 km/h sei genug um Wohnmobile umzustossen\\
\cite{Wind.3} Bei Windgeschwindigkeiten von mehr als 155 km/h können "high profile Trucks, Trailers and Busse" überkippen. Minimale OVERTURNING WIND SPEEDS von 108 km/h für 9 m langes Motor home und 160 km/h für ein 5m langes Wohnmobil.

\begin{description}
  \item \textbf{2.1 Wind}\\
  Schluss: bei 80km/h soll nicht mehr gefahren werden: Winddruck bei 100km/h\\
  Definitive Windgeschw. zum kippen kann schlussendlich mit dem FEM ermittelt werden. Dies soll nur eine erste Annahme sein.
  \begin{equation}
    \label{Winddruck}
    W_D = c_p \: \frac{\rho}{2}\: v^2
  \end{equation}
  $\rho = 1.2 \frac{kg}{m^3}$\\
  $c_p \: Rechteck = 1.05$\\

  $W_D \:@\: 100 km/h= 486 \left[MPa \right]$
\end{description}

\paragraph{Belastung durch Neigung}
Absprache mit Palmer: Furkapass max. 6.3° -> auslegung auf 10° (17.5\%)
Neigung in beide Richtungen
\begin{description}
  \item \textbf{3.1 Geneigte Strasse Länges} +-10° Neigung in Fahrtrichtung.
  \item \textbf{3.2 Geneigte Strasse Quer} +-10° Neigung normal zur Fahrtrichtung.

\end{description}

\paragraph{Mobiliar}
50kg Mobiliar im Fahrzeug verteilt
\begin{description}
  \item \textbf{4.1 Mobiliar vorne}
  \item \textbf{4.2 Mobiliar hinten}
\end{description}
