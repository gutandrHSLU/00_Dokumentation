
\section{Fazit}
\label{Fazit}

Vorgehen:
Steifigkeiten von Chassis ermitteln: Research oder Auswertung bereits verwendeter Chassis von Geser.
Steifigkeitskriterien bestimmen.
Was ist überhaupt möglich? zum einen Verbindungsarten und Chassis. Wandstärken, aussparungen usw.
Problematik: Wärmeausdehnung
Verbindungsmethoden zwischen Boden und Chassis. Befragung von Leuten mit erfahrungen. Kleben, nieten usw???

Lastenheft eventuell durch Feder-Dämpfer modell überprüfen.

ermüdung nochmals abchecken. Nachweis erbringen usw.

Klar definieren was "erlaub" ist und was nicht. Bei welchen Windgeschwindigkeiten muss zusammengeräumt werden? Wie müssen die Module ausgefahren werden? Wie handeln falls etwas nicht funktioniert? Potentiell gefährliche Situationen (durch Defekte) beschreiben. Wie müssen diese Situationen gehandhabt werden? z.B. was wenn ein Verschluss nicht schliessen kann?

Konzept zu wenig weit ausgearbeitet, dass diese s'Zenarien hätten bestimmt oder beschrieben werden können.


Für die Modellierung: Würde die FEM-Berechnung erneut durchgeführt werden, wird empfohlen, das Chassis mittels Schalenkörper zu modellieren, wodurch eine Kontakt-Verbindung (an Stelle einer Joint-Verbindung) verwendet werden könnte. Eine weitere möglichkeit währe, die Klebeverbindung mittels MPC-Kontakten (MPC184 Element) mit definierbarer Steifigkeit zu modellieren, was als eine exaktere Modellierung beurteilt wird.  (Aus Kap. Globales FEM)
