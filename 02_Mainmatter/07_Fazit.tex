
\section{Diskussion}
\label{Diskussion}

\textbf{Lastenheft}\\
Beim Durchführen der Handrechnungen und der Auswertung der FEM-Berechnungen hat sich gezeigt, dass die Lastfälle \emph{1.1 Vertikale Beschleunigung} und \emph{1.5 Rotatorische Beschleunigung} die stärksten Belastungen für den Solar Buttefly darstellen. Zugleich weisen diese beiden Lastfälle die grössten Risiken auf, da sie auf groben Annahmen beruhen und nur mit grösserem Aufwand exakter abgeschätzt werden können.\\
Bei der Wahl der Lastfälle wurde eine tendenziell konservative Position eingenommen, so dass die Beschleunigungen eher zu hoch als zu tief liegen. Folglich ist es wahrscheinlich, dass durch eine Überarbeitung des Lastenheftes und der genaueren Bestimmung der kritischen Beschleunigungen, tiefere Werte bestimmt werden können und der Solar Butterfly exakter für die auftretenden Belastungen ausgelegt werden kann.\\
Sofern dem Projekt genügen Ressourcen und Zeit zur Verfügung stehen und sich dafür entschieden wird das Lastenheft zu überarbeiten, wird empfohlen den Solar Butterfly als Feder-Dämpfer-Modell zu Modellieren und die Beschleunigungen für verschiede Bedingungen und Situationen zu ermitteln. Mit den gewonnen Erkenntnissen könnten die getroffenen Annahmen überprüft und der Solar Butterfly präziser Ausgelegt werden, was eine Gewichtsreduktion bringen könnte. 


Klar definieren was "erlaub" ist und was nicht. Bei welchen Windgeschwindigkeiten muss zusammengeräumt werden? Wie müssen die Module ausgefahren werden? Wie handeln falls etwas nicht funktioniert? Potentiell gefährliche Situationen (durch Defekte) beschreiben. Wie müssen diese Situationen gehandhabt werden? z.B. was wenn ein Verschluss nicht schliessen kann?

ermüdung nochmals abchecken. Nachweis erbringen usw.
Vorgehen:
Steifigkeiten von Chassis ermitteln: Research oder Auswertung bereits verwendeter Chassis von Geser.
Steifigkeitskriterien bestimmen.
Was ist überhaupt möglich? zum einen Verbindungsarten und Chassis. Wandstärken, aussparungen usw.
Problematik: Wärmeausdehnung in bezug auf Boden
Verbindungsmethoden zwischen Boden und Chassis. Befragung von Leuten mit erfahrungen. Kleben, nieten usw???






Konzept zu wenig weit ausgearbeitet, dass diese s'Zenarien hätten bestimmt oder beschrieben werden können.


\newpage

\section{Fazit}


\newpage

\section{Danksagung}
Ohne die Unterstützung folgender Personen wäre mir das Ausarbeiten der Bachelorarbeit in dieser Form nicht möglich gewesen. Dafür möchte ich an dieser Stelle meinen Dank aussprechen an:\\
Dejan Roman\v{c}uk, für die Betreuung und Unterstützung bei der Durchführung dieser Arbeit.\\
Damian Gwerder und David Schiffmann für das Durchführen und Auswerten der CT-Scans.\\
Simon Gerig und Marcel Furrer für das Herstellen der Proben.\\
Ruedi Pflugshaupt und sein Team für das mechanische Bearbeiten der Proben und Herstellen\\
der Probenhalterung\\
\newpage
