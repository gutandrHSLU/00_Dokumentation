\section{Diskussion}
\label{Diskussion}

\subsection{Lastenheft}
Beim Durchführen der Handrechnungen und der Auswertung der FEM-Berechnungen hat sich gezeigt, dass die Lastfälle \emph{1.1 Vertikale Beschleunigung} und \emph{1.5 Rotatorische Beschleunigung} die stärksten Belastungen für den Solar Buttefly darstellen. Zugleich weisen diese beiden Lastfälle die grössten Risiken auf, da sie auf groben Annahmen beruhen und nur mit grösserem Aufwand exakter abgeschätzt werden können.\\
Bei der Festlegung der Lasten im Lastenheft wurde eine tendenziell konservative Position eingenommen, so dass die Beschleunigungen eher zu hoch als zu tief liegen. Folglich ist es wahrscheinlich, dass durch eine Überarbeitung des Lastenheftes und der genaueren Bestimmung der kritischen Beschleunigungen, tiefere Werte bestimmt werden können und der Solar Butterfly exakter für die auftretenden Belastungen ausgelegt werden kann.\\
Sofern dem Projekt genügen Ressourcen und Zeit zur Verfügung stehen und sich dafür entschieden wird das Lastenheft zu überarbeiten, wird empfohlen, den Solar Butterfly als Feder-Dämpfer-Modell zu Modellieren und die Beschleunigungen für verschiede Bedingungen und Situationen zu ermitteln. Mit den gewonnenen Erkenntnissen könnte sich Gewissenheit bezüglich den bei der Ausarbeitung des Lastenheftes getroffenen Annahmen verschaffen werden und die Ungenauigkeit - und somit auch das Risiko - der Lastfälle minimiert werden. Dies würde eine präzisere und vorallem sichrere (weniger Unsicherheiten) Auslegung ermöglichen, was wiederrum mit einer eventuellen Reduktion des Gewichtes verbunden werden kann.\\
Es muss betont werden, dass das Lastenheft eine erste Abschätzung der Lasten darstellt und nicht als definitive oder abschliessende Beurteilung betrachtet werden soll.
Auch wenn die Aussagekraft des Lastenheftes hier zu einem gewissen Ausmass in frage gestellt wird, wird das Lastenheft als eine gute Einschätzung der Lasten erachtet und kann, unter Berücksichtigung der einhergehenden Risiken, für den weitere Verlauf des Projektes verwendet werden.

Den Fokus auf andere Punkte legen.

\subsection{Gebrauchsanleitung}
Ein weiterer Punkt, welcher als notwendig erachtet wird, ist die Ausarbeitung einer ``Gebrauchsanleitung'' in welcher der bedienenden Person klar beschrieben wird, wie der Solar Butterfly zu benützen ist, was ``erlaubt'' ist und was nicht und wie sich in spezifischen Situationen (z.B. bei Wind oder technischen Defekten) verhalten werden soll. Einige dieser Punkte sind im Lastenheft und der Anforderungsliste bereits definiert. Diese Dokumente sind in dieser hinsicht jedoch keines Wegs vollständig, decken nicht annähernd alle Szenarien ab und sind für die Funktion als Gebrauchsanleitung nicht passend strukturiert. Die Gebrauchsanleitung kann auch als eine Zustammenstellung von Informationen der verschiedenen Arbeiten wie dieser, welche für die Bedienung des Solar Butterflys relevant sind, betrachtet werden. Weiter kann mit einer Gebrauchsanleitung das Eintreten von gewissen Missbrauchslastfällen verhindert werden, da die Benützung engeschränkt wird.\\
Eine Gebrauchsanleitung müsste unter anderem folgende Fragestellungen beantworten können.
Was ist die maximale Zuladung und an welcher Stelle darf sich diese im Solar Butterfly befinden? (Schwerpunkt)
Was sind die erlaubten Windgeschiwndigkeiten für die unterschiedlichen Positionen in welchen sich der Solar Butterfly befinden kann? Wann darf zum Beispiel die Panelenreihe C oder D nicht mehr ausgefahren sein?
Wie können die Seitenmodule bei ausgefallener Pneumatik ein- und ausgefahren werden? Muss die Pneumatik unterhalten und gewartet werden?
Wie muss der Solar Butterfly im Fahr-Modus versperrt und befestigt werden dass während einer Fahrt keine Gefahren entstehen?

Eine Gebrauchsanleitung aus zu arbeiten ist an dieser Stelle des Projektes nicht dringend notwendig und auch schwierig zu bewerkstelligen, da viele Komponenten noch nicht definitig ausgewählt wurden. Zu einem späteren Zeitpunkt und weiter vortgeschrittenem Projektstand ist die Erstellung einer Gebrauchsanleitung jedoch empfehlenswert. Dies nicht nur um den Solar Butterfly sicherer zu machen, sondern auch, um dessen Bedienung zu vereinfachen. Dass der Solar Butterfly von verschiedenen Personen gefahren und bedient werden soll, spricht ebenfalls für das Erstellen einer Gebrauchsanleitung.

\subsection{Auslegung}

Noch nicht sehr weit.
ermüdung nochmals abchecken. Nachweis erbringen usw.
Dynamik?
Rechtliche Problemaitk? (Autobahn usw.) Nachweis erstellen.

\subsection{Gewichtsoptimierung}
Wie im Kapitel \ref{Deformation} beschrieben, wird das grösste Potential einer Gewichtsreduktion der Grundsturktur in der Optimierung des Chassis in Kombination mit dem Boden gesehen. Folglich wird empfohlen diese Optimierung vorzunehmen.\\
Die massgebenden Eigenschaften des Chassis sind dessen Festigkeit, sowie die Steifigkeit, welche entscheidend für das Fahrverhalten ist. Kriterien bezüglich diesen beiden Punkten müssten bei einer Optimierung in Erfahrung gebracht und eingehalten werden. Weiter müsste abgeklärt werden welche konstruktiven Optimierungen des Chassis und des Boden sich anbieten und von \emph{Geser Fahrzeugbau} und \emph{3A-Composites} umgesetzt werden können.


\newpage

\section{Fazit}
Grosses Projekt, viel kommunikation und austausch, Diskussionen welche hier nicht dokumentiert wurden. viel "Versteckte Arbeit"
Nicht sehr weit jedoch gutes Fundament für weiteres Vorgehen.

Nicht so weit wie erwartet, noch sehr viel zu tun und abzuklären.
Trotz Zeitdruck konzeptphase sauber Abschliessen da hier die Probleme am einfachsten gelöst werden können. Auf dem Papier.



\newpage

\section{Danksagung}
Ohne die Unterstützung folgender Personen wäre mir das Ausarbeiten der Bachelorarbeit in dieser Form nicht möglich gewesen. Dafür möchte ich an dieser Stelle meinen Dank aussprechen an:\\
Dejan Roman\v{c}uk, für die Betreuung und Unterstützung bei der Durchführung dieser Arbeit.\\
Louis Palmer für die Organisation und intensive Zusammenarbeit.
Mitstudenten für die Angenehme Zusammenarbeit
Betreuenden Dozenten NAMEN für ihre Unterstüztung
\newpage
