\section{Diskussion}
\label{Diskussion}

\subsection{Lastenheft}
Beim Durchführen der Handrechnungen und der Auswertung der FEM-Berechnungen hat sich gezeigt, dass die Lastfälle \emph{1.1 Vertikale Beschleunigung} und \emph{1.5 Rotatorische Beschleunigung} die stärksten Belastungen für den Solar Butterfly darstellen. Zugleich weisen diese beiden Lastfälle die grössten Risiken auf, da sie auf groben Annahmen beruhen und nur mit grösserem Aufwand exakter abgeschätzt werden können.\\
Bei der Festlegung der Lasten im Lastenheft wurde eine tendenziell konservative Position eingenommen, so dass die Beschleunigungen eher zu hoch als zu tief liegen. Folglich ist es wahrscheinlich, dass durch eine Überarbeitung des Lastenheftes und der genaueren Bestimmung der kritischen Beschleunigungen, tiefere Werte bestimmt werden können und der Solar Butterfly exakter für die auftretenden Belastungen ausgelegt werden kann.\\
Sofern dem Projekt genügen Ressourcen und Zeit zur Verfügung stehen und sich dafür entschieden wird das Lastenheft zu überarbeiten, wird empfohlen, den Solar Butterfly als Feder-Dämpfer-Modell zu Modellieren und die Beschleunigungen für verschiede Bedingungen und Situationen zu ermitteln. Mit den gewonnenen Erkenntnissen könnte sich Gewissheit bezüglich den bei der Ausarbeitung des Lastenheftes getroffenen Annahmen verschaffen werden und die Ungenauigkeit - und somit auch das Risiko - der Lastfälle minimiert werden. Dies würde eine präzisere und vor allem sicherere (in Form von weniger Unsicherheiten) Auslegung ermöglichen, was wiederrum mit einer eventuellen Reduktion des Gewichtes verbunden werden kann.\\
Es muss betont werden, dass das Lastenheft eine erste Abschätzung der Lasten darstellt und nicht als definitive oder abschliessende Beurteilung betrachtet werden soll.\\
Auch wenn die Aussagekraft des Lastenheftes hier zu einem gewissen Ausmass in Frage gestellt wird, wird das Lastenheft als eine gute Einschätzung der Lasten erachtet und kann, unter Berücksichtigung der einhergehenden Risiken, für den weitere Verlauf des Projektes verwendet werden.

% Die Überarbeitung des Lastenheftes wird daher nicht direkt empfohlen. Es ist zu erwarten, dass in anderen Bereichen des Projektes mehr Nutzen und Fortschritt erzielt werden können.

\subsection{Einschränkung der Benutzung - Benutzerhandbuch}
Bei der Ausarbeitung des Lastenheftes und Auslegung der Komponenten wurde sich mit verschiedenen Anwenungs- und Missbrauchslastfällen auseinandergesetzt. Da der Solar Butterfly ein relativ komplexes Produkt ist und sich viele verschiedene Szenarien und Situationen ergeben können, ist die Anzahl an Fällen entsprechend gross. Eine Auslegung geht mit dem Treffen von Annahmen und der Definierung der Anwendungsfälle einher, gegenüber welchen ein Bauteil ausgelegt werden soll. Nur wenn die Anwendung definiert und eingeschränkt ist, kann eine sinnvolle und optimale Auslegung erfolgen.
Was daher als notwendig erachtet wird, ist die Ausarbeitung eines "Benutzerhandbuches" in welchem der bedienenden Person klar beschrieben wird, wie der Solar Butterfly zu benutzen ist, was "erlaubt" ist und wie sich in spezifischen Situationen (z.B. bei Wind oder technischen Defekten) verhalten werden soll.\\
Mit diesem Benutzerhandbuch kann der Gebrauch des Solar Butterflys eingeschränkt und das Eintreten von ungünstigen Situationen oder Missbrauchslastfällen verhindert werden.\\
So könnte zum Beispiel beschrieben werden wo und wie eine zusätzliche Beladung (Gepäck, Kühlbox u. dgl.) im Solar Butterfly platziert und befestigt werden muss, um das Eintreten von ungünstige Lastfällen (z.B. durch Verlagerung des Schwerpunktes) verhindern zu können.\\
Einige dieser Punkte sind im Lastenheft und der Anforderungsliste bereits definiert. Diese Dokumente sind in dieser Hinsicht jedoch keineswegs  vollständig, decken nicht alle Szenarien ab und sind für die Funktion als Benutzerhandbuch nicht passend strukturiert.\\
Das Benutzerhandbuch kann auch als eine Zusammenstellung von Informationen der verschiedenen Arbeiten wie dieser, welche für die Bedienung und Benutzung des Solar Butterflys relevant sind, betrachtet werden.
Als Beispiel hierzu könnte beschrieben werden, wie die pneumatischen Komponenten gewartet und unterhalten werden müssen, dass die in der Auslegung getroffenen Annahmen im Betrieb noch immer zutreffen und die Funktionstauglichkeit gewährleisten werden kann.

Ein Benutzerhandbuch auszuarbeiten ist an dieser Stelle des Projektes nicht dringend notwendig und auch schwierig zu bewerkstelligen, da viele Komponenten noch nicht definitiv ausgewählt wurden. Zu einem späteren Zeitpunkt und weiter vorgeschrittenem Projektstand ist die Erstellung eines Benutzerhandbuchs jedoch empfehlenswert. Dies nicht nur um den Solar Butterfly sicherer zu machen, sondern auch, um dessen Bedienung und Handhabung zu vereinfachen. Dass der Solar Butterfly von verschiedenen Personen gefahren und bedient werden soll, spricht ebenfalls für das Erstellen eines Benutzerhandbuches.

% Eine Gebrauchsanleitung müsste unter anderem folgende Fragestellungen beantworten können.
% Was ist die maximale Zuladung und an welcher Stelle darf sich diese im Solar Butterfly befinden? (Schwerpunkt)
% Was sind die erlaubten Windgeschiwndigkeiten für die unterschiedlichen Positionen in welchen sich der Solar Butterfly befinden kann? Wann darf zum Beispiel die Panelenreihe C oder D nicht mehr ausgefahren sein?
% Wie können die seitlichen Raumelemente bei ausgefallener Pneumatik ein- und ausgefahren werden? Muss die Pneumatik unterhalten und gewartet werden?
% Wie muss der Solar Butterfly im Fahr-Modus versperrt und befestigt werden dass während einer Fahrt keine Gefahren entstehen?

\subsection{Auslegung}
Das Projekt \emph{Solar Butterfly} befindet sich zur Zeit noch in der Konzeptphase in welcher prinzipielle Lösungen zu auftretenden Problemen ausgearbeitet werden. Im Verlaufe der Arbeit sind häufiger grundlegende Veränderungen des Konzeptes eingetreten, da neue Möglichkeiten ein Problem zu lösen in Erfahrung gebracht oder entdeckt wurden. Diese Tatsache begrenzt und erschwert eine Auslegung der Komponenten da sie auf vielen und unischeren Annahmen beruht. Es wird davon ausgegangen, dass solche Veränderungen noch weiterhin auftreten werden.\\
So ist zum Beispiel das Konzept zur Versperrung der seitlichen Raumelemente noch nicht ausgearbeitet worden. Dies birgt die Gefahr, dass es in Zukunft zu einer grundlegenderen Veränderung des bereits ausgearbeiteten Konzeptes kommt und dass dieses erneut überarbeitet werden muss. Würde beispielsweise der Boden in der Versperrung eine wichtige Funktion übernehmen, müsste dessen Auslegung erneut durchgeführt und das CAD-Modell angepasst werden.\\
Die Aufgaben und Funktionen, welche die einzelnen Komponenten des Solar Butterflys übernehmen und erfüllen müssen, sollen vor einer Auslegung prinzipiell und abschliessend definiert werden. Solange dies nicht der Fall ist, müssen getätigte Auslegung bei einer Änderung verworfen oder neu überarbeitet werden.

Daher wird empfohlen die Konzeptphase systematisch zu durchlaufen und abzuschliessen, das Konzept einzufrieren und erst im Anschluss mit der Ausarbeitung eines Entwurfes zu beginnen.\\
% Es ist klar, dass die Ausarbeitung eines Konzeptes mit dem Durchführen von Berechnungen verbunden ist, jedoch nicht in der selben Tiefe wie dies für eine Auslegung nötig ist.

Ferner wird erwähnt, dass nach einer Auslegung Gebrauchstauglichkeitsnachweise zu erbringen sind, welche nachweisen, dass ein Bauteil über die Lebensdauer und unter verschiedenen Belastungen funktionstauglich bleibt. Dies beinhaltet unter anderem die Berücksichtigung des Ermüdungsverhaltens, wozu eine genauere Betrachtung des dynamischen Verhaltens erforderlich ist, was in dieser Arbeit nicht vorgenommen wurde.
Ebenfalls sind einige Punkte bezüglich den Rechtlichen Grundlagen der Strassenzulassung oder der Schlusszertifizierung des Chassis noch nicht abgeklärt worden. Es ist nicht klar welche Nachweise von Gesetzes wegen her verlangt werden und zwingend erbracht werden müssen oder wer diese erbringen wird.

Mit diesen Aussagen soll betont werden, dass noch einige weitere Abklärungen und Schritte vor dem Beginn der Produktion nötig sind, um einen sicheren Solar Butterfly zu erschaffen und den Erfolg des Projektes zu erzielen.

\subsection{Gewichtsoptimierung}
Wie im Kapitel \ref{Deformation} beschrieben, wird das grösste Potential einer Gewichtsreduktion der Grundstruktur in der Optimierung des Chassis in Kombination mit dem Boden gesehen. Folglich wird empfohlen diese Optimierung vorzunehmen.\\
Die massgebenden Eigenschaften des Chassis sind dessen Festigkeit, sowie die Steifigkeit, welche entscheidend für das Fahrverhalten ist. Kriterien bezüglich dieser beiden Punkten müssten bei einer Optimierung in Erfahrung gebracht und eingehalten werden. Weiter müsste abgeklärt werden welche konstruktiven Optimierungen des Chassis und des Boden sich anbieten und von \emph{Geser Fahrzeugbau} und \emph{3A-Composites} umgesetzt werden können.\\
In der selben Optimierung könnte die kritische Klebeverbindung zwischen Boden und Chassis erneut untersucht und ausgelegt werden.

\newpage

\section{Fazit}
Das Projekt \emph{Solar Butterfly} hat sich als grösser und intensiver (auf eine gute Art und Weise) herausgestellt als erwartet. Besonders die Menge an Kommunikation, Organisation und Austausch wurde unterschätzt, was durch eine angenehme Zusammenarbeit jedoch sehr erleichtert wurde.\\
Auch wenn im Lastenheft noch einige risiken in Form von Annahmen enthalten sind, wurde mit dem Lastenheft eine solide Basis für weitere Auslegungen geschaffen.


Lastenheft gut auch wenn viele Unsicherheiten
Handrechnungen: gute erste Abschätzung der Lasten, FEM dennoch nötig für exaktere Ergebnisse
Auslegung basierend auf sehr vielen Annahmen unter anderem auch wegen unausgearbeitetem konzept.
FEM nur grobes FEM möglich, dennoch aussagekräftig.


Nicht sehr weit jedoch gutes Fundament für weiteres Vorgehen.
Nicht so weit wie erwartet, noch sehr viel zu tun und abzuklären.

Trotz Zeitdruck konzeptphase sauber Abschliessen da hier die Probleme am einfachsten gelöst werden können. Auf dem Papier.
Mangelnde Führung und Organisation


\newpage

% \section{Danksagung}
% Ohne die Unterstützung folgender Personen wäre mir das Ausarbeiten der Bachelorarbeit in dieser Form nicht möglich gewesen. Dafür möchte ich an dieser Stelle meinen Dank aussprechen an:
%
% Dejan Roman\v{c}uk, für die Betreuung und Unterstützung bei der Durchführung dieser Arbeit,\\
% Louis Palmer für die Organisation und intensive und produktive Zusammenarbeit,\\
% den Mitstudenten Michael Huber, Dominic Bacher und Yannick Buholzer für die angenehme und kurzweilige Teamarbeit und\\
% den weiteren betreuenden Dozenten Pierre Kirchhofer, Rolf Kamps und Johann Lodewyks für ihre Unterstützung und Inputs.
% \newpage
