\section{Diskussion}
\label{Diskussion}

\subsection{Lastenheft}
Beim Durchführen der Handrechnungen und der Auswertung der FEM-Berechnungen hat sich gezeigt, dass die Lastfälle \emph{1.1 Vertikale Beschleunigung} und \emph{1.5 Rotatorische Beschleunigung} die stärksten Belastungen für den Solar Butterfly darstellen. Zugleich weisen diese beiden Lastfälle die grössten Risiken auf, da sie auf groben Annahmen beruhen und nur mit grösserem Aufwand exakter abgeschätzt werden können.\\
Bei der Festlegung der Lasten im Lastenheft wurde eine tendenziell konservative Position eingenommen, so dass die Beschleunigungen eher zu hoch als zu tief liegen. Folglich ist es wahrscheinlich, dass durch eine Überarbeitung des Lastenheftes und der genaueren Bestimmung der kritischen Beschleunigungen, tiefere Werte bestimmt werden können und der Solar Butterfly exakter für die auftretenden Belastungen ausgelegt werden kann.\\
Sofern dem Projekt genügen Ressourcen und Zeit zur Verfügung stehen und sich dafür entschieden wird das Lastenheft zu überarbeiten, wird empfohlen, den Solar Butterfly als Feder-Dämpfer-Modell zu Modellieren und die Beschleunigungen für verschiede Bedingungen und Situationen zu ermitteln. Mit den gewonnenen Erkenntnissen könnte sich Gewissheit bezüglich den bei der Ausarbeitung des Lastenheftes getroffenen Annahmen verschaffen werden und die Ungenauigkeit - und somit auch das Risiko - der Lastfälle minimiert werden. Dies würde eine präzisere und vor allem sicherere (in Form von weniger Unsicherheiten) Auslegung ermöglichen, was wiederrum mit einer eventuellen Reduktion des Gewichtes verbunden werden kann.\\
Es muss betont werden, dass das Lastenheft eine erste Abschätzung der Lasten darstellt und nicht als definitive oder abschliessende Beurteilung betrachtet werden soll.\\
Auch wenn die Aussagekraft des Lastenheftes hier zu einem gewissen Ausmass in Frage gestellt wird, wird das Lastenheft als eine gute Einschätzung der Lasten erachtet und kann, unter Berücksichtigung der einhergehenden Risiken, für den weitere Verlauf des Projektes verwendet werden.

Die Überarbeitung des Lastenheftes wird daher nicht direkt empfohlen. Es ist zu erwarten, dass in anderen Bereichen des Projektes mehr Nutzen und Fortschritt erzielt werden können.

\subsection{Benutzerhandbuch}
Bei der Ausarbeitung des Lastenheftes und Auslegung der Komponenten wurde sich mit verschiedenen Anwenungs- und Missbrauchslastfällen auseinandergesetzt. Da der Solar Butterfly ein relativ komplexes Produkt ist und sich viele verschiedene Positionen und Situationen ergeben können, ist die Anzahl an Fällen entsprechend gross. Eine Auslegung geht mit der definition der Anwendungsfälle einher. Nur wenn die Anwendung eingeschränkt ist, kann eine sinvolle Auslegung erfolgen.
Was daher als notwendig erachtet wird, ist die Ausarbeitung eines ``Benutzerhandbuches'' in welchem der bedienenden Person klar beschrieben wird, wie der Solar Butterfly zu benützen ist, was ``erlaubt'' ist und wie sich in spezifischen Situationen (z.B. bei Wind oder technischen Defekten) verhalten werden soll.\\
Mit diesem Benutzerhandbuch kann der Gebrauch des Solar Butterflys eingeschränkt und das eintreten von ungünstigen Situationen, Positionen oder Missbrauchslastfällen vermieden werden.
Einige dieser Punkte sind im Lastenheft und der Anforderungsliste bereits definiert. Diese Dokumente sind in dieser hinsicht jedoch keines Wegs vollständig, decken nicht annähernd alle Szenarien ab und sind für die Funktion als Benutzerhandbuch nicht passend strukturiert.\\
Das Benutzerhandbuch kann auch als eine Zustammenstellung von Informationen der verschiedenen Arbeiten wie dieser, welche für die Bedienung des Solar Butterflys relevant sind, betrachtet werden. Sie stellt also ein Bindeglied zwischen dem Benützer und den Anwendungsdetails in den technischen Dokumentationen dar.

Ein Benutzerhandbuch aus zu arbeiten ist an dieser Stelle des Projektes nicht dringend notwendig und auch schwierig zu bewerkstelligen, da viele Komponenten noch nicht definitiv ausgewählt wurden. Zu einem späteren Zeitpunkt und weiter vortgeschrittenem Projektstand ist die Erstellung eines Benutzerhandbuchs jedoch empfehlenswert. Dies nicht nur um den Solar Butterfly sicherer zu machen, sondern auch, um dessen Bedienung und Handhabung zu vereinfachen. Dass der Solar Butterfly von verschiedenen Personen gefahren und bedient werden soll, spricht ebenfalls für das Erstellen eines Benutzerhandbuches.

% Eine Gebrauchsanleitung müsste unter anderem folgende Fragestellungen beantworten können.
% Was ist die maximale Zuladung und an welcher Stelle darf sich diese im Solar Butterfly befinden? (Schwerpunkt)
% Was sind die erlaubten Windgeschiwndigkeiten für die unterschiedlichen Positionen in welchen sich der Solar Butterfly befinden kann? Wann darf zum Beispiel die Panelenreihe C oder D nicht mehr ausgefahren sein?
% Wie können die seitlichen Raumelemente bei ausgefallener Pneumatik ein- und ausgefahren werden? Muss die Pneumatik unterhalten und gewartet werden?
% Wie muss der Solar Butterfly im Fahr-Modus versperrt und befestigt werden dass während einer Fahrt keine Gefahren entstehen?

\subsection{Auslegung}
Das Projekt befindet sich zur Zeit noch in der Konzeptphase in welcher prinzipielle Lösungen zu auftetenden Problemen ausgearbeitet werden. Im Verlaufe der Arbeit sind häufiger grundlegende Veränderungen des Konzeptes eingetreten, da neue Möglichkeiten ein Problem zu lösen in Erfahrung gebracht oder entdeckt wurden. Diese Tatsache begrenzt und erschwert eine Auslegung der Komponenten da sie auf vielen und unischeren Annahmen beruht.\\

So ist zum Beispiel das Konzept zur Versperrung der seitlichen Raumelemente noch nicht ausgearbeitet worden. Dies birgt die Gefahr, dass es in Zukunft zu einer grundlegenderen Veränderung des bereits ausgearbeiteten Konzeptes kommt und dass dieses erneut überarbeitet werden muss.

Daher wird empfohlen die Konzeptphase systematisch zu durchlaufen und abzuschliessen, das Konzept einzufrieren und erst im Anschluss mit der Ausarbeitung eines Entwurfes zu beginen.\\

Ferner wird erwähnt, dass nach einer Auslegung Gebrauchstauglichkeitsnachweise zu erbringen sind, welche nachweisen, dass ein Bauteil über die Lebensdauer und unter verschiedenen Belastungen funktionstauglich bleibt. Dies beinhaltet unter anderem die Berücksichtigung des Ermüdungsverhaltens, wozu eine genauere Betrachtung des dynamischen Verhaltens erforderlich ist, was in dieser Arbeit nicht vorgenommen wurde.
Ebenfalls sind einige Punkte bezüglich den Rechtlichen Grundlagen der Strassenzulassung oder der Schlusszertifizierung des Chassis nocht nicht abgeklärt worden. Es ist nicht klar welche Nachweise von Gesetzes wegen her verlangt werden und zwingend erbracht werden müssen.

Mit diesen Aussagen soll betont werden, dass noch einige weitere Schritte und Abklärungen nötig sind um einen sicheren Solar Butterfly zu erschaffen und den Erfolg des Projektes zu erzielen.


\subsection{Gewichtsoptimierung}
Wie im Kapitel \ref{Deformation} beschrieben, wird das grösste Potential einer Gewichtsreduktion der Grundsturktur in der Optimierung des Chassis in Kombination mit dem Boden gesehen. Folglich wird empfohlen diese Optimierung vorzunehmen.\\
Die massgebenden Eigenschaften des Chassis sind dessen Festigkeit, sowie die Steifigkeit, welche entscheidend für das Fahrverhalten ist. Kriterien bezüglich diesen beiden Punkten müssten bei einer Optimierung in Erfahrung gebracht und eingehalten werden. Weiter müsste abgeklärt werden welche konstruktiven Optimierungen des Chassis und des Boden sich anbieten und von \emph{Geser Fahrzeugbau} und \emph{3A-Composites} umgesetzt werden können.\\
In der selben Optimierung könnte die kritische Klebeverbindung erneut untersucht und passend ausgelegt werden.

\newpage

\section{Fazit}
Auslegung wurde einfacher erwartet. Klarere Problematik, einfachere Lösung.
Abhängig von der Konstruktion welche erst im verlaufe des Projektes ausgearbeitet wurde.

Grosses Projekt, viel kommunikation und austausch, Diskussionen welche hier nicht dokumentiert wurden. viel "Versteckte Arbeit"
Nicht sehr weit jedoch gutes Fundament für weiteres Vorgehen.

Nicht so weit wie erwartet, noch sehr viel zu tun und abzuklären.
Trotz Zeitdruck konzeptphase sauber Abschliessen da hier die Probleme am einfachsten gelöst werden können. Auf dem Papier.



\newpage

\section{Danksagung}
Ohne die Unterstützung folgender Personen wäre mir das Ausarbeiten der Bachelorarbeit in dieser Form nicht möglich gewesen. Dafür möchte ich an dieser Stelle meinen Dank aussprechen an:\\
Dejan Roman\v{c}uk, für die Betreuung und Unterstützung bei der Durchführung dieser Arbeit.\\
Louis Palmer für die Organisation und intensive Zusammenarbeit.
Mitstudenten für die Angenehme Zusammenarbeit
Betreuenden Dozenten NAMEN für ihre Unterstüztung
\newpage
