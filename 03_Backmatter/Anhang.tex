\section{Rissfortschritt}
\let\cleardoublepage\clearpage
\subsection{Zeichnungen}
  \label{Zeichnungen}
  \newpage

  \subsubsection{Zeichnung des Probenrohlings - Erste Serie}
    \label{Zeichnung der Probe - Erste Serie}
    \includegraphics[scale = 0.75]{../Anhang/D_Rissfortschritt/D.1_Zeichnungen/D.1.1_Probenrohling_1.pdf}

  \subsubsection{Zeichnung des Probenrohlings - Zweite Serie}
    \label{Zeichnung der Probe - Zweite Serie}
    \includegraphics[scale = 0.75]{../Anhang/D_Rissfortschritt/D.1_Zeichnungen/D.1.2_Probenrohling_2.pdf}

  \subsubsection{Zeichnung der bearbeiteten Probe - Erste Serie}
    \label{Zeichnung des Probenrohlings - Erste Serie}
    \includegraphics[scale = 0.75]{../Anhang/D_Rissfortschritt/D.1_Zeichnungen/D.1.3_Probe bearbeitet_1.pdf}

  \subsubsection{Zeichnung der bearbeiteten Probe - Zweite Serie}
    \label{Zeichnung des Probenrohlings - Zweite Serie}
    \includegraphics[scale = 0.75]{../Anhang/D_Rissfortschritt/D.1_Zeichnungen/D.1.4_Probe bearbeitet_2.pdf}

  \subsection{Anordnung der Proben im Drucker}
  \label{Anordnung der Proben im Drucker}
    \begin{figure}[!ht]
      \centering
        \begin{subfigure}{.5\textwidth}
          \centering
          \includegraphics[width=0.95\textwidth]{04_Figures/02_Rissfortschritt/Probenanordnung 1 Beschnitten.jpg}
          \caption{Erste Serie}
          \label{Erste Serie}
        \end{subfigure}%
        \begin{subfigure}{.5\textwidth}
          \centering
          \includegraphics[width=0.95\textwidth]{04_Figures/02_Rissfortschritt/Probenanordnung 2 Beschnitten.jpg}
          \caption{Zweite Serie}
          \label{Zweite Serie}
        \end{subfigure}
    \end{figure}

  \subsection{Zerstörte Proben der ersten Serie}
    \label{Zerstörte Proben der ersten Serie}
    \begin{center}
      \includegraphics[width=0.5\textwidth]{04_Figures/02_Rissfortschritt/Zerstörte Proben der ersten Serie.jpeg}
    \end{center}

\section{Korrelation Dauerfestigkeit und CT-Scan}
  \subsection{Zeichnung der Probenhalterung}
    \label{Zeichnung der Probenhalterung}
    \includegraphics[scale = 0.71]{../Anhang/E_Korrelation Dauerfestigkeit/E.1_Probenhalterung/E.1_Probenhalterung.pdf}


  \subsection{Analyse-Einstellungen}
    \label{Analyse-Einstellungen}
    \begin{center}
      \includegraphics[width=\textwidth]{04_Figures/03_Korrelation Dauerfestigkeit und CT-Scans/Einstellung finale Auswertung.JPG}
    \end{center}

  \subsection{Beschreibung des Python Skriptes}
  \label{Beschreibung des Python Scriptes}

  \input{03_Backmatter/pythonDescription}

  \subsection{Histogramm der Kompaktheit}
    \label{Histogramm der Kompaktheit}
    \begin{center}
      \includegraphics[width=0.8\textwidth]{04_Figures/03_Korrelation Dauerfestigkeit und CT-Scans/H-Compactness.png}
    \end{center}

  \subsection{1\% Pore}
    \subsubsection{Weibullparameter und 1\% Poren}
    \label{Weibullparameter und 1Poren}
      \begingroup
      \renewcommand{\arraystretch}{1.5}
      %\begin{table}[ht]
      \centering
      \begin{tabularx}{0.95\textwidth}{Xccc|ccc}
      \thickhline
            & \multicolumn{3}{c}{\textbf{Mit Volumen}}      & \multicolumn{3}{c}{\textbf{Mit projizierter Fläche}} \\ \thickhline
      Probe & $\lambda$     & \(k\)         & Volumen [mm³] & $\lambda$     & \(k\)         & Fläche [mm²] \\ \thickhline
      3.2   & 6.482e-01     & 1.371e-05     & 2.040e-04     & 1.057e+00     & 1.208e-03     & 4.995e-03 \\ \hline
      3.3   & 6.913e-01     & 6.077e-06     & 7.260e-05     & 1.082e+00     & 7.307e-04     & 2.889e-03 \\ \hline
      3.6   & 6.674e-01     & 8.777e-06     & 1.180e-04     & 1.070e+00     & 8.892e-04     & 3.591e-03 \\ \hline
      3.7   & 6.502e-01     & 8.737e-06     & 1.285e-04     & 1.035e+00     & 1.127e-03     & 4.847e-03 \\ \hline
      3.8   & 6.795e-01     & 7.759e-06     & 9.820e-05     & 1.079e+00     & 7.682e-04     & 3.054e-03 \\ \hline
      3.11  & 6.970e-01     & 9.425e-06     & 1.098e-04     & 1.111e+00     & 9.151e-04     & 3.453e-03 \\ \hline
      3.12  & 6.812e-01     & 8.513e-06     & 1.068e-04     & 1.087e+00     & 8.732e-04     & 3.425e-03 \\ \hline
      3.13  & 6.596e-01     & 9.737e-06     & 1.363e-04     & 1.058e+00     & 1.084e-03     & 4.469e-03 \\ \hline
      3.18  & 6.888e-01     & 1.090e-05     & 1.319e-04     & 1.110e+00     & 1.034e-03     & 3.908e-03 \\ \hline
      3.19  & 6.663e-01     & 9.131e-06     & 1.234e-04     & 1.066e+00     & 9.258e-04     & 3.767e-03 \\ \hline
      \end{tabularx}
      %\end{table}
      \endgroup

  \subsubsection{Weibullverteilung mit der projizierten Fläche}
    \label{Weibullverteilung mit der projizierten Fläche}
    \begin{center}
      \includegraphics[width=0.8\textwidth]{04_Figures/03_Korrelation Dauerfestigkeit und CT-Scans/W-ProjZ1.png}
    \end{center}

  \subsubsection{Weibullverteilung der Proben 3.2 und 3.3 mit dem Volumen}
    \label{Weibullverteilung der Proben 3.2 und 3.3}
    \begin{center}
      \includegraphics[width=0.8\textwidth]{04_Figures/03_Korrelation Dauerfestigkeit und CT-Scans/W-3.2u3.3.png}
    \end{center}

  \subsubsection{\(\Delta\)K der 1\%Poren, welche mit den Volumen der Poren berechnet wurden, in Abhängigkeit deren Lastspielzahlen}
    \label{K vs N Volume1}
    \begin{center}
      \includegraphics[width=0.8\textwidth]{04_Figures/03_Korrelation Dauerfestigkeit und CT-Scans/Korrelation_Volumen.png}
    \end{center}

  \subsubsection{1\% Flächen und $\Delta$K berechnet aus den projizierten Flächen der Poren}
  \label{1pP mit Projizierter Fläche}
    \begin{center}
    \begin{tabular}{ccccc}
    \thickhline
    Lastniveau {[}MPa{]}  & Probe & Lastzyklen & 1\% Fläche {[}mm{]} & Delta K \\ \thickhline
    \multirow{4}{*}{300}  & 3.3   & 1957621    & 2.89E-03            & 4.959   \\
                          & 3.13  & 1589421    & 4.47E-03            & 5.530   \\
                          & 3.7   & 1169075    & 4.85E-03            & 5.643   \\
                          & 3.19  & 553603     & 3.77E-03            & 5.299   \\ \hline
    \multirow{6}{*}{280}  & 3.8   & 3200000    & 3.05E-03            & 5.028   \\
                          & 3.12  & 3000000    & 3.42E-03            & 5.174   \\
                          & 3.11  & 2860444    & 3.45E-03            & 5.185   \\
                          & 3.2   & 2780208    & 5.00E-03            & 5.686   \\
                          & 3.18  & 1938997    & 3.91E-03            & 5.348   \\
                          & 3.6   & 1526975    & 3.59E-03            & 5.236   \\\hline
    \end{tabular}
    \end{center}

  \subsubsection{1\% Flächen und $\Delta$K berechnet aus den Volumen der Poren}
  \label{1pP mit Volumen}
    \begin{center}
    \begin{tabular}{ccccc}
    \thickhline
    Lastniveau {[}MPa{]} & Probe & Lastzyklen & 1\% Fläche {[}mm{]} & Delta K \\ \thickhline
    \multirow{4}{*}{300} & 3.3   & 1957621    & 2.10E-03            & 4.581   \\
                         & 3.13  & 1589421    & 3.20E-03            & 5.088   \\
                         & 3.7   & 1169075    & 3.08E-03            & 5.038   \\
                         & 3.19  & 553603     & 3.00E-03            & 5.004   \\ \hline
    \multirow{6}{*}{280} & 3.8   & 3200000    & 2.57E-03            & 4.817   \\
                         & 3.12  & 3000000    & 2.72E-03            & 4.885   \\
                         & 3.11  & 2860444    & 2.77E-03            & 4.908   \\
                         & 3.2   & 2780208    & 4.19E-03            & 5.442   \\
                         & 3.18  & 1938997    & 3.13E-03            & 5.060   \\
                         & 3.6   & 1526975    & 2.91E-03            & 4.967   \\ \hline
    \end{tabular}
    \end{center}
